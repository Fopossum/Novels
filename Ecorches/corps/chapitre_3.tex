\chapter{Échappé des Enfers}

Je sentis instantanément le malaise. Alors qu'il était déjà pâle, Diaz blêmit encore un peu plus.
La simple évocation de son réveil semblait le secouer encore plus.

« Allez-y doucement. Prenez le temps qu'il vous faut. Vous voulez un verre d'eau ?

— Non merci. Ça va aller. C'est juste que… J'ai peur que vous ne me croyez pas. En y réfléchissant bien, après ce que
je vais vous dire, vous allez me caser avec les doux dingues qui racontent s'être fait enlevés par des extraterrestres
pour des expériences bizarres. »

Je tiquais. Je n'avais pas franchement envisagé les choses sous cet angle. Instantanément me vint à l'esprit l'image de
petits gris tenant à la main une sonde se penchant sur Diaz. Je ne pus retenir un sourire et je tentais vainement de
chasser cette pensée en le rassurant :

« Non, pas du tout. Je suis déjà au courant des grandes lignes. Allez-y, je vous crois.

— Je vois que ça vous fait sourire.

— Disons que d'imaginer un petit gris à vos côté entre deux carcasses de vache autopsiées est assez… Comique n'est-ce
pas ?

— Oui, vous avez raison. Il reprit quelques couleurs.

— Donc, vous vous réveillez et… »

Je laissais la phrase en suspend pour lui repasser la main. Il ne fallait pas que je donne l'impression d'orienter son
témoignage. Ce qu'il avait à m'apprendre devait être dit avec ses mots. Surtout pas les miens. Je fis mine de noter
quelque chose, mais plus pour me donner une contenance qu'autre chose. J'avais du mal à oublier le petit gris. Je
n'espérais qu'une chose, c'est que ces images mentales ne se retrouveraient pas sur les enregistrements de l'implant.
Parce que sinon, les collègues allaient se payer une bonne tranche de rigolade en visionnant la déposition.

« D'abord j'ai eu du mal à voir autour de moi. Une très forte lumière m'éblouissait. Vous savez, comme celle chez le
dentiste où dans les salles d'opération ?

— Oui, j'imagine assez bien, continuez lui dis-je en hochant la tête.

— J'avais froid et j'étais attaché. J'étais couché sur une surface métallique et j'avais des sangles aux chevilles et
aux poignets. J'ai d'abord cru que je faisais un cauchemar et je me suis dit que j'allais vite me réveiller. Mais comme
le temps passait, il a bien fallu que je me rende à l'évidence. J'étais à poil sanglé à une table. Après mes yeux se
sont habitués à la lumière, j'ai pu voir que j'étais dans une grande pièce. Blanche, immaculée. Vraiment comme une
salle d'opération. »

Il reprit son souffle et se servit un verre d'eau. J'en profitais pour noter rapidement \emph{Salle d'op' ? Médecin ?}
Cela paraissait un peu irréel. Le petit gris me regarda avec un sourire narquois.

« Une salle d'opération vous dites ?

— Oui. Des carreaux blancs partout. Et cette énorme lampe au dessus de moi. Il ne manquait que les instruments au
tableau.

— Qu'avez-vous fait ensuite ?

— J'ai commencé par crier. Fort et longtemps. Et je crois bien m'être pissé dessus de trouille.

— Personne n'est venu ?

— Non. Pas un bruit. Rien. J'ai attendu. Je ne sais pas combien de temps. Ça m'a paru infini. J'ai essayé de bouger mes
bras. Le droit était solidement attaché mais la sangle gauche était un peu plus lâche. En forçant j'ai réussi à
extraire ma main. À partir de là, j'ai pu me détacher l'autre main et les jambes. Je me suis redressé… — Sa voix baissa
d'un ton — Et c'est là que j'ai vu que j'étais allongé sur une table d'autopsie ou d'opération. Comme on en voit à la
télé dans les séries policières. Sauf que là, c'était réel et bien tangible. »

Je continuais à noter frénétiquement. J'imaginais assez bien l'horreur qu'il avait dû ressentir en se réveillant dans
cette ambiance. Tout à fait digne d'un scénario de film ou de série avec un tueur psychopathe. À la différence, comme
le disait Diaz, que là, c'était foutrement réel. Je frissonnais légèrement. On a beau être flic et endurci, ce genre de
récit garde un côté assez terrifiant.

« Continuez, je vous crois. Une image mentale très précise de ce que me décrivait Diaz s'esquissait.

— Je me suis levé et j'ai regardé partout dans la pièce. Et c'est là que j'ai vu les instruments. Ils n'étaient pas
dans mon champ de vision quand j'étais allongé. Ils reposaient sur une desserte dans un coin de la pièce. J'ai attrapé
un scalpel et j'ai cherché si il y avait autre chose qui pourrait me servir, mais rien de plus.

— Qu'avez-vous fait du scalpel ? le rapport n'indique pas qu'il était en votre possession lorsque l'on vous a retrouvé.

— Je l'ai perdu dans la forêt durant ma fuite.

— D'accord. Poursuivez.

— Je me suis dirigé vers la porte. Elle était déverrouillée. Je l'ai ouverte lentement. Elle donnait sur un couloir
dont la lumière était éteinte. J'ai cherché un interrupteur mais je n'en ai pas trouvé. La lumière venant de la
salle d'opération suffisait à me permettre de voir un peu. J'ai suivi le couloir sur quelque mètres jusqu'à un
escalier. J'ai monté quelques marches, je dirais une quinzaine au plus et je me suis retrouvé face à une autre porte.
La lumière de la salle ne suffisait plus à bien m'éclairer et j'ai failli tomber en l'atteignant.

— Est-ce que vous vous souvenez d'une décoration particulière dans le couloir ? D'autres portes ?

— Je vous avoue que non. Par contre, je crois avoir vu deux autres portes donner sur le couloir mais je n'avais aucune
envie d'explorer chaque pièce. Je voulais juste me tirer vite fait.

— Je comprends. \emph{D'autres portes. Cave étendue.}

— J'ai ouvert doucement la porte et je me suis retrouvé dans une espèce de grange. Et toujours rien à l'horizon pour me
couvrir. Je me souviens par contre très bien de l'odeur. Je ne pourrais pas vous la décrire exactement mais on aurait
dit des produits chimiques. C'était acide et ça piquait le nez. Pas de lumière non plus. Mais j'ai pu remarquer une
légère clarté qui dessinait une autre porte. Je me suis dirigé vers elle. Je sentais de la paille sous mes pieds. La
porte était en bois. Et pas verrouillée. Je l'ai tirée et je me suis retrouvé dans la campagne. La seule clarté venait
de la Lune et des étoiles.

— Avez-vous remarqué un autre bâtiment aux alentours ?

— Non, rien. J'ai vu une forêt un peu plus loin et je suis parti dans cette direction. Aussi vite que je le pouvais.
Avant d'arriver à la lisière, j'ai enjambé une clôture. Et j'ai récolté pas mal d'égratignures sur les barbelés. Une
fois dans les bois, j'ai continué ma route en essayant de me repérer aux étoiles. C'est à ce moment que j'ai regretté
de ne pas m'être plus intéressé à l'astronomie plus jeune au lycée. Vous y connaissez quelque choses aux étoiles vous ?

— Pas du tout, je crois que j'aurais été dans la même situation que vous — le petit gris ne souriait plus du tout — et
que je n'aurais pas fait mieux. C'est tout juste si j'arrive à trouver la Grande Ourse. Je lui adressait un regard
compréhensif.

— J'ai marché un certain temps qui m'a paru être une éternité et j'ai fini par m'adosser à un arbre. Je crois que je
n'ai jamais autant chialé de ma vie. Et je me suis endormi. Pas très longtemps puisque quand je me suis réveillé il
faisait nuit mais suffisamment pour perdre mon scalpel. Je me suis relevé et j'ai continué à marcher tant bien que mal.
Mes pieds me faisaient atrocement souffrir mais je n'avais que la fuite comme option en espérant tomber sur quelqu'un
qui pourrait m'aider. »

Il se servit un nouveau verre d'eau. Aussi dingue que pouvait être son histoire, il était crédible. Je sentais qu'il me
disait la vérité. Ça n'était définitivement pas des conneries. Il n'y avait aucune raison d'inventer tout ça. Et ça en
était d'autant plus effrayant.

« Au bout d'un moment, j'ai fini par apercevoir au loin les lumière de l'autoroute et je me suis dirigé vers elles. Je
n'ai jamais été aussi content d'être un piéton aux abords d'une autoroute que cette nuit. Très vite, un patrouiller
est arrivé et m'a embarqué.

— Il a sûrement été prévenu par les autres conducteurs qui venaient de vous voir en passant.

— Sûrement oui. Et j'imagine assez bien la tête qu'ils ont dû faire en me voyant dans le plus simple appareil en train
de tendre le pouce. »

Il se mit à rire. Et moi aussi d'ailleurs. Je dois avouer que j'aurais été surpris par un type à poil faisant du stop.

« Avez-vous quelque chose à ajouter à votre déclaration ?

— Oui, une chose. Essayez de coffrer le salaud qui m'a fait ça.

— C'est bien ce que je compte faire. Avez-vous un avocat ?

— Oui, celui qui s'occupe des contrats de mon groupe. Pourquoi ?

— Prenez dès aujourd'hui contact avec lui. Si nous attrapons ce type, vous en aurez besoin. »

Je me levais alors et sortis mon portefeuille pour attraper une carte de visite que je lui tendis :

« Si vous vous souvenez de quoique ce soit, même un détail, n'hésitez surtout pas à m'appeler.

— Merci lieutenant.

— Fin de la déposition de monsieur Diaz. Lieutenant Tersant. »

Une nouveau picotement dans la main m'indiqua que l'implant avait arrêté l'enregistrement. Je me dirigeais vers la
porte lorsque Diaz me lança :

« Lieutenant ? Vous croyais que vous allez le chopper ?

— Je vais tout faire pour. »

J'ouvris la porte et sortis dans le couloir. J'avais du pain sur la planche.