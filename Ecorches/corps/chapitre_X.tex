\chapter{Sans nom}

« Une vache dans un pré. Un ciel d'été. Une bibliothèque. Un tas de spaghettis ? Non ? Mais en réalité c'est quoi ce
truc là ? »

L'espèce de casque autour de ma tête bourdonnait. Les écrans placés devant mes yeux représentaient des animaux, des 
situations, des formes abstraites. Je devais les décrire avec précision, même si quelques fois cela relevait de la
gageure -- tentez donc de décrire un tableau abstrait fait de courbes de couleurs sans sens apparent.

« Essayez de rester concentré sur les planches M. Tersant. Si nous voulons que le diagnostic soit le plus précis
possible, il nous faut récupérer le maximum d'informations.

-- Vous êtes marrant. Vous me passez devant les yeux des choses quelques fois trop abstraites que je n'arrive même pas à
comprendre.

-- C'est tout à fait normal. Nous avons aussi besoin de ce genre de choses pour cartographier votre réseau neuronal et
comprendre si l'implant pose véritablement un problème ou si vos symptômes viennent d'ailleurs. »

J'étais déjà passé par la case scanner et IRM. Nous avions aussi testé les différentes fonctionnalités de l'implant,
enregistrement, transmisssion et rediffusion et tout s'était passé normalement. J'étais d'ailleurs étonné de la facilité
avec laquelle je m'y étais fait finalement. Normalement cet examen devait être le dernier que je devais subir avant 
d'avoir, je l'espérais, une réponse. Cela faisait trop longtemps que Bob -- j'avais fini par le baptiser comme ça -- me
tenait compagnie et me narguait de son petit sourire narquois. Il était quasiment tout le temps dans mon champ de
vision. Je ne trouvais du répit que lorsque je fermais les yeux.

J'étais en train de devenir dingue.

« Monsieur Tersant, nous en avons terminé. Nous allons très vite vous rappeler pour vous communiquer les résultats. »
L'infirmière retira le casque de ma tête en m'adressant un grand sourire. Mes yeux eurent du mal à s'habituer à la
lumière crue qui tombait du plafond et je sentis des larmes brouiller ma vision. Tout en me tendant un mouchoir elle me
dit « Ne vous en faites pas, c'est tout à fait normal. Vos yeux vont vite s'habituer. »

Tout en essuyant mes larmes j'observais à nouveau l'endroit où je me trouvais. Des murs pavés de carreaux d'un blanc
éclatant, des machines ronronnantes et bourdonnantes d'activité électrique, des écrans où se succédaient des indications
qui n'avaient pour moi aucun sens -- \emph{Si tu étais là mon amour, tu comprendrais tout ce qui s'affiche ici, tu
pourrais me dire ce qui ne va pas chez moi, tu pourrais me rassurer} -- toutes ces personnes qui virevoltaient d'une
machine à l'autre, caressant de leurs doigts gantés des claviers et des écrans. J'assistais à un ballet technologique à la
chorégraphie complexe qui se déroulait sous mes yeux encore embués.

Je descendis du fauteuil sur lequel j'avais été installé et me dirigeai vers la porte. Soudain, une image me traversa
l'esprit. Je me retournai et observait un peu plus la pièce. J'avais en face de moi la copie quasi conforme de la pièce
où devaient être maintenues les victimes dont les photos commençaient à remplir le dossier. Peut-être pas dans les
dimensions mais dans l'esprit. Il me fallait chercher si des hôpitaux ou des cliniques désaffectés se trouvaient dans la
région. Voire même des cliques vétérinaires ou des morgues.

La description de l'endroit qui nous avait été faire par Diaz, et le peu de décors présent sur les photos qui nous
étaient parvenues ne permettaient pas de deviner la taille de la pièce, mais, j'en aurais mis ma main à couper, il nous
fallait rajouter ces informations dans notre recherche géographique.

L'infirmière s'était tournée vers moi et m'interrogea du regard. « Non non, ce n'est rien, rien d'important. Merci et
bonne journée » répondis-je à son regard tout en ouvrant la porte de la pièce.

Dès que je fus sorti du bâtiment, j'attrapais mon téléphone et appelais Jensain. Il décrocha à la troisième sonnerie.

« Jensain, Tersant à l'appareil. Tu en es où de la délimitation de la zone géographique de recherche ?

-- Guère plus avancé qu'avant. Nous avons réduit légèrement la zone de recherche, mais rien de probant.

-- Dis-moi, tu pourrais essayer en rajoutant à ta recherche les hôpitaux, cliniques, cliniques vétérinaires et autres
établissements de pompes funèbres ? Je suis persuadé que c'est ce genre d'ancien bâtiment que nous recherchons. Ils ont
tous en commun des salles d'opération ou de \emph{préparation}. Et je pense que notre tueur n'a pas construit son local
de ses mains pas plus qu'il n'a dû s'adresser à des artisans locaux pour le faire construire. Il est resté discret
jusqu'à maintenant, aucune raison qu'il ne l'ait pas été avant.

-- Je m'y mets de suite. Au fait, le commissaire a demandé à ce que tu l'appelles dès que tu seras arrivé ici. Elle nous
a dit qu'il y avait du nouveau et est partie en trombe.

-- Noté. Je serai là dans une demie-heure environ. »
