\chapter[Chapitre Cinq]{Chapitre Cinq}
En attendant le retour de Gordy, je me plongeais dans les bases de données du FBI à la recherche de crimes similaires
ayant eu lieu ces dernières années dans la région. Des listes sans fins de crimes sordides défilaient sous mes yeux et
me donnaient la nausée. La violence dont les hommes pouvaient faire preuve envers leur prochain me désespérait.

Je passais deux heures à m'abreuver de la noirceur des âmes humaines. Les cadavres et les affaires non résolues
s'empilaient dans mon esprit. Je cherchais des liens. Rien. Rien ne correspondait à ce que nous avions vu dans cette
crypte sordide. Si ce groupe n'en était pas à sa première cérémonie, comme le laissaient supposer les traces de sang
plus anciennes, ils avaient été particulièrement discrets jusqu'à présent.

Pourquoi donc, des gens qui avaient réussi jusqu'à présent à passer totalement sous les radars des forces de l'ordre
avaient pu faire l'erreur aussi grossière de laisser leur victime crier avant de l'emmener à la mort. Cela ne collait
pas.

Je fus tiré de mes réflexions par un message de Gordy qui m'informait qu'il amenait le professeur Smitherson sur les
lieux du crimes pour qu'il nous éclaire de son savoir. Il l'amènerait ensuite au poste où nous pourrions lui montrer
les différents objets saisis sur les lieux.

Je profitais de cette pause pour aller me chercher un café dégueulasse à la machine de la salle de repos. Depuis des
années qui je travaillais ici, nous n'avions jamais réussi à avoir du café correct. Le jus de chaussette que mes
compatriotes osaient appeler du café n'avait rien à voir avec le délicieux breuvage auquel m'avait initié des années
plus tôt ma femme. On peut dire ce que l'on veut des Européens, mais eux au moins, ils savaient faire du café. Excepté
les Anglais peut-être\ldots{}

Je revins à mon bureau avec ma tasse et sorti de son tiroir l'étrange statuette. J'attrapai une paire de gants en latex
et découpais avec précaution l'étiquette des scellés pour extraire l'objet de emballage de plastique. Elle devait peser
environ un kilo mais paraissait étonnamment plus lourde. Elle dégageait une espèce d'aura malsaine qui semblait la
rendre plus lourde.

Je n'arrivais pas à en déterminer le matériau. De la pierre sûrement. Extrêmement bien taillée. La finesse des détails
était bluffante. J'attrapai une loupe pour l'observer de plus près. On ne distinguait aucune trace d'outil qui aurait pu
être utilisé pour la taille. Elle était parfaitement polie, même sur ses parties les plus fines. Les ailes étaient
presque diaphanes malgré la couleur sombre de la pierre. Chaque détail de la face du poulpe était visible. Les pupilles
verticales au fond des yeux étaient parfaitement visibles alors qu'elles ne devait mesurer que quelques millimètres.

Les bras puissants étaient terminés par d'énormes mains qui semblaient prêtes à me sauter à la gorge. Les pieds étaient
profondément enfoncés dans le sol au bas du trône dans lequel la créature était assise. C'est alors que je remarquai
qu'entre les pieds de la chose se tenaient des silhouettes vaguement humaines. La loupe me révéla un ensemble de formes
humanoïdes en vénération devant le monstre. L'une d'elle portait même un plateau à bout de bras sur lequel était
posé\ldots{} Un enfant !

Je faillis lâcher la statue. Quel était donc ce culte où des hommes présentaient à une créature aussi monstrueuse un
enfant ? Quels dégénérés pouvaient imaginer de telles choses ? J'étais horrifié. Je remis l'immonde statue dans son sac
et la renvoyait dans l'obscurité du tiroir d'où je regrettais de l'avoir tirée. Certaines choses méritaient de rester
dans le noir avec les monstres du placard.

Je décidai alors de faire quelques recherches sur ce fameux Lovecraft dont m'avait parlé Bob. Je découvris avec stupeur
un auteur adulé par des centaines de millier de personnes à travers le monde. Un mythe créé de toutes pièces ayant
engendré des centaines de romans, de romans, situés dans l'univers délirant de l'homme. Je lus des pages entières sur
des montres aux noms imprononçables. Un merchandising très important s'était développé autour du \emph{Mythe de Cthulhu}
allant des peluches aux jeux de plateau. Je découvrais un monde qui m'était totalement jusqu'alors totalement inconnu.

Les livres fondateurs étaient disponibles et je téléchargeai quelques recueils de nouvelles avec la ferme intention de
me renseigner plus avant sur ce qui entourait le monstre dont la statue attendait dans son tiroir.

Une main apparu sous mes yeux, tendue, suivie immédiatement d'un :

« Bonjour lieutenant. Je suis le professeur Smitherson. »

Je me redressai sur mon siège. L'homme qui se tenait devant moi paraissait sortir d'un des romans de Lovecraft.
Redingote noire impeccable, haut de forme à la main, boutons de manchette, montre de gousset dont la chaînette barrait
d'un trait d'or son ventre. Il portait même des favoris qui lui mangeaient une bonne partie des joues. Je lui serrai 
la main. Gordy se marrait dans le dos du professeur.

« Pardonnez ma surprise Professeur. Je ne vous attendais pas si tôt. Et je dois avouer que vous êtes assez\ldots{}
surprenant. 

-- Je dois avouer que mon apparence à effectivement de quoi surprendre. Mais j'aime assez cultiver ma différence.

-- Et bien je dois dire que vous réussissez parfaitement à être différent ! Mais je vous en prie, asseyez-vous, lui
dis-je en désignant une chaise à côté de mon bureau. »

Il s'assit et Gordy en fit de même à son bureau. Il prit la parole :

« J'ai donc amené le professeur sur les lieux du crime. Enfin, si crime il y a eu, puisque nous n'avons toujours pas de
cadavre. Et j'ai bien peur que tu n'aimes pas ce qu'il a à te dire John.

-- C'est à dire ?

-- Tu verras par toi même\ldots{} »
