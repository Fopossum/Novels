\chapter[Chapitre Un]{Chapitre Un}
Un cri déchirant la nuit. Suivi d'un autre, étouffé cette fois-ci. Puis, plus rien. Je jetai un coup d'œil rapide à ma 
montre. 2 h 17. J'avais terminé mon service mais je ne pouvais pas ignorer ce cri. Même pendant cette nuit d'Halloween, 
il était possible que cela fût une agression et non pas quelques jeunes jouant à se faire peur. Les cris provenaient du 
fond de l'ancien cimetière. J'accélérai le pas vers la vieille grille rouillée qui se découpait dans les hauts murs 
d'enceinte. L'ombre de la grille qui se déployait devant moi, léchant les pierres tombales à l'abandon avait quelque 
chose de dérangeant. Les nuages fins qui courraient devant la Lune déjà haute dans le ciel voilaient partiellement sa 
lumière et donnaient un côté fantomatique à la scène.

« Ressaisis-toi un peu John » marmonnai-je en posant une main sur la grille et attrapant ma fidèle lampe torche de 
l'autre. Le cadenas et la chaîne, qui d'habitude barraient l'entrée du lieu aux passants, gisaient sur le sol. Dans la 
lumière crue de ma Maglite la coupure nette des maillons indiquait l'utilisation d'une pince. Mauvais signe.

Je sortis mon Glock de son holster et j'avançai prudemment entre les tombes noyées dans les herbes folles. Çà et là, 
des ifs centenaires déployaient leur longues branches qui dessinaient des formes absurdes dans le faisceau de ma 
torche. Une lumière verte diffusait depuis le fond du cimetière et nimbait l'atmosphère d'une aura glauque. Au fur et à 
mesure que je me rapprochais, je ralentissais le rythme en utilisant au mieux les tombes pour me dissimuler.

La lueur poisseuse venait d'un caveau tout proche dont les portes grandes ouvertes grinçaient avec un couinement 
sinistre dans la brise légère qui venait de se lever. Une odeur infecte m'agressa alors les narines. Une odeur telle 
que je n'en avais jamais connu. Mon cœur se mit à battre plus fort dans ma poitrine et je raffermis ma prise sur la 
crosse de mon arme.

Je faillis m'évanouir lorsque j'entrais dans le caveau. L'odeur qui y régnait était insoutenable. Il me sembla 
reconnaitre du souffre et la senteur caractéristique de qui accompagne la Faucheuse. La puanteur était telle qu'elle 
semblait littéralement coller à mes vêtements, imprégnant le tissus en profondeur, se mêlant à ma sueur et dégoulinant 
entre mes omoplates. L'air lourd et épais charriait le désespoir et la perdition.

« Il fallait que ça me tombe dessus, à moi, à deux semaines de la retraite. Qu'est-ce que je fous ici… » Une pulsation 
malsaine emplit alors mes oreilles. Une psalmodie impie. Une langue inconnue et trainante. Des mots incompréhensibles 
qui me glacèrent le sang, s'insinuant au plus profond de mon être, jusque dans mes os. Un langue qui ne devrait pas 
être. Une langue qui portait en elle la promesse d'abominations sans nom, de destruction aveugle et de damnation 
éternelle. Des mots venus d'un autre âge.

L'étrange mélopée se répétait sans cesse, variant d'intensité mais toujours glaçante. La teinte verdâtre de la 
lumière conférait aux ors et aux cuivres ornant l'intérieur du caveau un aspect malsain que la lumière de ma torche 
n'arrivait pas à dissiper. Un escalier aux marches taillées dans le marbre s'ouvrait face à moi et plongeait vers 
les profondeurs de la Terre. je descendis prudemment la volée de marches et débouchai dans un long couloir dallé. Des 
torches placées sur les murs brûlaient, noyant l'environnement de cette violente teinte verte. L'odeur de mort encore 
plus oppressante qu'à la surface laissait présager du pire.

Je me retournai et jetai un œil vers le haut de l'escalier. Si je devais fuir, ce serait la seule voie possible. Je 
n'aimais pas me sentir dépourvu de solutions de repli et cela me mit encore plus mal à l'aise. Je décidai néanmoins de 
continuer vers le fond du couloir. Il fallait que je sache ce qui se tramait ici, sous le cimetière. Les chants étaient 
maintenant plus distincts. Ils semblaient provenir de la pièce située au bout du couloir dont l'accès était barré par 
une lourde porte en bois massif richement ouvragée de fines gravures réalisées dans un métal inconnu qui renvoyait des 
éclats violacés et dorés malgré la lumière verte baignant le lieu.

Prudemment, je posai ma main sur la poignée et tentai d'entrouvrir la porte. Elle refusa de bouger d'un pouce et 
resta fermée. Je collais alors son oreille au bois tiède et je pus distinguer les mots des cantiques. Cela ne 
ressemblait à aucune langue que je connaissais. Les mots, même si je ne les comprenais pas, portaient en eux le Mal 
absolu, palpable et s'imprimaient durablement dans ma mémoire. Ils ne me quitteraient plus jamais, j'en étais persuadé.

Une décharge électrique parcourut mon échine. Je fus pris de sueurs froides, peinant à respirer. Mon sang sembla 
s'épaissir dans mes veines, mon estomac fut pris de soubresauts violents et il s'en fallut de peu pour que je ne perde 
le contrôle de ma vessie. Une terreur primale déferlait le long de mes nerfs à la vitesse de la lumière, m'immobilisait 
sur place. Je ne pouvais pas me détacher de cette porte qui cachait, j'en étais sûr, des pratiques impies auxquelles 
seuls des fous pouvaient s'adonner.

Avec de grandes difficultés je m'éloignais du bois et j'observais les enluminures décorant la porte. Des phrases dans 
une langue inconnue, peut-être celles des cantiques, accompagnaient des dessins représentant des créatures abominables, 
indescriptibles. Elle semblaient sortir des délires éthyliques d'un ivrogne en manque d'alcool bon marché. Soudain, les 
chants s'interrompirent et une odeur méphitique s'insinua à travers la porte. En quelques secondes, je fus entouré 
d'un brouillard fétide. Quelque chose qui ne venait pas de ce monde. La puanteur était telle que je me demandais 
comment je parvenais encore respirer.

Un grognement sourd, presque solide accompagnait le brouillard. Un cri perçant déchira l'air. Un autre cri plus rauque 
répondit au premier. Le grognement continuait et diffusait ses basses épouvantables tout autour de moi. Les chants 
recommencèrent, plus rapides, accélérant, vibration malsaine. Les battements de mon cœur s'emballèrent subitement. Une 
douleur aiguë me traversa la poitrine et le bras gauche. Je cherchai frénétiquement ma petite boîte de pilules et en 
avalai deux. Si je voulais survivre, il me fallait du renfort.

Faire demi-tour pour remonter l'escalier puis sortir du caveau s'avéra être une épreuve presque insurmontable. Je 
luttais contre l'engourdissement de mes membres et la douleur qui me déchirait pour aller m'effondrer une dizaine de 
mètres plus loin près d'une tombe en piteux état. La douleur diminua mais resta là pour me rappeler que j'avais été le 
témoin de quelque chose que je n'aurais pas dû voir. Mais pas le temps de s'apitoyer sur mon sort. Le cri que j'avais 
entendu provenait — j'en étais certain — de la même gorge que ceux qui m'avaient détourné du chemin de mon appartement.

J'empoignai mon téléphone mobile et, tout en me dirigeant vers la rue en trainant la patte, appelai le central pour 
demander de l'aide.
