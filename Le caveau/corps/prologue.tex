\chapter[Prologue]{Prologue}
Un chat errant famélique se dirigeait vers la ruelle en quête de nourriture. Il savait qu'ici, il pourrait trouver des 
poubelles regorgeant de matière à se mettre sous la dent. Un vrai restaurant étoilé. Seulement, il n'était pas le 
seul à connaître l'endroit. Le danger, c'était les chiens. Il espérait qu'aujourd'hui il n'en croiserait pas. Cela 
faisait deux jours qu'il n'avait rien mangé et il commençait à avoir vraiment très faim.

Coup d'œil rapide à droite et à gauche, personne. Ça ne sentait pas le chien non plus. Il haïssait les chiens. 
Presque autant que les humains. Vite il courut  vers la ruelle en quête d'une poubelle à explorer. La seule raison 
pour laquelle il tolérait les humains : ils lui permettaient de se nourrir avec toutes ces bonnes choses qu'ils 
gâchaient. Quand il serait maître du monde, il garderait sûrement les humains comme esclaves. 

Alors qu'il fourrageait un sac au fumet très prometteur, deux silhouettes entrèrent dans son champ de vision. Ils 
avaient quelque chose de bizarre. Étaient-ce les capuches qui couvraient leurs têtes et cachaient leurs visages ou
 le fait que la lumière venue des lampadaires adjacents à la ruelle sembla subitement disparaître, comme si une chape 
d'ombre était tombée entre les immeubles. Il s'assit et observa.

La plus grande prit la parole :

« Tout est prêt ?

— Absolument, répondit l'autre. Tout a été réalisé selon vos instructions.

— L'offrande a été choisie ?

— Oui, l'offrande a été choisie et sera bien au bon endroit au bon moment. 

— Alors c'est parfait. La prophétie doit se réaliser. Tout indique que notre heure de gloire est enfin venue. 

— Puissent les Dieux vous entendre. »

Les silhouettes se séparèrent et partirent chacune de leur côté. L'é\-change avait été bref, presque instantané. La 
lumière éclaira de nouveau l'entrée de la ruelle, plus rien ne l'empêchait d'y pénétrer. Une odeur diffuse mais tenace 
agaça les narines du félin. Une odeur qu'il ne connaissait que trop bien. Une odeur qui aurait dû le mettre en transe 
mais qui lui hérissait le poil en cet instant précis.

L'odeur de la Mort.
