\chapter{}
%\begin{epigraphs}
%\qitem{Et c'est alors qu'il parut. Blême et morne dans l'embrasure où son
%ombre se découpait.}{Chants Obscurs}
%\end{epigraphs}

\lettrine[lines=1]{L}{e} neuro-implant me démangeait. Une démangeaison profonde.
Avec un gros problème. Il est impossible de gratter quelque chose situé à
l'intérieur de soi.

Les toubibs m'avaient prévenu. Pendant quelque jours après la pose de l'implant,
j'allais ressentir de temps à autres une « petite gène ». Petite, mon œil !
C'était intolérable. L'impression qu'une armée de fourmis utilisaient mes nerf
comme des toboggans. Je pris une des pilules magiques qui m'avaient été
prescrites à la sortie de l'hôpital. Elles étaient censée faire disparaître la
gène en quelques minutes. C'est du moins le baratin que m'avait servi le
pharmacien en me remettant en mains la boîte. « Attention, c'est puissant comme
médicament. On recense quelques cas d'addiction ». Il avait rajouté ces derniers
mots au moment où je sortais ma carte pour payer. « Merci Bien, bonne journée
monsieur » avec un rictus désagréable.

Je rentrais dans ma voiture et programmais la direction du commissariat central
place du Capitole, en avalant un de ces comprimés magiques. J'allais voir si le
pharmacien avait raison ou si son petit sourire en coin était juste une manière
de se foutre de moi. Alors que la voiture filait le long des voies réservées je
pensais à tout ce qui avait amené à ce que petit à petit, chaque flic ou
gendarme doive être équipé d'un neuro-implant.

Il avait été rendu obligatoire pour toutes les forces de police depuis quelques
années. J'avais pour l'instant réussi à passer mon tour et à repousser
l'échéance, n'aimant pas vraiment l'idée d'avoir de l'électronique directement
nichée au creux des méandres de mon cerveau. C'était soit disant pour éviter les
bavures. Elles avaient été rendues obligatoire à la suite d'une erreur
judiciaire retentissante où un suspect avait été forcé d'avouer pour une série
de crime qu'il n'avait pas commis. Dans le secret de la salle d'interrogatoire,
il arrivait quelques fois que nous malmenions un peu nos suspects mais il avait
fallut qu'un flic un peu trop zélé face à un type dont la gueule ne lui revenait
pour qu'on en arrive à ça.

Le pauvre type avait été tellement malmené qu'il avait endossé la responsabilité
pour une dizaine de viols sur mineurs. Avec des actes de torture en prime. Ça
avait été une sale affaire. Pas d'ADN exploitable, les gamins avaient donné une
description vague de leur agresseur qui frappait toujours entre chien et loup.
Et ce pauvre hère avait eu le malheur de trainer là où il ne fallait pas quand
il ne fallait pas. Il avait été vu par plusieurs personnes aux environs des
lieux des crimes à des heures qui collaient à peu près et surtout il avait la
tête de l'emploi.

Que voulez-vous ? Un éboueur quarantenaire, un peu porté sur la boisson, bâti
comme un monstre de foire, sans femme ni enfants, sans relations à long terme
connue. Les enquêteurs n'avaient rien trouvé chez lui de probant. Pas d'images
pédopornographiques, rien. Mais il n'était pas connecté au Réseau. Et ça,
c'était suspect. Seuls les gens qui ont quelque chose à cacher ou à se
reprocher ne sont pas connectés. Seuls les criminels, les instables et les
marginaux ne racontent pas leur vie sur les réseaux sociaux et ne passent pas
leur temps à publier des photos d'eux faisant n'importe quoi.

Rajoutez à ça qu'il avait une gigantesque collection de « bouquins » et qu'il
affirmait les avoirs tous lu. Plus personne ne lisait de livres depuis des
années. Les tablette reliées au Réseau permettaient pour une modique somme de
louer un « livre ». Pourquoi avoir des livres papier ? Ça polluait, participait
à la déforestation, c'était lourd, encombrant, fragile, ça pourrissait. Et
surtout, vous pouviez lire des livres que le Grand Conseil avait strictement
interdit à cause des idées subversives qu'ils contenaient. La paix dans notre
société en dépendait.

Après son arrestation, la série de viols s'était interrompue et cela n'avait
que conforté les enquêteurs dans leur sentiment d'avoir serré la bonne
personne. L'enquête, bouclée — bâclée disaient certains — en quelques jours
avait permis l'arrêt des agressions et les enfants pouvaient de nouveau sortir
jouer.

Au tribunal il criait son innocence à qui voulait l'entendre. Il avait hérité
d'un avocat débutant et en avait pris pour perpette. La presse s'était
intéressée à son cas et il avait fait les premières pages des journaux pendant
des semaines. Son visage était entré dans la conscience collective comme celui
du mal absolu.

Jusqu'à ce que les viols reprennent. Même mode opératoire. Même type de
victimes. Même violeur. Certains détails scabreux non révélés à la presse
étaient présents, indiquant pas là même qu'il ne pouvait s'agir d'un imitateur.
Le prédateur était toujours en liberté et un innocent avait été envoyé en
centre de réhabilitation — il y avait des années qu'on ne disait plus prison —
pour des crimes qu'ils n'avait pas commis.

Brisé par la machine judiciaire implacable, par les médias qui l'avaient
transformé en monstre sans âme, le pauvre homme avait décidé de mettre fin à
ses jours.

Une enquête des services internes fut diligentée. Elle révéla que les
inspecteurs d'alors avaient extirpé des aveux sous la contrainte, profitant de
la faiblesse d'un homme et n'avaient pas respecté les procédures en empêchant
que les interrogatoires soient filmés. Les comptes rendus d'audition avaient été
modifiés après coup. Un coupable crédible devait être jeté en pâture aux médias.
Il fallait calmer la populace et ce pauvre éboueur avait été le bouc émissaire
rêvé. 

L'affaire fut rejugée de façon posthume et les chefs d'accusation modifiés en
« Recel de matériel de propagande ». La peine fut allégée. Mais bien trop tard.

Les ramifications de l'affaire montrèrent que des pressions venant des hautes
sphères avaient été exercées et amenèrent à la démission du Ministre de
l'Intérieur de l'époque et il fut décidé de lancer un programme de recherche
permettant le développement de technologies qui permettraient d'éviter de
nouveau ce scandale.

C'est de là qu'étaient nés les neuro-implants aujourd'hui obligatoire. Dès lors
qu'on entrait en salle d'interrogatoire, ils ne pouvaient pas être désactivés et
enregistraient fidèlement toute la séance. Un « filigrane » unique, dépendant
de chaque inspecteur était apposé sur les enregistrements et les développeurs
assuraient qu'il était humainement impossible de contrefaire ou d'altérer les
données. Dès qu'elles étaient captées, elles étaient automatiquement répliquées
dans plusieurs centres géographiquement distants, chiffrées et archivées pour
pouvoir être utilisée comme preuves.

Des années de « Vidéo-protection », de traçage de vos moindres faits et gestes,
de fichage systématique dans des bases de données de plus en plus
interconnectées, de personnes prenant l'habitude de tout révéler leur vie sur le
Réseau — souvenez-vous, seuls ceux qui ont quelque chose à se reprocher ne sont
pas connectés au Réseau — avaient permis de préparer le terrain.

Bien sûr, vous imaginez bien que des activistes de la « Liberté » s'étaient
indignés contre ces mesures. Pointant du doigt les risques d'atteinte à la vie
privée, les dérives potentielles du système. Mais la population était prête à
accepter ces petites entorses à la liberté pourvu que cela permette à la
Justice d'être plus sûre et plus efficace.

Je fait partie des rares flics qui ne voulaient pas de cet implant. Mes
convictions profondes se révoltaient face à cet état de fait. Cela avait valu à
ma carrière un sacré coup de frein d'ailleurs mais je gardais de bons résultats
quand même. En utilisant les bonnes vieilles méthodes enseignées par mon père,
juge d'instruction désormais retraité mais qui avait connu le temps béni où la
société faisait plus confiance aux gens qu'aux machines.

Je ne croisais aucune autre voiture dans les rues animées et bondées de
chalands qui s'écartaient de mon chemin. Depuis que tout les véhicules à moteur
avaient été bannis de la ville pour favoriser les transports en commun moins
polluants, seuls les services d'état disposaient de véhicules individuels. Un
des rares privilèges auquel ma fonction donnait droit.