\chapter[Chapitre Un]{Chapitre Un}
Un cri déchirant la nuit. Suivi d'un second, étouffé cette fois-ci. Puis, plus rien. Je jetai un coup d'œil rapide à ma 
montre. 2 h 17. J'avais terminé mon service mais je ne pouvais pas ignorer ce cri. Même pendant cette nuit d'Halloween, 
il était possible que cela fût une agression et non pas quelques gamins désœuvrés jouant à se faire peur. Les cris 
provenaient du fond de l'ancien cimetière. J'accélérai le pas vers la vieille grille rouillée qui se découpait dans les 
hauts murs d'enceinte.

Les ombres de la grille qui se déployait devant moi, léchant les pierres tombales à l'abandon avaient quelque chose de 
dérangeant. Les nuages fins qui courraient devant la Lune déjà haute dans le ciel voilaient partiellement sa lumière et 
donnaient un côté fantomatique supplémentaire à la scène. J'aurais presque pu croire qu'un loup-garou ou un
vampire allait me sauter dessus. Et ce froid\ldots{}

Et dire que j'étais sur le chemin de mon lit\ldots{} « Ressaisis-toi un peu John » marmonnai-je en posant une main sur
la grille et attrapant ma lampe torche de l'autre. Le cadenas et la chaîne, qui d'habitude barraient l'entrée du 
lieu aux passants, gisaient sur le sol. Dans la lumière crue de ma Maglite la coupure nette des maillons indiquait 
l'utilisation d'une pince. Et c'était récent. Mauvais signe.

Je sortis mon Glock de son holster et j'avançai prudemment entre les tombes noyées dans les herbes folles. Çà et là, 
des ifs centenaires déployaient leurs longues branches qui dessinaient des formes absurdes dans le faisceau de ma 
torche. Dizaines de bras décharnais tentant de m'agriper par leurs ombres. Une lumière verte diffusait depuis le fond du cimetière et nimbait l'atmosphère d'une aura glauque. Au fur et à mesure que je me rapprochais, je ralentissais le rythme 
en utilisant au mieux les tombes pour me dissimuler.

La lueur poisseuse venait d'un caveau tout proche dont les portes grandes ouvertes grinçaient avec un couinement 
sinistre dans la brise légère qui venait de se lever. Une odeur infecte m'agressa alors les narines. Une odeur telle 
que je n'en avais jamais connu. Mon cœur se mit à battre plus fort dans ma poitrine et je raffermis ma prise sur la 
crosse de mon arme. À première vue, il devait s'agir de celui d'une famille riche. Des ornementations torturées 
étaient gravées dans la pierre. De farouches gargouilles semblaient veiller sur le repos des morts de cette famille 
maintenant oubliée à en juger par la quantité de mousse sur les pierres. Le temps avait effacé le nom gravé au 
frontispice et je ne pouvais pas distinguer autre chose que de vagues griffures incompréhensibles.

Je faillis m'évanouir lorsque j'entrais dans le caveau. L'odeur qui y régnait était insoutenable. Il me sembla 
reconnaître du souffre et la senteur caractéristique de qui accompagne la Faucheuse. Cette odeur de mort qui accompagne 
toujours les cadavres. La puanteur était telle qu'elle semblait littéralement coller à mes vêtements, imprégnant les 
tissus en profondeur, se mêlant à ma sueur et dégoulinant entre mes omoplates. L'air lourd et épais semblait charrier 
le désespoir et la perdition.

« Il fallait que ça me tombe dessus, à moi, à deux semaines de ma putain de retraite. Qu'est-ce que je fous ici… » Une 
pulsation malsaine emplit alors mes oreilles. Une psalmodie. Une langue inconnue et traînante. Des mots incompréhensibles 
qui me glacèrent le sang, s'insinuant au plus profond de mon être, jusque dans mes os.

Impossible de comprendre la moindre parole mais quelque chose dans la mélodie semblait faire appel à une partie presque 
primale de mon être. Quelque chose me disait, ma conscience peut-être, que je ne devrais surtout pas être là et que 
je ferais bien mieux de prendre mes jambes à mon cou et prévenir le central plutôt que de m'aventurer plus avant. Une 
autre partie de moi m'intimait d'aller voir et était irresistiblement attirée par les chants malgré le profond 
dégoût qu'ils suscitaient en moi.

L'étrange mélopée se répétait sans cesse, variant d'intensité mais toujours glaçante. La teinte verdâtre de la 
lumière semblait pulser avec les chants et conférait aux dorures et aux cuivres ornant l'intérieur du caveau un aspect 
malsain que la lumière de ma torche n'arrivait pas à dissiper. Quelque chose ne collait pas. 

Un escalier taillé dans la pierre s'ouvrait face à moi et plongeait vers les profondeurs de la terre. Je descendis 
prudemment la volée de marches et débouchais dans un long couloir dallé. Des torches placées sur les murs brûlaient, 
noyant l'environnement de cette violente teinte verte. L'odeur de mort encore plus oppressante qu'à la surface laissait 
présager du pire. Sur quoi allais-je tomber ?

Je me retournai et jetai un œil vers le haut de l'escalier. Si je devais fuir, ce serait la seule voie possible. Aucune 
autre porte ne donnait sur ce couloir. Je n'aimais pas me sentir dépourvu de solutions de repli et cela me mit encore 
plus mal à l'aise. Il fallait néanmoins continuer vers le fond du couloir. Il fallait que je sache ce qui se tramait ici, 
sous le cimetière. Les chants étaient maintenant plus distincts. Ils semblaient provenir d'une pièce située au bout du 
couloir dont l'accès était barré par une lourde porte en bois massif, richement ouvragée de fines gravures, qui semblait 
renvoyer des éclats violacés et dorés malgré la lumière verte baignant le lieu.

Tout ce que je voyais me hurlait que cette situtation n'était pas naturelle. Étais-je sur les lieux d'un tournage de 
film d'horreur pour ces adolescents attardés qui se gavaient de vidéos sur Internet ? Je ne voyais pourtant aucune 
caméra, aucun artifice auquel j'aurais pu m'attendre sur un plateau. Personne aux alentours. Seules ces voix, ces 
terribles voix qui provenaient de l'autre côté de la porte.

Non, il fallait être rationnel, plutôt qu'une équipe de tournage, cela devait être quelques jeunes en mal de sensations 
qui pratiquaient, comment appelaient-ils ça déjà ? Ah oui, un jeu de rôles. C'était cela. Obligatoire. Quoi d'autre ?

Prudemment, je posai la main sur la poignée et tentai d'entrouvrir. Elle refusa de bouger d'un pouce et la porte  
resta fermée. Je collais alors son oreille au bois tiède et je pus distinguer les mots des cantiques. Cela ne 
ressemblait à aucune langue que je connaissais. Les mots, même si je ne les comprenais pas, portaient en eux le Mal 
absolu, palpable et s'imprimaient durablement dans ma mémoire. Ils ne me quitteraient plus jamais, j'en étais persuadé.

Avec de grandes difficultés je m'éloignais du bois et j'observais les enluminures décorant la porte. Des phrases dans 
une langue inconnue, peut-être celles des cantiques, accompagnaient des dessins représentant des créatures abominables, 
indescriptibles. Elle semblaient sortir des délires de quelque fou en proie à des hallucinations. Soudain, les 
chants s'interrompirent et une odeur méphitique s'insinua à travers la porte. En quelques secondes, je fus entouré 
d'un brouillard fétide. La puanteur était telle que je me demandais comment je parvenais encore respirer.

Une décharge électrique parcourut mon échine. Je fus pris de sueurs froides, peinant à respirer. Mon sang sembla 
s'épaissir dans mes veines, mon estomac fut pris de soubresauts violents et il s'en fallut de peu pour que je ne perde 
le contrôle de ma vessie. Une terreur infernale déferlait le long de mes nerfs à la vitesse de la lumière, m'immobilisait 
sur place. Je ne pouvais pas me détacher de cette porte qui cachait, j'en étais sûr sans que je ne comprenne d'où venait 
cette certitude, des pratiques auxquelles seuls des fous pouvaient s'adonner.

Un grognement sourd, presque solide accompagnait le brouillard. Un cri perçant déchira l'air. Un autre cri plus rauque 
répondit au premier. Le grognement continuait et diffusait ses basses épouvantables tout autour de moi. Les chants 
recommencèrent, plus rapides, vibration malsaine quasi frénétique. Les battements de mon cœur s'emballèrent encore plus 
et une douleur aiguë me traversa la poitrine et le bras gauche. Je cherchai ma petite boîte de pilules et en 
avalai deux. Si je voulais survivre, il me fallait du renfort.

Faire demi-tour pour remonter l'escalier puis sortir du caveau s'avéra être une épreuve presque insurmontable. Je 
luttais contre l'engourdissement de mes membres et la douleur qui me déchirait pour aller m'effondrer une dizaine de 
mètres plus loin près d'une tombe en piteux état. La douleur diminua mais resta là pour me rappeler que j'avais été le 
témoin de quelque chose que je n'aurais pas dû voir. Mais pas le temps de s'apitoyer sur mon sort. Le cri que j'avais 
entendu provenait — j'en étais certain — de la même gorge que ceux qui m'avaient détourné du chemin de mon appartement.

J'empoignai mon téléphone mobile et appelai le central pour demander de l'aide.

Tapie au creux de ma conscience, une petite voix me taradait que ce n'étaient sûrement pas des gamins qui jouaient à se 
faire peur mais quelque chose de plus grave et dangereux que des hippies en manque de sensations fortes.
