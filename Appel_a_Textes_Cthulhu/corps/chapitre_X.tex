\chapter{Chapitre X}
Le réveil sonna. Le bip-bip strident me vrilla les tympans et me força à me réveiller. Encore une nuit peuplée de ces 
mêmes cauchemars qui me hantaient depuis maintenant deux semaines. Depuis cette nuit d'Halloween où mon univers 
avait basculé dans l'irréel et la folie. Mais ça irait mieux maintenant. Penser à désactiver le réveil. Je suis en 
\emph{vacances} !

Fini la chasse aux sectes et aux sorciers. Je laissais tout cela à mes successeurs et à Gordy. Ils sauraient 
parfaitement comment gérer les choses. J'en avais enfin fini avec tout ça et je comptais bien enfin profiter de mon 
temps. La retraite. Enfin. Des années passées au service de la communauté qui avaient eu raison de mon couple, de ma 
famille et enfin de ma santé. Je me redressai pour me lever. La douleur fut brève et fulgurante. Encore une alerte 
de mon vieux coeur. Je devais absolument essayer de me ménager un peu. 

J'allais enfin pouvoir profiter de mes petits enfants et m'occuper du jardin. L'idée que je ne devais pas aujourd'hui 
aller au bureau remuer les histoires sombres de tous ces inconnus me fit sourire. Mais avant tout cela, il fallait 
que je pense à appeler mon cardiologue. Et que je range les cartons contenant toutes les affaires que j'avais rammené 
du 
bureau hier. Toute une vie d'enquêtes qui tenait dans quelques boîtes en carton.

Je me levai avec difficultés, enfilai les pantoufles en forme de pattes de chat que ma petite fille de quatre ans avait 
choisi pour mon dernier anniversaire et me dirigeai d'un pas résolu et pelucheux vers la cuisine pour aller me préparer
un café. Alors que je passais la porte de ma chambre, j'évitais de justesse le chat qui se dirigeait ventre à terre 
vers 
ma chambre. Depuis des années, il était mon seul compagnon et profitait toujours de mon lever pour que je lui laisse 
la place chaude sous la couette.

Je l'entendis miauler. Ce n'était pas dans ses habitudes et l'appelai pour qu'il me rejoigne. Nouveau miaulement, 
différent des habituels. Bah, ça lui passerait avant peu. Je me dirigeais vers la cafetière lorsque qu'il vint sauter 
sur le plan de travail et s'assoir devant moi tout en continuant à miauler. Il me regardait droit dans les yeux.

« Mais qu'est-ce que tu as toi ? »

Je ne m'attendais pas à une réponse. Mais un miaulement rauque, venu du fond de sa gorge accueillit ma question. 
Il se passait décidément quelque chose. Il pencha la tête sur le côté et miaula de nouveau. L'espace d'un instant, je 
cru percevoir une question. J'étais persuadé que si il savait parler, il serait en train de me demander si j'allais 
bien.

« Mais oui, tout va bien. Ne t'en fais pas. »

Nouveau miaulement et il détala vers la chambre. Voulait-il me montrer quelque chose ? À tout les coups, il m'avait 
encore ramené une souris et voulait me montrer à quel point il m'aimait.

En passant la porte, mon souffle se coupa. Je manquais d'air et il me fallut quelques secondes pour reprendre mon 
souffle. Le chat était sur le lit et surmontait une forme massive.

« Merde… Qu'est-ce que… »

La forme sur le lit était un homme. Allongé. Il me tournait le dos. Le chat se tourna vers moi et miaula de nouveau. Il 
me regardait avec insistance m'invitant à venir vers lui. Je m'approchai de l'homme. Il paraissait endormi 
profondément. 

Les questions se bousculaient dans ma tête. Rien de tout cela n'était normal.

Comment cela était-il possible ?

Comment un homme pouvait-il être en train de dormir dans \emph{mon} lit.

Par \emph{où} était-il rentré sans que je ne l'entende ?

Comment avait-il fait pour passer inaperçu ? 

Le chat sauta du lit emportant dans son mouvement le drap qui recouvrait l'homme et s'assit par terre.

« Nom de Dieu ! »

L'homme portait mon pyjama.

La compréhension claqua dans mon esprit comme un coup de tonnerre. L'homme paisiblement endormi dans mon lit était moi. 

J'étais mort. C'est ce que le chat essayait de me dire depuis le début. Je ne me demandais même pas comment il pouvait 
me voir. Cela me paraissait tout à fait… Naturel. Je me tournai vers lui :

« Alors le chat, c'est ça, je suis mort n'est-ce pas ?

— Oui. Tu es mort. »

Nouveau choc. J'avais distinctement entendu mon chat me répondre bien que je le l'ai pas entendu miauler.

« Et oui John, nous autres chats avons quelques capacités insoupçonnées par vous autres les humains. En particulier 
celle d'interagir avec ce que vous appelez le monde des morts. Nous voyons sur plusieurs plans d'existence en même 
temps et sommes le lien entre ces plans. De tout temps, nous vous avons accompagnés dans ce passage. Il me faut 
maintenant t'expliquer ce qu'il va t'arriver.

— Mais… Mais…

— Nous avons peu de temps et j'ai beaucoup de choses à te dire John. Assieds-toi donc et écoute-moi très attentivement.»

Je m'assis sur le fauteuil à côté de mon lit. Sans réussir à comprendre comment tout en étant mort je pouvais encore 
avoir des interactions avec le monde physique.

« Ta mort a été organisée et tu étais la dernière pièce d'un plan énorme qui va entraîner l'éradication totale de 
l'homme. À quelques exceptions près, évidemment. Survivront ceux qui serviront leurs nouveaux maîtres. Mais je ne pense 
pas que l'on pourra qualifier ça de vie.

— Quoi ? Co… Comment ?

— Oui John. Tu as été assassiné. Par les mêmes personnes que tu traquais après avoir découvert la crypte. D'ailleurs, 
cette cérémonie t'étais explicitement destinée. Afin de pouvoir pénétrer le monde de tes rêves. Tu te souviens de tous 
ces cauchemars ? Et bien c'était un petit aperçu de ce qui allait arriver.

— Mais pourquoi moi ?

— Tu étais spécial John. Une histoire de naissance. D'endroit, de date. Tout cela est une minuscule pièce du puzzle qui 
se met en place depuis des siècles et des siècle. Mais tu étais la dernière pièce John. Désormais, leur plan va pouvoir 
s'appliquer. D'ici peu, quelques centaines d'années tout au plus, le monde aura été envahi par les atrocités que tu as 
entraperçu dans tes délires et s'en sera fini de l'humanité telle que tu la connais.

— On doit pouvoir faire quelque chose contre ça ? Vous là, les chats, vous pouvez sûrement faire quelque chose bon Dieu 
! »

Je m'étais levé et faisait les cent pas dans ma chambre.

« Nous ne sommes que spectateurs. Nous n'avons aucun pouvoir. »

Le chat s'éloigna. Au moment de passer la porte il se retourna vers moi.

« Et au fait, John, mon nom est Sehkris. Et Dieu n'existe pas. Bon courage.»