\chapter{Briefing}

Arrivé au commissariat, j'allais directement dans le bureau du commissaire. Je frappais rapidement à la porte de son
bureau et j'entrais sans y avoir été invité.

« Tersant ? Vous vous croyez dans un moulin ?

— Madame, veuillez m'excuser, mais je crois qu'on a un sérieux problème sur les bras.

— Qu'est-ce que qui vous fait dire ça ? »

Je lui relatais alors dans les grandes lignes mon  entrevue avec Diaz lui décrivant ce qu'il m'avait raconté. Je lui
expliquais la présence de drogue dans son organisme, son absence totale de souvenirs après qu'il soit descendu du
métro et son réveil dans la salle d'opération. Je lui racontais sa fuite à travers les bois pour finir par arriver 
sur le bord de l'autoroute.

« Vous vous foutez de moi ? Vous voulez dire que ce type a été enlevé puis attaché à une table d'autopsie avant de
réussir à s'enfuir ?

— En gros c'est ça.

— Et il vous a décrit des instruments chirurgicaux à côté de lui ?

— Tout à fait.

— Et vous croyez vraiment ce qu'il vous a raconté ?

— Il m'a paru sincère. Le toubib confirmera la présence de drogue et c'est dans son rapport, ainsi que l'amnésie. Vous
pouvez de toute manière visionner tout l'enregistrement - je montrais du doigt ma tête en même temps - si vous ne me
croyez pas. Mais je pense sincèrement qu'il faut prendre cette affaire au sérieux. »

Elle effleura un interrupteur et sa voix retentit dans les bureaux : « Je veux toute l'équipe en salle de briefing
immédiatement. » Elle se leva et se dirigea vers la porte. D'un geste de la main elle m'intima l'ordre de la suivre. Il
était inutile de finasser. 

Nous arrivâmes les premiers suivis de près par Valdeski et Charenton. Portal, Jensain et Brumel arrivèrent par la 
suite. La grande table de la salle était entourée de quelques chaises et chacun d'entre nous s'assit. Un projecteur 
dissimulé  dans le plafond permettait de diffuser des vidéos ou des diapos si on en avait besoin. Seule le commissaire 
restait debout, en bout de table, ses yeux perçants nous observant tous. Nous avions parfois l'impression d'être face à 
une maîtresse d'école et c'est exactement ce que nous ressentions tous à cet instant. Comme si elle pouvait lire en nous 
à livre ouvert et que rien ne pouvait lui échapper.

Portal me lança un regard incrédule l'air de me demander ce qui pouvait bien se passer pour que nous soyons tous ici.
Brumel jouait nonchalamment avec un stylo, Valdeski qui ne se séparait jamais de sa tablette tapotait quelque chose
dessus. Charenton et Jensain s'assirent et sortirent de quoi prendre des notes. Dès que tout le monde fut installé, 
elle prit la parole. « Tersant vient de prendre la déposition d'un type qui a été retrouvé nu errant le long de 
l'autoroute cette nuit. Il a des raisons de croire que le gars ne délire pas et est tout à fait sérieux. Si ce qu'il 
dit est vrai, nous avons affaire à un vrai sadique qu'il va falloir vite arrêter. Tersant, vous avez la parole. »

Je me levais et me dirigeais vers le grand tableau blanc au mur. « Vous avez accès à la déposition complète du témoin 
sur le réseau, ainsi que le rapport médical. Mais, voici, rapidement, ce que je sais déjà.  » Je leur racontais ce que 
j'avais entendu tout en notant aut tableau les points essentiels de la déposition. L'heure présumée de départ du bar, 
le métro. La descente aux Argoulets, le black-out avant le réveil ligoté. La fuite et enfin l'arrivée le long de 
l'autoroute.

A la fin de mon exposé, Valdeski m'interpella :

« T'es sérieux là ? Ton type il est pas un peu cintré des fois ?

— Le toubib que j'ai vu confirme qu'on a trouvé de la drogue dans son organisme. Je ne pense pas qu'il l'ait consommé 
de manière récréative et volontaire. Il n'avait aucune raison de  cacher qu'il s'était défoncé, si ça avait été le cas. 
Pour ma part, je n'ai aucune raison, à première vue de ne pas le croire…

— Mais, sans déconner, une table d'autopsie ? Charenton doutait toujours — quelques fois trop —. C'était ce qui faisait 
de lui un bon flic. Je me tournais vers lui.

— Il n'a pas pu me préciser plus de choses. Table d'autopsie ou d'opération. Avouez que quand on fait un boulot 
normal, on a pas souvent l'occasion d'assister à une autopsie. Il m'a dit lui-même qu'il n'était pas très sûr de lui 
et que c'est d'après ce qu'il avait vu à la télévision qu'il avait compris sur quoi il était allongé. »

Le commissaire reprit la parole :

« Si ce que Tersant nous a expliqué est vrai, je pense que nous avons un taré en liberté. Vous allez tous me visionner 
la déposition du témoin. Une fois ceci fait, Valdeski, vous me récupérez toutes les bandes des caméras situées sur le 
chemin présumé de Diaz, le suivre et voir si quelqu'un s'en approche après sa descente du métro. Jensain, vous vous 
occupez de délimiter une zone géographique où pourrait se trouver le lieu de détention en fonction de où notre homme a 
été récupéré. Brumel avec Portal, vous me disséquez la vie de Diaz. Je veux tout savoir de lui. Et quand je dis tout, 
c'est tout. Tersant, vous m'accompagnerez au bar ce soir et on va gentiment demander au patron si il n'aurait pas 
quelque chose à nous raconter. En attendant, vous me commencez votre rapport. Je le veux sur mon bureau avant que nous 
n'allions au bar. Des questions messieurs ? Non ? Alors au boulot ! »

Ses yeux lançaient des éclairs. Quand elle était comme ça, nous savions tous que nous avions intérêt à filer droit et à
obéir sur le champ. Elle avait cette capacité innée à vous faire faire ce qu'elle désirait exactement sans que l'idée 
de la contredire ne vous vienne à l'esprit. Son regard indiquait clairement qu'elle m'avait cru et qu'elle aussi 
pensait que l'on avait dans la nature un dingue qui prenait son pied à torturer des gens.

J'allais vers mon bureau et m'installais sur mon fauteuil. L'écran de mon ordinateur me narguait et j'allais devoir 
m'atteler à mon activité favorite. La rédaction de rapports. La paperasse. Encore et toujours. Je me souvenais de 
mon père pestant sur tous ces maudits papiers qu'il fallait remplir pour le suivi, les archives, les dossiers en 
cours… Au moins, je pouvais profiter de ce petit moment de calme pour manger un bout. Je n'avais rien dans l'estomac 
depuis la veille et la journée s'annonçait longue.

Je décrochais mon téléphone et composait le numéro de ma sandwicherie préférée. Ils livraient dans les bureaux et 
finissaient par me connaitre par cœur, comme pas mal des flics du commissariat. Nous étions des clients réguliers des 
brasseries, pizzerias kébabs et autres sandwicheries des environs. Toujours à devoir manger rapidement.

J'allumais mon ordinateur. La petite musique d'ouverture de session sonna, me promettant un moment d'une joie intense. 
J'en profitais pour vérifier si je n'avais pas de message interne. Rien. Parfait. J'avais tout mon temps pour rédiger 
cette saloperie de rapport sur la matinée.

Vingt minutes après avoir commencé, un jeune livreur m'apporta ma pitance. Je le dédommageais de quelques piécettes 
supplémentaires pour le déplacement. Mon repas se composait d'un sandwich au pastrami et d'une bouteille d'eau 
minérale. Cela suffirait bien jusqu'au soir. Une fois ce fleuron de la gastronomie englouti, je me penchait à 
nouveau sur mon rapport.

Au bout d'une heure environ, j'avais fait à peu près la moitié du boulot et je m'offrait une pause bien méritée. 
J'allais pouvoir enfin fumer une cigarette. Mon paquet de clopes dans une main et mon briquet dans l'autre, j'allais me 
caler sur le balcon. Essayant tant bien que mal de me mettre à l'abri de la pluie j'allumais ma sucette à cancer. Les 
volutes de fumée que j'exhalais me détendirent et je pus enfin souffler. 

Jensain me rejoignit et engagea la conversation :

« Tu y crois vraiment ?

— Oui. Et tu aurais été avec moi dans cette chambre, tu penserais comme moi. Son récit à le goût désagréable de la 
vérité. Le genre de choses que tu ne peux que croire. Même si ça te paraît complètement fou. Tu sais que c'est vrai.

— Ok. Et… Au fait… Toi ? tout va ?

— Tu sais ce que c'est. Tu es déjà équipé depuis un moment.

— N'essaye pas d'éluder la question. Tu sais très bien de quoi je veux parler.

— Ça ira lui dis-je d'un ton sec — peut-être un peu trop. Je vais faire aller. 

— Bon, moi, j'ai commencé a délimiter une zone géographique de recherches. Ça va être tendu. D'après ce que tu nous a 
dit, quand Diaz à réussi à s'échapper, il faisait nuit. Il a été ramassé vers cinq heures du matin près de l'autoroute. 
Considérant qu'un marcheur normal se déplace en terrain plat à environ 5 km/h on peut sans trop se planter évaluer 
que notre gars a plutôt gambadé aux alentours de  3 km/h. La nuit se couchant à environ 20h30 à cette date là, il 
aura marché au maximum pendant 8h30. Tu me suis ?

— Oui oui, continue. J'allumais une deuxième cigarette.

— Tu nous a dit qu'il avait dormi mais qu'il ne savait pas combien de temps. Cela peut être 2 heures comme 10 minutes. 
Nous n'avons aucun moyen de le savoir, donc on va prendre une hypothèse pessimiste et considérer qu'il n'a pas dormi. 
Si on s'en tient à ces hypothèses, il a marché dans un cercle d'environ 20 km de rayon. Ce qui nous fait un disque 
d'environ 1250 km carrés.

— Ah ouais, quand même…

— Il va falloir le réinterroger pour voir si il se souvient de points de repères remarquables pour limiter la surface à 
couvrir. Parce que là, on en a pour des jours à arpenter la campagne pour trouver. 

— Et tu n'as pas moyen d'affiner l'estimation ?

— Si bien sûr, je vais éliminer les zones urbanisées et me concentrer sur les forêts. Mais cela risque de ne pas être 
très concluant sans plus de précisions. Je voulais te prévenir au cas où tu aurais trop compté sur ça.

— Le commissaire ne va pas être très contente.

— Je sais… Mais pour l'instant, c'est le mieux que je puisse faire sans informations complémentaires. »

Je quittais le balcon et allais mettre la touche finale à mon rapport en incluant les premières constatation de 
Jersain. Une fois ceci fait, j'envoyais mon rapport préliminaire au commissaire et j'allais frapper à la porte de son 
bureau.

« Entrez Tersant.

— Madame, je viens de vous envoyer mon rapport préliminaire.

— J'ai vu. Merci de votre rapidité. Je lirais tout cela ce soir. Elle regarda sa montre. Je pense qu'il est l'heure 
d'aller faire un petit tour au bar. Qu'en dites-vous ?

— Je suis à votre disposition madame.

— Passez devant, je vous rejoins en bas au parking dans cinq minutes. »

Le commissaire me rejoignit devant l'ascenseur. Elle avait passé un long manteau noir. Décidément, même après cette 
dernière année à la côtoyer, je trouvais que tout ce qu'elle portait lui allait parfaitement. Elle avait un goût très 
sûr dans ses choix vestimentaires qui arrivaient à la mettre en valeur tout en gardant un côté sévère. Lorsque la porte 
s'ouvrit, je m'effaçait légèrement pour la laisser passer devant. Elle me remercia d'un sourire à la fois glaçant et 
envoutant. Un long frisson remonta le long de ma colonne.

D'un geste rapide elle effleura le bouton du parking et la cabine se mit en marche silencieusement. Nous ne décrochâmes 
pas un mot le temps de la descente. Elle était de marbre. Une statue du commandeur dégageant puissance et justice 
divine.

C'est lorsque je fus installé au volant qu'elle se décida à parler.

« Tersant, est-ce que tout va bien ?

— Oui madame. Tout va bien.

— Quand est-ce que vous avez votre rendez-vous de suivi pour l'implant ?

— D'ici une semaine. »

Mes réponses laconiques ne semblaient pas la satisfaire.

« Si quelque chose n'allait pas, je vous demande de me le dire tout de suite. Je ne compte pas perdre d'enquêteur.

— Oui madame.

— S'il vous plaît. Ne m'en veuillez pas. Il fallait que je vous donne cet ordre. Nous avons eu des instructions venant 
de plus haut. Tous les enquêteurs de la PJ doivent être implantés avant la fin de l'année. Et vous étiez le dernier de 
notre équipe à refuser.

— Vous savez très bien pourquoi.

— Tersant. Elle est morte il y a quoi ? Quatre ans maintenant ?

— Dans deux mois, cela fera quatre ans oui.

— Vous… Vous ne pensez pas qu'il serait temps pour vous d'aller de l'avant ? »

Sa voix s'était adoucie. À la limite du murmure. Je n'arrivais pas à croire qu'elle ait pu montrer un peu de compassion.

« J'ai relu vos états de service. Vous étiez un excellent flic. Vous devriez déjà être à ma place et avoir votre 
propre équipe. Vous étiez promis à un bel avenir. Jusqu'à ce que… Ne laissez pas le passé vous détruire. Vous êtes de 
loin le meilleur de ce groupe voire de toute le commissariat et vous gâchez vos possibilités en continuant à vous 
morfondre. Je suis persuadée que vous pouvez avancer. »

Je conduisais machinalement. Je voulais me concentrer sur la route. Ne pas penser à tout ça. Pas maintenant. Pas devant 
elle. Je devais couper court au plus vite à cette conversation qui me retournait l'estomac. J'avais envie de vomir. Je 
luttais pour garder mon calme.

« S'il vous plaît madame, restons-en aux relations hiérarchiques. Je vous remercie de votre sollicitude, mais vous ne 
pouvez rien pour moi. Ce sont mes démons, et j'entends bien les combattre seul.

— Comme vous voudrez. Mais croyez bien que si vous aviez besoin d'une épaule sur laquelle vous appuyer, je serai là. »

Le silence s'installa. Pesant. Épais. Palpable. Un mur venait de se dresser entre nous. Les rues défilaient. J'essayais 
de retenir les larmes qui me brûlaient les yeux. Comment osait-elle ? Comment osait-elle me dire qu'elle voulait m'aider 
alors que par sa faute, je portais en moi ce qui avait tué ma femme. Les articulations de mes doigts blanchirent alors 
que je me crispais sur le volant. Je dus faire un effort surhumain pour me détendre un peu.

L'espace d'un instant, je crus percevoir dans le rétroviseur le sourire narquois du petit gris assis sur la banquette 
arrière. Un deuxième regard me rassura quand à l'absence d'un extraterrestre dans ma voiture. Je commençais à perdre 
les pédales. Il fallait que je me calme. Et vite.

\fancybreak{***}

Ma voiture s'immobilisa à quelques mètres du bar. Dans la lumière grise de cette fin d'après-midi nuageuse, l'enseigne 
lumineuse du \emph{Calice} était un phare dans la nuit, attirant les passants frigorifiés par la soudaine baisse de 
température et la fine pluie qui venait de s'arrêter. La grande baie vitrée révélait les tables et les banquettes 
alignées devant le long bar. J'ouvrais la porte et rentrais dans l'établissement, le commissaire sur mes talons. 
Quelques rares clients étaient installés et sirotaient leurs bières dans des grands verres sérigraphiés.

De grandes étagères chargées de verres renvoyant les éclats des luminaires du plafond couvraient le mur derrière le 
comptoir sombre d'où émergeaient une dizaine de pompes cuivrées aux reflets d'or. Une musique discrète emplissait le 
lieu, diffusée par des enceintes invisibles. Les grands bardeaux de bois flotté remontant jusqu'à mi hauteur des murs 
gris bleu tranchaient avec les longues poutres noires qui courraient sur un plafond d'un blanc immaculé. Le fond de la 
salle était entièrement occupé par une estrade qui pouvait servir de scène à en juger par l'emplacement d'une console 
de mixage dans un coin à côté d'un escalier d'acier et verre descendant vers le sous-sol.

Le commissaire s'approcha du comptoir et s'assit sur un des hauts tabourets. Je m'assis à côté d'elle en remarquant 
les grandes ardoises sous les étagères couvertes de noms évocateurs de bières du monde entier. Le barman, qui 
n'avait pour l'instant pas pipé mot s'approcha de nous et tendis la main. « Bienvenue au Calice. Que puis-je pour votre 
service ? »

Il n'était pas très grand, voire plutôt petit et avait la dégaine d'un moine débonnaire dans le tablier bleu qui 
enserrait sa taille. Son visage était mangé par une imposante barbe d'un roux profond autour d'un grand sourire 
avenant et ses yeux pétillaient d'un air malicieux sous un crâne parfaitement rasé.

Ma supérieure sortit d'une poche intérieure de son manteau sa carte tricolore et annonça tout de suite la couleur. « 
Commissaire Maretz. Et voici le capitaine Tersant. Police judiciaire. Nous aurions quelques questions à poser au 
patron si vous le permettez. »

Le sourire du barman disparut aussitôt pour laisser la place à une moue circonspecte. « Je suis le propriétaire de cet 
établissement. J'imagine que j'ai peu de chance de vous voir consommer puisque vous êtes en service. » Alors que je 
hochais la tête d'un air résigné le commissaire me surprit. « Bien qu'en service, nous pouvons aussi nous accorder un 
peu de plaisir. Et je ne pense pas que le capitaine me tiendra rigueur d'un verre partagé avec lui. »

Je restais sans voix. Je n'étais pas au bon de mes surprises avec elle. Ce n'était pas la première fois que nous devions 
enquêter dans un bar et que nous refusions une boisson. Pourquoi aujourd'hui faisait-elle une entorse à la sacro-sainte 
règle interdisant de boire pendant le service ? Elle poursuivit : 

« Après tout, la journée est bientôt finie pour nous aussi et j'avoue que j'ai un petit faible pour les bonnes bières. 
Que pouvez-vous nous proposer qui sorte de l'ordinaire ? Sans toutefois abuser puisque nous avons un rôle à tenir tout 
de même ! Je ne voudrais pas que mon capitaine ne me coffre pour ivresse sur la voie publique. » Elle laissa échapper 
un rire cristallin que je ne lui connaissait pas. L'avais-je seulement entendu rire une seule fois ? Le patron du bar 
se dérida et de nouveau un sourire barrait son visage rond.

« Alors ? Qu'est-ce que je vous sers ?

— Surprenez moi. Mais évitons quelque chose de trop fort s'il vous plait.

— J'ai peut-être quelque chose pour vous. Et pour votre collègue à l'air morne, ce sera ?

— La même chose qu'elle. »

Le patron attrapa deux verres où était écrit en larges lettres rouge \emph{Saison} et se dirigea vers l'alignement des 
pompes. Le liquide doré coula dans les verres et un grand col de mousse se forma. Les verres, rafraichis par le 
breuvage se couvraient déjà de condensation lorsqu'il les posa devant nous.

« Saint Feuillien Saison. Légère, un peu d'amertume, relativement peu alcoolisée, parfaite pour une fin de journée 
fatigante. Maintenant, si vous me disiez ce qui vous amène ici ?

— Avant de vous répondre, je dois vous informer que cette conversation va être enregistrée grâce à nos implants et 
pourra être considérée comme preuve. Avez vous bien compris ?

— Parfaitement. Et je n'ai rien à vous cacher. »

Le picotement familier se fit ressentir dans mes doigts. Je jetais un regard à ma commissaire. Elle était lumineuse. 
Parfaitement à l'aise. Je ne l'avais jamais vu sous cet angle. Elle donnait l'impression d'être dans son élément. Mais 
très vite, elle redevint de glace et la professionnelle réapparut, éclipsant la jeune femme séduisante qu'elle 
avait été l'instant d'avant l'espace d'une seconde. Ce fut si rapide que je me demandais si je n'avais pas une nouvelle 
fois halluciné.

« Nous devons tout d'abord vérifier votre identité. Tersant ?

— Veuillez poser votre pouce dans le rectangle blanc lui dis-je en présentant mon téléphone. »

La stridulation caractéristique se fit entendre. Les empreintes correspondaient à Mathias Heulin. 35 ans. Pas de 
casier. La photo d'identité collait avec le gars que j'avais en face de moi. 

« Il est clean. Mathias Heulin.

— Merci Tersant. Monsieur Heulin, quand avez-vous vu Francisco Diaz pour la dernière fois ?

— Francisco ? Qu'est-ce qu'il a bien pu faire comme connerie pour que je me retrouve face à la PJ ?

— Rien de bien méchant. Il s'est contenté d'être enlevé et s'est échappé des griffes de son ravisseur avant d'être 
retrouvé hier errant le long de l'autoroute.

— Merde… Il accusa le coup. Il va bien ?

— Il est hors de danger. Tout va bien pour lui. Mais nous aimerions retracer son emploi du temps dans les heures 
précédent sa disparition. Nous avons quelques zones d'ombre à éclaircir. »

Elle but une gorgée de sa bière et continua :

« Vous aviez raison, parfaite pour une fin de journée. Mais revenons-en à nos moutons.

— Je l'ai vu, il y a de ça. Quoi… Trois jours je crois. C'est un habitué maintenant et son groupe et lui ont joué ici 
quelques fois. Il est venu seul et a retrouvé des amis à lui. Ils ont bu quelques bières et on discuté pendant la 
soirée.

— Avez vous remarqué quelqu'un qui sortait de l'ordinaire avec qui il aurait pu discuter ?

— Rien de particulier. Il sembla plonger dans ses souvenirs recherchant un détail. Si. Il a passé une demie heure 
environ à discuter avec une jeune femme. Francisco a toujours été un dragueur. Et quand il voit une fille qui lui 
plaît, il ne peut pas s'empêcher de l'aborder. La chose inhabituelle c'est que ce soir là, c'est elle qui l'a abordé.

— Une femme dites-vous ? Vous pourriez la décrire ? »

Je prenais des notes au fur et à mesure. Je remarquais un subtil changement dans l'attitude du commissaire. Elle 
s'était légèrement penchée en avant et il émanait d'elle une aura de… Chasseresse. Elle tenait quelque chose. Nous 
avions peut-être un autre témoin. La présence de la femme avait été mentionnée par Diaz lors de sa déposition. Avec un 
peu de chance on pourrait l'identifier et avoir sa version.

« Autant que je me souvienne, elle était plutôt séduisante. À peu près de votre taille. Brune. Un joli visage. Mais 
rien qui m'ait marqué. Vous savez, je vois pas mal de gens et pas mal de jolies filles.

— Ce n'était pas une habituée ?

— Non. Je l'avais peut-être croisée avant mais ça ne m'a pas marqué. »

Vague de déception clairement visible sur son visage. Je bus un peu de ma bière. Effectivement, elle était plutôt 
bonne. Une deuxième gorgée suivit la première.

« Désolé de ne pas pouvoir vous aider plus.

— Ce n'est pas grave. Donc, il a discuté avec cette jeune femme et puis ?

— Elle est partie avant lui. Il a bu une bière de plus puis a quitté le bar. Avant minuit. Il voulait attraper le 
dernier métro pour rentrer.

— Il vous a paru différent de d'habitude ?

— Non. Un peu éméché mais rien de bien méchant. Ses amis sont partis plus tard puisque si j'ai bien compris, ils vivent 
moins loin et peuvent rentrer à pied. »

Nouvelle gorgée de bière pour elle et moi. Vraiment, elle se laissait boire, pas désagréable du tout. Le patron avait 
su exactement ce qu'il nous fallait à ce moment de la journée. Si il était comme ça avec tous ses clients, je pouvais 
d'ores et déjà lui prédire du succès et une clientèle fidèle dont je pourrais faire partie.

« Vous auriez quelque chose à ajouter ?

— Non. Rien qui me revienne. Comme je vous disais, une soirée habituelle sans rien de marquant.

— Merci pour votre coopération monsieur Heulin.

— Pas de problème. Vous savez si Francisco peut recevoir des visites ? »

Je pris la parole « Oui, il peut. Il est à Purpan. Voici le numéro de sa chambre. » Je notais rapidement le nom du 
pavillon et le numéro de la chambre sur une feuille de mon calepin et lui tendis. « Merci capitaine. » Il s'éloigna 
vers sa caisse, visiblement abattu par cette histoire.

« Tersant ?

— Oui madame ?

— Vous essayerez d'identifier cette fille. Quelque chose me dit qu'elle pourrait nous en apprendre sur cette histoire.

— Bien sûr. Je vais tout de suite passer un coup de fil à Valdeski pour lui demander de voir si il la remarque sur les 
bandes. »

J'attrapais mon téléphone et allais vers le fond de la salle tout en composant le numéro. Trois tonalités plus tard, la 
voix de Valdeski me répondait :

« Marco ? C'est Tersant. Nous avons un début de piste mais nous avons besoin de ton œil de lynx.

— Dis toujours.

— Nous savons que Diaz a parlé avec une jeune femme dans le bar. Cela a été confirmé par le patron. Elle est partie 
avant lui. Essaye de nous la dénicher su les bandes. Grande, brune, mignonne.

— T'as pas plus vague comme signalement ?

— Si, c'est une femme.

— Tersant, je te hais.

— Moi aussi je t'aime.

— Au fait, on a vu l'enregistrement de l'audition de Diaz. Et on est d'accord avec toi, y'a de la matière pour mener 
l'enquête. Bon Dieu, ce que ce pauvre gars a enduré.

— Ouais. Et c'est pour ça que tu vas me trouver cette nana pour qu'on puisse l'interroger.

— Je m'y mets de suite. 

— Merci Marco. À plus tard. »

Je raccrochais et retrouvais le commissaire au bar. Son verre était à moité vide. Et le mien m'attendais bien sagement 
sur le comptoir.

« Marco s'y colle tout de suite. Il nous tiendra au courant si il trouve quelque chose.

— Merci Tersant. Puis-je vous demander une faveur ? »

Je repassais instantanément sur la défensive. Qu'allait-elle encore trouver pour me torturer ?

« Je vous écoute.

— J'aimerais que vous considériez ce verre comme le calumet de la paix. Je ne voulais pas vous blesser tout à l'heure. 
J'ai bien vu que j'y étais allé un peu fort — tu l'as dit oui, enfin moi je n'aurais pas dit un peu — et j'aimerais 
vous présenter mes excuses. Vous savez, je suis des fois un peu trop directe, je sais que ça peut blesser sans que je 
n'en ait l'intention. J'aimerais vraiment faire la paix avec vous. Nous avons pas mal de boulot et ça serait bien 
que ça se passe dans les meilleures conditions possibles. Je voudrais savoir que je peux compter sur mon capitaine. »

C'est alors que je fis quelque chose que je n'aurais jamais pensé faire. Quelque chose de dingue. « Excuses acceptées 
commissaire. Triquons. »

Et dans le bruit des verres s'entrechoquant je vis sur le visage de ma supérieure un sourire radieux qui m'enchanta.