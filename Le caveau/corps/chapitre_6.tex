\chapter[Chapitre Six]{Chapitre Six}
Le professeur s'éclaircit la voix et reprit :

« Voyez-vous, mon problème c'est que je suis absolument incapable de vous donner une datation précise pour cette crypte.
Pas plus qu'il ne m'est possible de vous donner une datation imprécise.

-- J'ai peur de ne pas vous suivre professeur.

-- Et bien, ce que à quoi vous êtes confrontés est tout bonnement impossible.

-- Comment ?

-- Comment vous expliquer\ldots{} La crypte présente des aspects architecturaux qui laisseraient penser qu'elle a été
construite sous l'Empire Romain, puis utilisée pendant le moyen-âge, restaurée pendant la révolution industrielle et
enfin largement utilisée à notre époque. Or, comme vous le savez, les Romains n'ont pas étendu leur domination au
nouveau monde.

-- Attendez, vous êtes en train de me dire que cet endroit ne peut pas avoir été construit ici ?

-- C'est tout à fait ça. La seule explication logique est que ce caveau a été transporté pierres par pierres depuis
l'Europe et remonté ici à l'identique. Cependant, pour confirmer ou infirmer cette hypothèse, il nous faudrait procéder
à des fouilles archéologiques sur le site.

-- Et nous risquons d'avoir du mal à obtenir une autorisation rapidement. Nous ne savons même pas à qui appartient ce
caveau. Il n'est pas dans les registres du cimetière, ajouta Gordy. Nous avons vérifié au passage.

-- Bon. Admettons. Ce caveau a été construit en Europe, démonté, transporté, puis reconstruit ici. Ce genre de choses
doit laisser des traces quelque part non ? Gordy, tu te mets là dessus, et tu me déniches tout ce que tu peux. Coupures
de journaux, registres paroissiaux, anciens plans du cadastre. Tout ce qui a un rapport avec cet endroit, je le veux au
plus tôt.

-- Lieutenant, je peux obtenir, si vous le désirez, un accès aux archives de l'université pour votre inspecteur.

-- Cela nous serait d'une grande aide professeur. Je vous en remercie.

-- Ce n'est rien. Je dois avouer que ceci m'intrigue au plus haut point. Et que cela pique ma curiosité. J'aimerais en
savoir plus moi aussi. »

J'accusais le coup. Choux blanc pour l'instant. Il allait falloir des semaines avant que nous puissions trouver des
informations pertinentes sur cette endroit. Soudain, l'image de la statue me revint à l'esprit.

« Professeur, puisque vous êtes là, pourriez vous jeter un œil à un objet que nous avons trouvé dans la crypte ?

-- Mais avec plaisir, dit-il avec un soupçon d'excitation dans la voix.

-- Voilà, j'attrapai la statue dans mon tiroir, ôtait l'emballage et lui présentait. Nous avons trouvé ceci dans une
niche cachée sous l'espèce d'autel. »

Il observa la statue sous tous les angles, la faisant tourner entre ses mains fines et habiles. De temps à autres, il
marmonna des mots inintelligibles puis au bout d'environ dix minutes d'observation reposa l'horrible chose sur mon
bureau.

« Voilà qui est fort intéressant lieutenant. Me permettez-vous d'en prendre des photographies ? Une de mes collègues
spécialisée dans les mythes et légendes modernes sera ravie de les voir.

-- Savez-vous ce que c'est ?

-- Bien sûr lieutenant. Il s'agit d'une représentation du Grand Cthulhu. Il est parfaitement aux descriptions qui en
sont faites habituellement.

-- Mais ? Il n'y a donc que moi qui ne connaît pas ce truc ici ! M'exclamai-je.

-- Non John, je ne connais pas non plus.

-- Merci Gordy, tu me rassures. Professeur, comment connaissez-vous l'existence de cette chose ?

-- Disons que certains ouvrages gardés précieusement à l'université en font mention. Ces ouvrages ont toujours été
sujets à caution. Mais dites-moi, savez en quel matière est réalisée cette statue ?

-- Nous n'avons pas encore procédé aux analyses. Mais nous le saurons bientôt.

-- Une fois encore, je peux faire mettre à disposition nos laboratoires si cela peut vous aider.

-- Merci professeur. Nous allons en discuter et nous vous tiendrons au courant.

-- J'aimerais vraiment en savoir plus sur cette statuette. Elle est remarquable de finesse. Et je pense qu'elle réserve
quelques surprises.

-- J'ai moi aussi cette intuition professeur. »

Il se leva, sortit un appareil photo d'une de ses poches et prit quelques clichés de l'hideuse sculpture. Une fois ceci
fait, il me serra la main et quitta mon bureau.

« Gordy, j'ai bien l'impression que je ne resterais pas assez longtemps en service pour voir la fin de cette histoire.

-- J'en ai bien peur John, j'en ai bien peur. »
