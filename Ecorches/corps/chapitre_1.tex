\chapter{}

\lettrine[lines=1]{L}{e} neuro-implant me démangeait. Une démangeaison profonde. Avec un gros problème. Il est
impossible de gratter quelque chose situé à l'intérieur de soi.

Les toubibs m'avaient prévenu. Pendant quelque jours après la pose de l'implant, j'allais ressentir de temps à autre
une « petite gène ». Petite, mon œil ! C'était intolérable. L'impression qu'une armée de fourmis utilisaient mes nerf
comme des toboggans. Je pris une des pilules magiques qui m'avaient été prescrites à la sortie de l'hôpital. Elles
étaient censée faire disparaître la gène en quelques minutes. C'est du moins le baratin que m'avait servi le
pharmacien en me remettant en mains la boîte. « Attention, c'est puissant comme médicament. On recense quelques cas
d'addiction ». Il avait rajouté ces derniers mots au moment où je sortais ma carte pour payer. « Merci Bien, bonne
journée monsieur » avec un rictus désagréable.

Je rentrais dans ma voiture et programmais la direction du commissariat central place du Capitole, en avalant un de ces
comprimés magiques. J'allais voir si le pharmacien avait raison ou si son petit sourire en coin était juste une manière
de se foutre de moi. Alors que la voiture filait le long des voies réservées je pensais à tout ce qui avait amené à ce
que petit à petit, chaque flic ou gendarme doive être équipé d'un neuro-implant.

Il avait été rendu obligatoire pour toutes les forces de police depuis quelques années. J'avais pour l'instant réussi à
passer mon tour et à repousser l'échéance, n'aimant pas vraiment l'idée d'avoir de l'électronique directement nichée au
creux des méandres de mon cerveau. C'était soit disant pour éviter les bavures. Elles avaient été rendues obligatoire à
la suite d'une erreur judiciaire retentissante où un suspect avait été forcé d'avouer pour une série de crime qu'il
n'avait pas commis. Dans le secret de la salle d'interrogatoire, il arrivait quelques fois que nous malmenions un peu
nos suspects mais il avait fallut qu'un flic un peu trop zélé face à un type dont la gueule ne lui revenait pour qu'on
en arrive à ça.

Le pauvre type avait été tellement malmené qu'il avait endossé la responsabilité pour une dizaine de viols sur mineurs.
Avec des actes de torture en prime. Ça avait été une sale affaire. Pas d'ADN exploitable, les gamins avaient donné une
description vague de leur agresseur qui frappait toujours entre chien et loup. Et ce pauvre hère avait eu le malheur de
traîner là où il ne fallait pas quand il ne fallait pas. Il avait été vu par plusieurs personnes aux environs des lieux
des crimes à des heures qui collaient à peu près et surtout il avait la tête de l'emploi.

Que voulez-vous ? Un éboueur quarantenaire, un peu porté sur la boisson, bâti comme un monstre de foire, sans femme ni
enfants, sans relations à long terme connue. Les enquêteurs n'avaient rien trouvé chez lui de probant. Pas d'images
pédopornographiques, rien. Mais il n'était pas connecté au Réseau. Et ça, c'était suspect. Seuls les gens qui ont
quelque chose à cacher ou à se reprocher ne sont pas connectés. Seuls les criminels, les instables et les marginaux ne
racontent pas leur vie sur les réseaux sociaux et ne passent pas leur temps à publier des photos d'eux faisant
n'importe quoi.

Rajoutez à ça qu'il avait une gigantesque collection de « bouquins » et qu'il affirmait les avoir tous lu. Plus
personne ne lisait de livres depuis des années. Les tablette reliées au Réseau permettaient pour une modique somme de
louer un « livre ». Pourquoi avoir des livres papier ? Ça polluait, participait à la déforestation, c'était lourd,
encombrant, fragile, ça pourrissait. Et surtout, vous pouviez lire des livres que le Grand Conseil avait strictement
interdit à cause des idées subversives qu'ils contenaient. La paix dans notre société en dépendait.

Après son arrestation, la série de viols s'était interrompue et cela n'avait que conforté les enquêteurs dans leur
sentiment d'avoir serré la bonne personne. L'enquête, bouclée — bâclée disaient certains — en quelques jours avait
permis l'arrêt des agressions et les enfants pouvaient de nouveau sortir jouer.

Au tribunal il criait son innocence à qui voulait l'entendre. Il avait hérité d'un avocat débutant et en avait pris pour
perpète. La presse s'était intéressée à son cas et il avait fait les premières pages des journaux pendant des semaines.
Son visage était entré dans la conscience collective comme celui du mal absolu.

Jusqu'à ce que les viols reprennent. Même mode opératoire. Même type de victimes. Même violeur. Certains détails
scabreux non révélés à la presse étaient présents, indiquant pas là même qu'il ne pouvait s'agir d'un imitateur. Le
prédateur était toujours en liberté et un innocent avait été envoyé en centre de réhabilitation — il y avait des années
qu'on ne disait plus prison — pour des crimes qu'ils n'avait pas commis.

Brisé par la machine judiciaire implacable et par les médias qui l'avaient transformé en monstre sans âme, le pauvre
homme avait décidé de mettre fin à ses jours.

Une enquête des services internes fut diligentée. Elle révéla que les inspecteurs d'alors avaient extirpé des aveux
sous la contrainte, profitant de la faiblesse d'un homme et n'avaient pas respecté les procédures en empêchant que les
interrogatoires soient filmés. Les comptes rendus d'audition avaient été modifiés après coup. Un coupable crédible
devait être jeté en pâture aux médias. Il fallait calmer la populace et ce pauvre éboueur avait été le bouc émissaire
rêvé. 

L'affaire fut rejugée de façon posthume et les chefs d'accusation modifiés en « Recel de matériel de propagande ». La
peine fut allégée. Mais bien trop tard.

Les ramifications de l'affaire montrèrent que des pressions venant des hautes sphères avaient été exercées et amenèrent
à la démission du Ministre de l'Intérieur de l'époque et il fut décidé de lancer un programme de recherche permettant le
développement de technologies qui permettraient d'éviter de nouveau ce scandale.

C'est de là qu'étaient nés les neuro-implants aujourd'hui obligatoire. Dès lors qu'on entrait en salle d'interrogatoire,
ils ne pouvaient pas être désactivés et enregistraient fidèlement toute la séance. Un « filigrane » unique, dépendant de
chaque inspecteur était apposé sur les enregistrements et les développeurs assuraient qu'il était humainement impossible
de contrefaire ou d'altérer les données. Dès qu'elles étaient captées, elles étaient automatiquement répliquées dans
plusieurs centres géographiquement distants, chiffrées et archivées pour pouvoir être utilisée comme preuves.

Des années de « Vidéo-protection », de traçage de vos moindres faits et gestes, de fichage systématique dans des bases
de données de plus en plus interconnectées, de personnes prenant l'habitude de tout révéler leur vie sur le Réseau —
souvenez-vous, seuls ceux qui ont quelque chose à se reprocher ne sont pas connectés au Réseau — avaient permis de
préparer le terrain.

Bien sûr, vous imaginez bien que des activistes de la « Liberté » s'étaient indignés contre ces mesures. Pointant du
doigt les risques d'atteinte à la vie privée, les dérives potentielles du système. Mais la population était prête à
accepter ces petites entorses à la liberté pourvu que cela permette à la Justice d'être plus sûre et plus efficace.

Je fait partie des rares flics qui ne voulaient pas de cet implant. Mes convictions profondes se révoltaient face à cet
état de fait. Cela avait valu à ma carrière un sacré coup de frein d'ailleurs mais je gardais de bons résultats quand
même. En utilisant les bonnes vieilles méthodes enseignées par mon père, juge d'instruction désormais retraité mais qui
avait connu le temps béni où la société faisait plus confiance aux gens qu'aux machines.

Je ne croisais aucune autre voiture dans les rues animées et bondées de chalands qui s'écartaient de mon chemin. Depuis
que tout les véhicules à moteur avaient été bannis de la ville pour favoriser les transports en commun moins polluants,
seuls les services d'état disposaient de véhicules individuels. Un des rares privilèges auquel ma fonction donnait
droit.

Ma voiture s'engouffra dans le parking souterrain de la place, maintenant réservé aux véhicules officiels. Elle alla se
garer sur sa place attitrée et la voix suave du GPS débita son « Vous êtes arrivé à destination ». Je détestais cette
voix. Je ne comprendrai jamais pourquoi il fallait obligatoirement que ma voiture me parle sur un ton sensuel, à la
limite de l'obscène. La portière s'ouvrit et je me dirigeais alors vers l'ascenseur d'un pas rapide. La démangeaison
avait baissé d'intensité et devenait plus supportable. Finalement, le pharmacien avait peut-être raison.

Alors que je présentais mon badge à l'œil électronique de l'ascenseur, mon téléphone vibra furieusement dans ma poche.
La cadence de la vibration indiquait qu'un message important venait d'arriver. La porte s'ouvrit au moment où
j'attrapais mon téléphone. En pénétrant dans la petite boîte exiguë, je ne pus retenir un frisson. Tout ce métal
étincelant et froid me donnait toujours l'impression que je rentrais dans une cellule sans vie, prête à m'engloutir à
jamais. La plupart du temps, je prenais les escaliers. Mais, depuis qu'il avait été décrété que les escaliers étaient
une chose dangereuse — trop d'accidents inutiles —, ces derniers étaient réservés aux évacuations d'urgence. Ça ne
manquait pas d'une certaine ironie. On recensait plus d'accidents durant les évacuations, les gens n'ayant plus
l'habitude de dévaler des marches rapidement. Mais comme les évacuations se faisaient de plus en plus rares, le nombre
d'accidents restait en dessous de celui en utilisation normale.

Le message sur mon téléphone indiquait que ma présence était requise au plus vite. Une nouvelle affaire assez spéciale
requérait ma présence. Les étages défilaient et la porte s'ouvrit enfin à l'étage de la section criminelle. Le long
couloir desservait les nombreux bureaux et les salles de réunion. Tout au bout se trouvait le vaste bureau du
commissaire et je dirigeais mes pas vers cet endroit. Après avoir légèrement frappé à la porte de verre je rentrai dans
la pièce sans attendre confirmation. La commissaire était assise à son bureau et elle regardait d'un air absent le fin
écran devant elle.

Son regard se posa sur moi, légèrement au dessus de ses fine lunettes cerclées de métal. Bien qu'elle aurait pu se faire
corriger chirurgicalement son défaut de vue, elle continuait à porter des lunettes. Ce qui, je devais avoir, lui donnait
un certain charme légèrement suranné. Son bureau était sobre et dépouillé de tout le superflu. Elle ressemblait à une
vierge de glace dans son royaume. Froide et incisive. Nous la surnommions « Ice Queen » dans son dos bien que je
soupçonnais qu'elle le savait parfaitement et qu'elle feignait l'ignorance.

« Ah, Tersant, vous voilà. Asseyez-vous.

— J'ai eu votre message. Qu'elle est donc cette affaire bizarre qui doit retenir mon attention ? Lui lançai-je en
m'asseyant.

— C'est assez hors du commun. Un homme a été retrouvé nu, errant le long de l'autoroute. Il présentait des blessures et
paraissait en état de choc d'après les retours des pompiers. Ils l'ont transporté vers Purpan.

— Bon, un fou à poil le long de l'autoroute. En quoi est-ce que la crime à quelque chose à voir dans cette histoire ? »

Le commissaire ôta ses lunettes et entreprit de les nettoyer. Même sans ses lunettes elle restait attirante et je ne pus
m'empêcher de jeter un coup d'œil à son décolleté mis en valeur par un tailleur sur mesure. Elle sembla s'en rendre
compte mais ne parut pas s'en formaliser outre mesure. En remettant ses lunettes, elle me dit d'une voix douce « Il
affirme qu'il a été enlevé, qu'il était ligoté à une table d'autopsie et qu'il a pu s'échapper par miracle. »

C'était bien ma veine, un illuminé. Un doux dingue.

« Et qu'en disent les psy ?

— Il paraît sain d'esprit et n'est pas délirant. À vous de jouer maintenant Tersant. Le dossier est accessible sur le
Réseau. »

Je me levais, la saluais brièvement et redescendais au plus vite au parking. Une seule phrase résonnait dans mon
esprit : « Mais quelle affaire de merde ! » Je savais que je n'étais pas dans les bonnes grâces du commissaire mais là,
me coller sur ça… J'avais comme la vague impression qu'elle désirait vraiment m'en faire baver pour avoir si longtemps
refusé l'implant.

Le commissaire faisait partie des zélotes prêchant la surveillance à tout crin. Elle vouait une espèce d'idolâtrie sans
bornes à tout ce qui était électronique et ne jurait que par les \emph{grandes avancées} dans la lutte contre le crime
permises par toutes ces babioles. Ça me mettait hors de moi et cela nous avait valu elle et moi quelques frictions. Et
depuis, j'avais l'impression qu'elle me faisait payer ce qu'elle devait considérer comme de l'insubordination. Je ne
comprenais pas trop comment elle fonctionnait avec moi. Un coup froide comme l'acier, plus tard chaleureuse, presque
amicale.

L'ouverture de la porte de l'ascenseur sur le parking mit fin à mes tergiversations et je m'engouffrais dans la
voiture. En programmant la destination j'accédais au dossier via le Réseau. L'affichage tête haute s'illumina et les
principales informations se dessinèrent en lettres brillantes sur le pare-brise. Je devais concéder à la technologie
que les voitures entièrement automatisée avaient quand même l'avantage de permettre de lire au volant sans danger.

Mon homme s'appelait Francisco Diaz, était d'origine espagnole, avait 31 ans. C'était un musicien qui cachetonnait à
droite et à gauche. Deux condamnations mineures sur son casier : possession de drogue en vue de sa consommation et
ébriété sur la voie publique. Plutôt beau gosse d'après sa photo. Le type même du fêtard qui abuse un peu et qui ne sait
plus le nom de la nana qu'il a levé le matin au réveil. Il devait avoir pris un truc bizarre et a préféré inventer cette
histoire plutôt que d'avouer qu'il était complètement défoncé et qu'il était incapable de se rappeler quoique ce soit.
Sa disparition avait été signalée deux jours plus tôt par les membres de son groupe qui s'étonnaient de ne pas le voir à
une répétition.

Lien vers son identité numérique… Et voilà… Comme je m'en doutais, encore un qui balançait tout de sa vie privée à tout
le monde. Je passais rapidement sur les photos, dont certaines étaient répréhensibles au regard des dernières lois sur
la protection des mineurs et remarquait que l'intégralité de son emploi du temps figurait en bonne place dans son petit
monde électronique. J'appris donc que la veille du signalement de sa disparition, il devait aller boire un coup avec
les membres de son groupe.

D'un geste rapide, j'accédais à ses relevés de carte bancaire et je découvris vite que des consommations avaient été
payé la veille du signalement de sa disparition dans un bar du centre.

Ouverture de la page du bar. Tiens, un bar spécialisé dans les bières belges. Intéressant. Il avait ouvert quelques
mois plus tôt. Je ne le connaissais pas encore. Il allait falloir que j'aille y faire un tour, pour demander au barman
si il se souvenait de notre homme.

Même si l'affaire me paraissait simplissime, j'aimais faire les choses bien et ne rien laisser au hasard. Je crois que
cette qualité — certains me considéraient trop tatillon — qui me permettait de rester flic malgré les rapports pas
forcément favorables rédigés par mes différents supérieurs au cours de ma carrière.

Ma voiture stoppa au poste de garde de l'hôpital et je déclinais mon identité et la raison de ma venue au planton qui
m'indiqua le bâtiment où attendait mon fêtard.