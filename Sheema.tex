La jeune fille s'assit sur le bord de son lit. La décoration de la pièce était
réduite à sa plus simple expression et on n'y trouvait que le strict minimum et
fonctionnel. Elle passa sa main dans sa longue chevelure rousse puis sortit de
sa veste un enregistreur vocal. Elle pressa un bouton situé sur le côté et se
mit à parler :

— Journal de bord du presque lieutenant Rahelia Sheema. Date 6 point 2 point
8569 AG. Je ne sais pas pourquoi j'ai décidé de tenir ce journal mais je sens
que ce sera un jour important. Et je n'ai aucune idée de pourquoi journal de
bord. Je ne suis même pas encore sur un vaisseau ! Enfin, il y a suffisamment
d'espace dans cet enregistreur pour sauvegarder tout ce que je pourrais dire
jusqu'à la fin de mes jours. Cependant, une sauvegarde sera automatiquement
envoyée après chaque entrée par communication sub-spatiale vers mes quartiers
sur Jonction où je me trouve actuellement. Je viens d'arriver à l'académie. Si
tout se passe bien, j'en sortirai d'ici 6 mois avec la connaissance suffisante
pour intégrer une équipe de rétablissement. Moi qui ait toujours voulu voyager
et découvrir de nouvelles choses je devrais être servie. La journée a été quasi
entièrement consacrée aux formalités administratives. Je n'en reviens toujours
pas, comment à notre époque peut-on encore mettre autant de temps à vérifier
l'identité d'une personne et à vérifier qu'elle est bien inscrite dans la bonne
formation. Mais bon, dans 6 mois j'aurais enfin fini mes études et je pourrais
découvrir le vaste monde ! »

Pendant qu'elle parlait, la jeune fille se leva et alla appuyer sur un bouton
discret sur le mur opposé au lit. Une projection de la galaxie apparut et elle
pianota sur un petit clavier qui venait de sortir de mur. L'image zooma vers une
une partie située presque au bord de la galaxie et finit par s'arrêter sur le
système Sol. Sa planète natale, Sol 3, était là. Accompagnée de son unique
satellite naturel, Sol 3-1. Dans les comptes pour enfants le satellite était
appelé Lüne, Leune ou quelque chose d'approchant. Elle avait du mal à se
souvenir, ces histoires de grand-mères ne l'avaient jamais trop captivée.

Toujours observant sa planète natale décrire son ellipse autour de son étoile
elle reprit son récit :

— Les six mois qui viennent doivent me permettre d'acquérir les connaissances
nécessaires à mon intégration dans une équipe de correction. Nous serons envoyés
sur les planètes de niveau D et inférieur pour corriger les interactions avec
des espèces plus évoluées. Il ne faut surtout pas que des espèces plus évoluées
viennent influencer le développement normal des planètes.

Elle se dirigea vers la partie de ses quartiers qui faisaient office de cuisine,
effleura un contact et quelques secondes plus tard un café chaud apparut dans
une alcôve. Elle prit la tasse chaude avec précaution et la porta à ses lèvres.

— Où en étais-je ? Ah oui, je parlais du principe de non ingérence. Enfin, vous
le savez tout aussi bien que moi, les espèces les plus avancées
technologiquement ou maîtrisant les arcanes ont la plus stricte interdiction de
contaminer les planètes de catégories D et inférieures. Nous ne pouvons nous
permettre de refaire les mêmes erreurs qui amenèrent au déclenchement des
Grandes Guerres.

Un instant, elle se demanda à qui pouvait bien être destiné cette série
d'enregistrements qu'elle commençait. Elle n'en avait aucune idée mais elle
ressentait le besoin de le faire. Comme si une force impérieuse tapie au fond de
son inconscient lui intimait l'ordre de le faire. Elle était sûre que ces
enregistrements serviraient à quelqu'un, un jour quelque part. Un rapide frisson
d'angoisse parcouru son dos et elle ne put s'empêcher de trembler un peu. La
gorgée de café qu'elle avala lui fit du bien en réchauffant son corps.

— Je me trouve actuellement dans mes quartiers à l'Académie. C'est une pièce
d'environ 25 mètres carrés contenant tout ce qu'il faut pour vivre en ermite
pendant des années. Un lit, une table, deux chaises, un synthétiseur de
nourriture - oh j'aime composer des menus improbables - et un coin sanitaires.
Les murs sont gris pâles. Ce n'est pas franchement gai mais pas trop triste non
plus. Je suis sûre que cette couleur a été judicieusement choisie pour ne
heurter aucune race. C'est le problème de cohabiter avec des tas de races
différentes. On m'a expliqué que la pièce avait été configurée pour un être
humanoïde bipède. J'imagine que chaque chambre est différente en fonction de la
race de la personne, l'entité — je n'ai jamais réussi à trouver un mot correct
pour désigner les autres — qui y habite. Enfin, je divague là. J'espère trouver
le temps de rédiger mon journal tous les jours. En attendant, fin de cette
première journée, je suis harassée.

Sheema se dirigea alors vers les sanitaires et ôta ses vêtements avant de se
glisser sous la douche bienfaisante. Elle composa son menu du soir, un civet
d'anguille de FIXME et rejoignit sa couche avant de commander aux lumières de
s'éteindre. Demain serait sûrement une longue journée. encore affichée sur le
mur, la simulation de la galaxie parut frémir alors que la presque lieutenant
sombrait dans le sommeil.


Le réveil sonna et Sheema sorti avec difficulté de son lit. Sa nuit
avait été agitée. Elle était collant de sueur et sa longue chevelure rousse
était emmêlée. Elle ressentait une douleur tenace dans le dos, comme si elle
avait dormi couchée sur une pierre ou comme si elle avait reçu un coup violent.
Tout son corps était engourdi.

— Eh bien ma grande tu t'es battue cette nuit ? lança-t-elle alors vers le mur.
Personne ne lui répondit alors si ce n'est le silence.

Trois mois avaient passé depuis son arrivée à l'académie. Elle n'allait pas
tarder à passer un des examens les plus importants de sa vie. Les résultats
décideraient si elle pouvait continuer les études où si il fallait qu'elle
change de carrière.

 — Et forcément, c'est ce jour là que tu es moulue !

Son reflet dans le miroir ne lui répondit pas.

— Voilà que je parle toute seule. Ma grande, il va falloir que tu dormes un peu
plus. Tu es en train de devenir dingue.

Elle sauta dans la douche et se délassa un long moment avant de s'atteler à sa
coiffure. Ses long cheveux roux étaient sa fierté et elle passait chaque matin
de longues minutes à réaliser des coiffures toujours plus extravagantes mais
pratiques.

Un coup d'œil à la pendule l'informa qu'elle avait largement le temps de se
préparer pour son examen et qu'elle pouvait même s'offrir un bon petit déjeuner.
Elle tapota avec dextérité une combinaison sur le clavier du synthétiseur de
nourriture et son petit déjeuner apparût quelques secondes plus tard dans
l’alcôve.

  — J'aimerais quand même un jour manger quelque chose de non synthétisé.

Elle regarda le synthétiseur et se souvint de la première fois qu'elle avait
voulu démonter quelque chose. Elle avait huit ans et elle était chez elle.
C'était un jour où elle n'avait pas classe et elle s'ennuyait. Elle s'était
toujours demandé comment la nourriture apparaissait par magie dans les creux du
mur. Elle avait remarqué qu'il y avait une petite trappe de maintenance à côté
du clavier. Un jour que le synthétiseur était en panne, elle avait vu un
technicien y farfouiller. Elle réfléchit un moment en regardant la trappe et
alla chercher le tournevis dans le placard de la cave. Elle n'avait aucune idée
de ce que pouvait être une vis et pourquoi il fallait les tourner mais elle
savait qu'avec le tournevis elle pourrait ouvrir la trappe et découvrir les
lutins cachés qui créaient la nourriture. Elle était sûre que de minuscules
êtres vivaient dans le synthétiseur que que le clavier permettait aux humains de
leur dire ce qu'ils voulaient que le gnomes préparent à la vitesse de l'éclair.

Gnomes ou lutins ou autre chose, il fallait qu'elle en ait le cœur net. Quelque
chose là dedans créait la nourriture et aujourd'hui elle allait découvrir quoi.
La porte du placard lui résista un petit moment. Légèrement voilée, elle
coinçait et il fallait une force herculéenne pour l'ouvrir. Papa avait dit qu'il
la réparerait mais il avait oublié. La porte s'ouvrit enfin avec un grincement.
Une goutte de sueur perlait sur le visage de la petite. Elle se saisit alors du
tournevis posé sur l'étagère. C'était un objet carré d'environ deux centimètres
d'épaisseur de quatre de côtés. De nombreux boutons se trouvaient sur une face.
Elle appuya sur l'un d'eux. Rien ne se passa. Elle appuya sur un second, rouge.

— Merci d'avoir activé le mode d'emploi de votre tournevis universel. Que
puis-je faire pour vous ?

Avec un cri de surprise elle lâcha l'objet qu'elle tenait dans les mains et le
tournevis tomba au sol dans un bruit métallique.

  — Et bien jeune fille ! En voilà une manière de traiter un mode d'emploi !

Un hologramme était projeté depuis le tournevis et il représentait un homme en
costume sombre à rayures blanches. Il portait des lunettes et semblait
s'adresser directement à la petite Rahélia.

  − Qui… Qui êtes-vous monsieur ?

— Le mode d'emploi bien sûr ! Tu n'as jamais utilisé de mode d'emploi
holographique ?

  — Non monsieur. C'est… C'est la première fois.

  — Ah d'accord d'accord.

L'hologramme se mit à marcher de long en large dans la cave.

  — Voyons voir reprit-il. Que puis-je faire pour toi ?

  — Je… Je voudrais voir les lutins du synthétiseur de nourriture.

  — Pardon ? Les lutins ?

— Oui, vous savez monsieur, les lutins qui préparent la nourriture dans le
synthétiseur. Ceux à qui on passe commande d'un milkshake et qui le font
apparaître immédiatement.

  — Je ne suis pas programmé pour les lutins ma petite mais…

  — Je ne suis pas petite !

Rahélia s'était redressée brusquement et levait le menton en signe de défi vers
l'hologramme.

— Non, tu n'est pas petite. Tu as raison. Disons, que je suis plus grand que
toi. Tu vois ? Te es obligée de lever la tête pour me parler.

— Mais ? Monsieur ? Je ne comprends pas, je peux voir à travers vous et vous me
voyez. Vous êtes un fantôme ?

  — Non, pas du tout. Je suis juste une image.

La fillette entreprit alors de faire le tour de l'hologramme. Celui-ci ne réagit
pas et la laissa faire sans l'interrompre. Elle revint se planter devant lui et
l'apostropha:

— D’accord, vous êtes une image. Mais ça ne répond pas à ma question. Comment
est-ce que je peux faire pour voir les lutins ? J'ai vu mon père se servir de la
boîte pour ouvrir des choses. Expliquez-moi !

— Attends un peu. Je vais d'abord trouver une forme un peu plus adaptée. Laisse
moi deux secondes s'il te plaît.

L'image changea et à la place de l'homme se trouvait maintenant un enfant aux
traits fins paraissant avoir une dizaine d'années.

  — Comment tu peux faire ça ?

  — Je suis un hologramme.

  — Un quoi ?

— Un hologramme. Je suis une image tri-dimensionnelle projetée au travers de
prismes. Je suis piloté par l'intelligence artificielle située dans le boîtier
que tu tenais et tu m'as activé.

— Une image tri-quoi ? Projetée à travers quoi ? Monsieur, je n'ai rien compris.

— Tu peux me tutoyer tu sais, je suis un enfant maintenant.

  — D'accord. Comment tu t'appelles ? Moi c'est Rahélia. Rahélia Sheema.

  — Euh… Je m'appelle KRSD4X259-Sigma-7. Mais tu peux m'appeler KR si tu veux.

  — D’accord Caer.

  — Pas Caer. Mais K-R. Comme les lettres.

  — D'accord. K-R. Mais comment ça marche ?

  — Je ne suis malheureusement pas programmé pour répondre à ta question.

  — T'es nul.

  — Mais, j'ai peut-être une solution pour toi.

  — Ah bon ?

  — Est-ce que tu as un holodock dans la maison ?

  — Bien sûr ! Pour qui nous prenez-vous ?

— Comment ? Ah, hum, oui. Je vois. Alors si tu branches le tournevis dans
l'holodock, il sera relié à la maison et je pourrais accéder aux autres modes
d'emploi et répondre à tes questions. Je t'expliquerai comment réaliser le
branchement.

  — Je vais te brancher à la maison.

Rahélia prit le tournevis et remonta au salon. Là, en suivant les instructions
données par l'hologramme, elle brancha le tournevis et oublia qu'elle voulait
voir les lutins trop occupée à essayer d'apprendre comment fonctionnait un
hologramme. Ce fut le jour où elle sût que quand elle serait grande, elle
maîtriserait tout cela. Et qu'elle serait la meilleure.

Elle regarda sa montre. Elle venait de se perdre dans ses souvenirs et elle
était maintenant en retard. Elle attrapa sa tablette sur la table, enfila sa
veste et couru dans le couloir vers l’ascenseur abandonnant le reste de son
petit-déjeuner.




Sheema commanda une bière accoudée au bar. Demain aurait lieu la remise des
diplômes et les affectations aux différentes équipes. Elle allait enfin voir
l'Univers ! Enfin.

Sa première affectation venait de lui être communiquée. Elle allait rejoindre
son équipe dès le lendemain. Elle faisait désormais partie d'une équipe de
Rétablissement. Elle serait le Lieutenant Ingénieur de bord.

--- Pourvu que ma première mission arrive vite

--- Pardon mademoiselle ? Je n'ai pas bien compris votre commande.

Sheema regarda le barman qui lui faisait face.

--- Non non rien. Je marmonnais.

--- A votre service mademoiselle. Quelque soit votre désir.

Elle regarda plus attentivement le barman. Est-ce qu'il était en train de la
draguer ? Il était certes plutôt à son goût mais elle n'avait pas la tête à ça.
Du moins, pas ce soir.

Elle termina son verre, régla l'addition et retourna à ses quartiers.

Assise sur son lit, elle sortit son enregistreur de poche pour son rituel du
soir :

--- Journal du lieutenant Sheema. Date 10 point 8 point 8569 AG. Oui,
lieutenant. Ca y est. Je vais pouvoir parcourir la galaxie. Je ressens à la fois
une grande excitation et une immense peur. Excitée à l'idée de voir d'autres
mondes, de découvrir d'autres civilisations et de faire en sorte de les
préserver de toute ingérence. Mais terrifiée à l'idée de ne pas y arriver.
Pourtant, j'ai réussi avec brio tous les tests pour arriver là où je suis
maintenant. Mais je ne peux m'empêcher de penser au pire.

Elle s'allongea avec un bâillement.

--- J'ai encore fait ce rêve étrange cette nuit. Celui où je suis dans une
grotte et où je m'approche d'une shpère lumineuse. Quelqu'un crie mon nom dans
mon dos. Je ne reconnais pas la voix et je ne me retourne pas. Je suis
hypnotisée par la sphère et je tends la main vers elle. Elle donne l'impression
d'avoir une vie propre. Elle est là, flottant dans un champ de force. Du coin de
l'oeil je distingue des consoles de ce qui semble être un grand ordinateur.
Lorsque je vais toucher la sphère, je ressens une grande douleur. Comme si
j'étais traversée d'une décharge électrique. La voix qui m'appelait crie des
choses inintelligibles. Et je me réveille à ce moment là.

Sheema commanda à la lumière de se baisser et choisit de la musique d'ambiance
pour se relaxer.

--- Je ne comprends pas la signification de ce rêve. Si tant est qu'il en ait
une. Ce que je ne comprends pas, c'est qu'il paraît vraiment réel et qu'il
revient, toujours exactement pareil. Aucune différence. Et autant il ne se
produisait que rarement jusqu'à présent, cela devient de plus en plus souvent.
Il faudrait que j'aille voir un toubib pour savoir ce que ça veut dire. Mais ça
doit sans doute être le stress. Il faut juste que j'arrive à me calmer.

Elle éteignit l'enregistreur et se fit rapidement à diner puis alla se coucher.
Elle devait être en forme pour le lendemain. Sa première affectation.



