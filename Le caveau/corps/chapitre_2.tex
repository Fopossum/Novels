\chapter[Chapitre Deux]{Chapitre Deux}
Vingt minutes plus tard, une équipe du SWAT lourdement armée déboula devant la haute grille du cimetière éclaboussée 
par la lumière des gyrophares et des projecteurs. Les flashes rouges et bleus teintaient l'air et me rassuraient un peu. 
J'étais heureux d'avoir auprès de moi les collègues. Il y avait quelque chose de rassurant dans la présence de ces hommes 
autour de moi. Qu'ils soient là me ramenait à la réalité tangible et permettait d'apaiser le sentiment diffus de terreur 
qui s'était emparé de moi quelques minutes plus tôt.

Le commandant du détachement prit alors la parole.

« Messieurs, je tiens à ce que tout se passe dans le calme. Il est possible qu'une victime innocente soit retenue par 
des fous et nous devons tout faire pour qu'elle ressorte vivante. D'après les dires de l'inspecteur, je pense que nous 
avons affaire à des sectateurs d'un culte satanique ou à d'autres tarés de la même espèce et je les veux vivants pour 
pouvoir les interroger. Nous ne voulons pas d'un autre wacko. Personne n'étant sorti jusqu'à présent, il est fort 
probable qu'ils soient encore dans ce souterrain. Le couloir donnant accès à la salle où ils semblent se trouver n'est 
pas très large ce qui nous obligera à être deux de front au maximum. On ouvre, on balance les lacrymos et vous connaissez 
la chanson. »  

Ces hommes, malgré l'heure tardive étaient en pleine forme et prêts à en découdre. L'équipe qui était devant moi avait 
l'habitude de ce genre d'interventions musclées.

« Je veux quatre hommes dehors pour sécuriser le périmètre, un de plus restera à l'entrée du caveau. Le reste avec moi 
dans le couloir. Soyez sur vos gardes, on ne sait pas à quoi on doit s'attendre, nous ne savons pas si ils sont armés 
mais nous pouvons raisonnablement envisager que la résistance sera forte. Nous garderons le silence radio complet 
jusqu'à ce que nous soyons devant la porte. Des questions ? »

Aucun homme ne broncha. On pouvait lire dans leurs yeux la détermination à mener à bien cette opération.

Dans le cimetière tout était calme. Plus un bruit ne venait déranger la quiétude des morts. L'étrange lumière verte 
avait disparu. Les puissantes torches des SWAT illuminaient l'endroit. Nous investîmes le caveau désormais silencieux. 
Le couloir semblait vide. Aucune des torches ne brûlait dans le couloir. Aucune torche n'était même là. Je commençais 
à me demander si je n'avais pas rêvé lorsque qu'après s'être annoncés et n'avoir obtenu que le silence comme réponse 
aux sommations, le bélier défonça la serrure de la porte.

Les grenades volèrent à l'intérieur et une épaisse fumée emplit les lieux. Bizarrement aucune voix ne se fit entendre 
mises à part celles des policiers. Aucune plainte, rien. Les premiers hommes entrèrent et nous informèrent que la pièce 
était vide avant de ressortir. Nous attendîmes quelques minutes que les gaz se dissipent avant de pénétrer de nouveau 
dans cet antre où il me semblait que la Mort elle-même nous attendait.

Le spectacle qui s'offrit à nos yeux était atroce. Les jets de lumière crue des torches dévoilaient un autel maculé 
d'une substance sombre et poisseuse. Des signes cabalistiques étaient gravés dans la pierre et sur les mur de la salle. 
Elle était carrée et mesurait à vue d'œil dix mètres de côté. Une puanteur épouvantable prenait à la gorge. L'un des 
hommes s'appuya contre un mur et rendit son repas dans des gargouillis liquides.

Personne. Nous remontâmes à la surface. Les agents restés en haut n'avaient vu personne non plus. Les bandes jaunes 
interdisant l'accès aux lieux furent bientôt déployées tout autour de la zone et nous contactâmes les équipes 
scientifiques pour qu'elles fassent les première constatations. Quelques temps plus tard les uniformes sombres des SWAT 
furent remplacés par les blouses et charlottes plus claires des techniciens et agents. De lourds projecteurs 
furent descendus dans ce nous appelions tous désormais \emph{La Crypte}.

La puissante lumière des projecteurs illumina la salle toute entière chassant les ombres et les peurs des heures 
précédentes. La pièce semblait avoir été le théâtre d'une cérémonie sacrificielle en l'honneur d'une divinité avide de 
sang et de violence. Ce qui ressemblait à du sang avait giclé en longues éclaboussures depuis ce qui ressemblait à 
première à un autel et dessinait d'hideuses fresques sur le sol autour.

Les hommes de la scientifique procédaient aux prélèvements et photographies pour figer à jamais la scène. Les flashes 
crépitaient, des mesures étaient prises, chaque goutte de sang dûment numérotée, inventoriée. Chaque trace suspecte serait 
analysée, décortiquée. Des échantillons de toutes sortes étaient empaquetés dans des petits sacs plastiques qui 
disparaissaient dans de grandes malettes pour leur traitement ultérieur.

L'assistant du coroner s'approcha de moi. Il enleva ses petites lunettes rondes et entreprit de les nettoyer 
frénétiquement.

« Ce n'est vraiment pas joli à voir. Je vous fais un topo rapide ?

— Je vous écoute. Je sortis mon carnet.

— À en juger par la grande… Flaque de sang, la victime est morte. Je pense qu'elle a été égorgée voire littéralement 
saignée. Les artères fémorales ont été elles aussi entaillées d'après la forme de la tache sur l'autel. Elle n'a pas dû 
survivre plus de trois minutes à ce traitement. Autre chose. Je pense qu'il y a eu plusieurs victimes cette nuit. La 
quantité de sang présente est trop importante pour une seule personne. »

J'accusais le coup. Plusieurs victimes. Plusieurs personnes avaient été saignées dans cette crypte sinistre. Pour 
quelle raison ?

« Tant qu'à faire dans le morbide, on remarque aussi nettement des traces plus anciennes. Ce ne sont pas les premiers 
sacrifices que ce lieu connaît.

— Quoi ? J'ai bien entendu ?

— Oui… — Il détacha bien ses mots — Des traces de sang bien plus anciennes sont présentes sur l'autel et autour. Tout 
figurera dans mon rapport. Sur ce, si vous voulez bien m'excuser, j'ai de la paperasse à remplir. »

Il s'éloigna dans le couloir avec un petit signe de la main. Les techniciens étaient en train de remballer leur 
matériel. Juste avant d'arriver devant l'escalier, il se retourna et me héla :

« Ah, j'oubliais. Cette crypte est très ancienne. Je suis un peu archéologue à mes heures perdues. Et je peux vous 
garantir que vous allez avoir besoin d'eux. Si vous voulez, je peux vous conseiller quelques noms. Je joindrai leurs 
coordonnées à mon rapport. Bonne journée ! »

Je quittai les lieux et décidai d'aller dormir un peu. Il était environ huit heure du matin et je n'avais pas fermé 
l'œil depuis 36 heures. Il fallait que je sois en forme pour réussir à débrouiller cet écheveau.

Deux semaines. Il me restait deux semaines avant la retraite…
