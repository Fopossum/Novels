\chapter{Amnésie ?}

À peine arrivé au bureau d'accueil du service, un médecin s'approcha de moi et me tendis la main.
Il avait l'air parfaitement détendu et jovial. Un grand sourire se dessinait sur son visage. Je jetai un coup d'œil
rapide au badge ornant sa blouse : Docteur \textsc{Courtois}. À première vue, il portait bien son nom.

« Vous devez être le lieutenant Tersant ?

— Oui en effet. Que pouvez me dire sur notre homme.

— Il va bien. Du moins aussi bien que l'on peut aller après quelques heures de manche nu dans la campagne. Il a
quelques égratignures et les pieds dans un état lamentable. Mais rien de grave physiquement parlant. Par contre, il est
en état de choc et a un peu de mal à faire des phrases complètes.

— Vous voulez dire qu'il est… Je laissais ma voix s'éteindre en tamponnant ma tempe de mon index

— Non pas fou du tout. Uniquement choqué. Mais il s'en remettra vite avec un suivi correct. Et de vous rencontrer pour
que vous écoutiez son histoire lui permettra d'aller mieux. »

Je levais vers lui un regard interrogateur. Je voyais mal comment ma présence pouvait le faire aller mieux.

« Si vous voulez bien me suivre, je vous explique en chemin. »

Il se retourna et avança dans le couloir. Je restais à sa hauteur tout en notant mentalement la disposition des lieux.
Dans le cas où il faudrait un présence policière pour le protéger — ou le surveiller — je voulais avoir une petite idée
de l'emplacement des accès et de la topographie des lieux. Je sortais mon carnet de notes.

Le toubib reprit alors la parole et sa voix si fit un peu plus professorale :

« Tout d'abord, sachez que nos premiers examens psychiatrique ne montrent pas a priori de symptômes d'une maladie
mentale quelconque. Rien dans son dossier médical passé ne montre de signes précurseurs d'une pathologie mentale. Il
était parfaitement saint d'esprit avant cette… Mésaventure pourrait-on dire.

— Donc vous me dites que c'est un type tout à fait normal ?

— Disons, un peu trop fêtard pour que son foie tienne longtemps la cadence, mais à par ça, normal. Comme vous et moi,
rajouta-t-il avec un clin d'œil malicieux.

— Donc, à part son amour pour les boissons alcoolisées, rien de notable.

— Non. Parfaite condition physique. Il s'entretient régulièrement, ne fume pas. Mais revenons en à ce qui motive votre
présence ici. Ce qu'il décrit l'a profondément choqué. Il souffre d'amnésie rétrograde, vraisemblablement passagère…

— Amnésie rétrograde ? C'est à dire ? Le coupai-je rapidement tout en notant \emph{Amnésie ??}.

— Il ne se souvient pas des heures précédent son réveil. Cela est dû aux drogues que nous avons décelé dans son
organisme.

— On l'aurait drogué avant de l'enlever ? Nouvelle note dans mon carnet.

— On ne peut pas l'affirmer avec certitude. Nous avons trouvé des traces de GHB.

— Le GHB ? Je croyais cette drogue dépassée depuis longtemps. »

Il s'arrêta devant une porte. Le vert clair de la peinture agressait mes yeux et je sentais revenir la douleur de
l'implant. Lancinante. Toujours là. Il allait falloir que j'avale un autre cachet pour la calmer. Et avant d'interroger
la victime.

« Le fait qu'il y ait du GHB dans son sang n'indique pas qu'il l'ait absorbé volontairement. Même si on en trouve très
difficilement de nos jours, on connait quelques précurseurs qui métabolisés par l'organisme sont dégradé en GHB et ont
donc l'effet voulu. Il est tout à fait possible qu'il l'ait absorbé à son insu »

Alors qu'il m'expliquait cela, je sortis de ma poche la plaquette de comprimés et en j'en avalai un rapidement.

« Vous êtes souffrant ?

— Rien de grave, un mal de tête qui passera vite. Mais continuez je vous en prie.

— Comme vous voulez. Où en étais-je ? Ah, oui. l'amnésie. Un des effets du GHB peut être l'amnésie. Nous ne pouvons
dire si celle-ci est causée par la drogue ou par le choc. Seul le temps pourra apporter des réponses à cette question.
Nous allons de toute manière le garder en observation quelques jours et nous vous tiendrons bien évidemment au courant
dès qu'il y aura du nouveau.

— Merci. Puis-je vous recontacter en cas de besoin lors de l'enquête ?

— Bien sûr. Et je vais vous faire parvenir une copie de son dossier dans les plus brefs délais. »

Il me serra à nouveau la main et s'éloigna, me laissant devant la porte qui me séparait de ma victime.

Je notais rapidement sur mon carnet les quelques questions que j'allais lui poser. Maintenant que j'avais appris qu'il
était sain d'esprit et qu'il avait peut-être été drogué, cela changeait la donne. Je n'avais plus affaire à un bringueur
qui avait perdu un pari idiot mais peut être bien à un rescapé d'une expérience traumatisante.

Je levais la main et frappais doucement à la porte.

Une voix assourdie me donna la permission d'entrer.

Une chambre d'hôpital, classique. Un seul lit. Mon client avait les moyens. Il paraissait jeune et en bonne santé.
Seules quelques égratignures étaient visibles sur ses bras et ses mains qui tenaient à l'heure actuelle une tablette.
Sitôt réveillé et le voilà en train de se reconnecter au Réseau. Il posa sa tablette et tourna son regard vers moi.

Je tendis la main vers lui et me présentait « Lieutenant Tersant, Police Judiciaire. Je viens relever votre déposition.
» Ce faisant, j'activais mon implant pour que l'audition soit intégralement enregistrée. Une simple pensée suffisait à
le mettre en marche et à l'arrêter. Un léger picotement dans la main gauche me signala que l'enregistrement avait
débuté.

Il serra ma main. Une poignée de main un peu mollassonne et vaguement moite — exactement du genre de celles que je
n'aimais pas particulièrement — avec un léger signe de tête. Il prit la parole :

« Comment ça se passe ? Qu'est-ce que je dois faire ?

— Ça va aller tout seul. Je vais d'abord vous poser quelques question de routine puis vous me raconterez votre
histoire. Essayez d'être précis. Le moindre détail peut avoir son importance, même si cela vous semble futile ou sans
intérêt, dites le quand même.

— D'ac… D'accord… »

Sa voix était presque un murmure.

Je lui tendis mon téléphone après avoir lancé le logiciel de reconnaissance d'empreintes « Je dois d'abord vérifier
votre identité. Pouvez-vous poser le pouce sur ici s'il vous plaît ? Là, dans le rectangle. »

Le téléphone releva l'empreinte avec un \emph{bip bip} irritant — allez savoir pourquoi, si l'engin ne faisait pas de
bruit, les gens croyaient qu'il ne fonctionnait pas — puis vibra quelques secondes plus tard. L'écran affichait le
pédigrée de l'individu alité devant moi. Je le gratifiez d'un « Merci bien » puis repris :

« 12 avril, 11 h 32, début de l'audition. Plaignant identifié comme étant monsieur Francisco Diaz, 31 ans, musicien.
Victime probable d'enlèvement et séquestration. L'audition se déroule dans sa chambre à l'hôpital Purpan à Toulouse. Je
me dois de vous informer que cette déposition est intégralement enregistrée grâce à l'implant que je possède.»

Je fis un légère pause et m'installais sur la chaise posée à côté du lit. Je le regardais droit dans les yeux. Il
paraissait absent. Comme un voile dans ses yeux. Sûrement les calmants.

« Monsieur Diaz, vous engagez-vous à ne pas travestir la vérité et à n'omettre aucun détail utile à l'enquête qui
pourrait résulter de vos déclarations ?

— Oui, bien sûr. Sa voix tremblait légèrement.

— Nous pouvons commencer. Veuillez s'il vous plait me décrire les circonstances qui on ont amené à votre présence ici
même.

— Et bien… Comment dire, je ne me souviens plus très bien. J'ai des trous dans mes souvenirs. Je ne sais pas trop par
où commencer.

— Essayez de commencer par le début. Quel est le dernier souvenir clair que vous avez de la soirée où vous avez disparu.

— C'est… C'est assez vague. Je me souviens être allé au Calice pour retrouver des amis. Nous avons bu deux ou trois
bières. Peut-être plus. Je ne sais plus trop bien. Vers une heure du matin, je suis parti pour rentrer chez moi. Je me
souviens avoir pris le métro puis plus rien. Le black out jusqu'à ce que je me réveille ligoté.

— Vous souvenez-vous d'avoir croisé quelqu'un en particulier qui vous aurait paru suspect ?

— Non non. Pas au Calice. Le bar à ouvert il y a peu. La clientèle n'est pas encore très large. Je connais bien le
patron et nous y avons joué une ou deux fois avec mon groupe pour faire un peu de pub. Les gens qui viennent pour
l'instant sont des habitués. Quelques gens de passage, mais je n'ai pas lié connaissance avec eux.

— Et sur le chemin du retour ou dans le métro ?

— Rien de notable. Les noctambules habituels. Mais ça reste flou. Rien ne m'a choqué sur le chemin du retour. La
dernière chose dont je me souviens de cette nuit là c'est d'entrer dans le métro et de m'assoir dans la rame. »

Il avait l'air sincère et il donnait l'impression de vraiment se creuser la tête sur ce qu'il avait fait cette nuit là.

Je pris le temps de noter quelques mots. Il me fallait un accès aux enregistrements des caméras sur son chemin. Et
aussi celles du métro.

« Vous souvenez-vous quelles rues vous avez emprunté pour rejoindre le métro ?

— Non… Je pourrais vous l'indiquer sur un plan, mais j'ai toujours du mal à retenir le noms des rues. Et ça, ça ne date
pas d'hier, ajouta-t-il dans un petit rire.

— Donc, vous entrez dans la station, vous passez les portiques et quand la rame arrive, vous vous installez dedans puis
plus rien ?

— C'est ça. J'ai bien peur de ne pas vous être d'une grande aide…

— Ce n'est pas grave. Nous utiliserons les enregistrements pour vous localiser précisément. À quelle station
descendez-vous habituellement ?

— Aux argoulets. J'aime bien marcher un peu avant de rentrer chez moi. Ça dégrise un peu. »

Je notais \emph{Argoulets puis marche}. Ce n'était pas de chance. Comme on s'éloignait du centre les caméras se
faisaient plus rares. Cela allait être plus dur de le suivre après sa sortie du métro. À cet instant, mon téléphone
m'indiqua que je venais de recevoir un mail. C'était le docteur Courtois qui m'indiquait le lien pour accéder au dossier
médical de Diaz. Il avait été rapide.

« Parlez moi maintenant de votre réveil si vous le voulez bien. »