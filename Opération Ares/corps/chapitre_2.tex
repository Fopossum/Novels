\chapter*{2}
% Comment on en est arrivé à découvrir qu’un tueur en série rôde.
% Une victime accidentée de la route. Problème, la victime est nue. C'est un 
% homme. Il est choqué et gravement blessé. Il a marché pieds nus dans la
% campagne avant d'être percuté par une voiture.
% La conductrice de la voiture, choquée elle aussi, amène l'homme aux urgences 
% tout en ayant soin de prévenir la gendarmerie locale qui sera chargée de 
% l'enquête dans un premier temps avant que l'affaire n'aterrisse sur le bureau 
% du narrateur à Toulouse.
% La victime décèdera de ses blessures peu après avoir pu faire une déposition 
% partielle indiquant vaguement les circonstances de son enlèvement. Il se 
% souvient qu'il était dans un bar, seul. Puis trou noir jusqu'à son réveil 
% dans une pièce brillament éclairée, comme une salle d'opération. Il est 
% sous perfusion, sondé et nu, sanglé à une table. Cependant, une de ses sangles est 
% mal attachée et il parvient à se libérer. Il s'échappe alors d'une bâtisse 
% perdue dans les bois. Sa déposition est émaillée de descriptions lugubres du 
% bâtiment, impliquant des sculptures bizarres, des bustes tenant des livres, 
% une teinte rougeâtre omniprésente.
% Le rapport d'autopsie de la victime indiquera qu'il était en bonne santé et 
% que ce sont les blessures consécutives à l'accident qui l'ont tué, non pas ce 
% qu'il aurait pu subir.
% Une enquête de voisinage est menée mais elle n'aboutit à rien de particulier. 
% Les personnes vivant à proximité de l'endroit où la victime a été retrouvée 
% ont été interrogées et leurs dépositions prises. La tueuse a bien sûr été 
% entendue elle aussi mais n'a pas été soupçonnée. Elle est honorablement 
% connue dans la région et ne "connaissait" pas la victime.
% Première visite chez la tueuse odeur omniprésente de produits de tannage. Elle
% indiquera qu'elle a comme hobby la réfection de livres anciens et qu'elle tanne
% elle-même ses peaux. Elle montrera aux enquêteurs un des livres sur lequel elle
% travaille actuellement qui est un précis d'anatomie du XVIIIème dont elle refait
% la couverture.