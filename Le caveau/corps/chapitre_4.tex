\chapter[Chapitre Quatre]{Chapitre Quatre}
Un monceau de paperasse m'attendait sur mon bureau. Cette vue apocalyptique m'anéantit presque. Gordon, assis au bureau  
en face du mien me fit un grand sourire.

« Alors John ? La nuit a été mouvementée à ce qu'il parait.

— Mouais… On va dire qu'elle fut courte et bien remplie. »

Le rapport du coroner trônait sur la pile de papier. Un Post-It collé sur la couverture. Les coordonnées d'un
archéologue. Le professeur Smitherson. Un universitaire.

« Gordy ? Tu vas de suite à l'université et tu me ramènes ce professeur dans la Crypte. Je veux son avis d'expert au plus
vite sur cette histoire.

-- Ok chef ! »

J'allumai mon PC et m'attaquai à la pile de mails qui encombraient ma boîte de réception. Un en particulier retint mon
attention. Son titre était \emph{Des sectateurs amateurs de Hentai ?}. Il m'était envoyé par le responsable du dépôt des 
pièces à conviction et il me demandait de venir le rejoindre au plus vite.

L'ascenseur me déposa à l'étage des pièces à convictions et archives. Ici, dans les plus bas souterrains du commissariat se
cachaient les innombrables objets ayant un rapport plus ou moins lointain avec les affaires en cours. Un dédale de
cartons où seules arrivaient à s'orienter quelques personnes. Une grande grille barrait de toute manière le passage à
quiconque aurait eu l'idée saugrenue de se lancer dans une exploration.

Bob Gilford, chef du département approchant la quarantaine, leva les yeux du magazine de modélisme qu'il feuilletait en 
m'entendant arriver. Il enleva ses lunettes et se frotta les yeux avant de les tourner vers moi.

« Ah, John, te voilà. Dis donc, ce n'est pas très commun ce que tu m'as apporté là.

-- Comment ça ?

-- Oui, tout ce qui a rapport avec ton histoire de crypte. T'es sûr que t'es pas tombé sur une bande de joueur de jeux
de rôles qui auraient disjoncté ? Des amateurs de tentacules ?

-- Mais qu'est-ce que tu me racontes là ?

-- Bon, je te vois un peu perdu. Je vais te montrer. »

Il posa son magazine sur son bureau et vint m'ouvrir la grille.

« Suis moi. »

Le chemin entre les étagères débordant de boîtes archives me parut interminable. Il était ponctué de grandes taches de
lumière à intervalle régulier. Pendant qu'il marchait devant moi, Bob reprit :

« Tu as entendu parler de H.P. Lovecraft ?

-- Non. Qui est-ce ?

-- Qui était-ce plutôt. Un auteur du début du XXème siècle. Il a presque fondé un genre, un peu dans la lignée de Poe.
Ses histoires tournaient autour d'entités venues de l'espace qui étaient sur Terre depuis, je cite \emph{des éons}. Ses
nouvelles et romans ont toujours été emprunts d'une certaine lourdeur dans le style mais elle ont eu un grand succès.

-- Et quel rapport avec moi et mon affaire ?

-- Disons que là, -- Il me fit un clin d'œil -- je pourrais très bien t'appeler \emph{Inspecteur Legrasse} que ça ne
détonnerait pas.

-- Allez, accouche un peu. »

Bob aimait faire durer le suspense. C'était un excellent conteur et il mettait à profit ses talents pour une troupe de
théâtre locale. Autant habituellement j'aimais l'écouter nous raconter ses histoires, autant là, il commençait à me
courir un peu sur le système.

« Et bien, mon vieux, tu es en plein dans \emph{l'appel de Cthulhu}. »

Il s'arrêta soudainement devant une étagère et grimpa sur un escabeau pour aller attraper un carton en particulier. Une
fois redescendu, il se dirigea vers une des tables qui se trouvaient au bout de chaque travée pour y déposer son colis.

« C'est une partie des objets qui ont été retrouvés dans la crypte. Quand les agents les ont amenés, ça jasait pas mal
à propos d'une statue représentant un poulpe humanoïde. Et quand j'ai enregistré la pièce en question, j'ai été assez
surpris. »

Il extirpa un sac plastique du carton. Il contenait une statuette d'environ vingt-cinq à trente centimètres de haut. Un
corps humanoïde massif assis sur un trône de pierre. Ce qui faisait office de tête n'était qu'un amas de tentacules
partant en tout sens. De gigantesques ailes de dragon ornaient son dos. Elle avait un côté effrayant et fascinant à la
fois. J'avais du mal à détacher mon regard de cette abomination.

« \emph{Ph’nglui mglw’nafh Cthulhu R’lyeh wgah’nagl fhtagn}

-- Comment ? Qu'est-ce que tu as dis ?

-- Dans sa demeure de R'lyeh, le défunt Cthulhu attend en rêvant. Du moins ce sont les mots qui figurent dans les
ouvrages de Lovecraft. Bon, tu excuseras ma prononciation, mon accent n'a jamais été très bon pour prononcer les
incantations -- il partit d'un grand rire caverneux, du genre que pourrait avoir le grand méchant dans un vieux James 
Bond.

-- Je comprends mieux maintenant ta référence au Hentai maintenant que je vois cette\ldots{} chose\ldots{} Mais je ne
comprends pas pourquoi tu me parle de jeu de rôles.

-- Tout simplement car le \emph{Mythe de Cthulhu} a été adapté en jeu. Par contre, et je sais que tu as toujours eu du 
mal avec ça, je ne pense pas que cela soit un accessoire de jeu. Si tu regardes bien la statuette, tu ne trouveras 
aucune inscription façon "Made in China". Pas plus que d'indices que c'est un produit manufacturé. J'ai bien 
l'impression que c'est assez vieux comme objet. Et pas de la camelote. Je ne mettrais pas ma main à couper, je ne 
suis pas géologue, mais je parierais que c'est de la malachite. Ta statue semble avoir été taillée dans un bloc unique 
puisque je ne distingue pas de cassures. Mais ce n'est pas ma spécialité.

-- C'est à dire ? Tu es en train de me dire que nous avons retrouvé sur les lieux d'un probable crime une statue 
représentant une entité décrite par un obscur auteur du début du siècle dernier ?

-- Pas n'importe quel auteur John. On trouve son influence dans de nombreuses œuvres actuelles. Dans les films de Carpenter, 
dans Alien, chez Stephen King. Même chez Metallica chez qui on trouve « The call of Ktulu ». L'histoire raconte qu'ils 
ont simplifié l'écriture parce qu'ils pensaient que leurs fans n'arriveraient pas à prononcer « Cthulhu ».

-- Admettons. Mais quel est le rapport ? Cela voudrait dire qu'on doit se coltiner une secte qui vénère les écrits d'un 
auteur de livres d'horreur ?

-- Je n'en sais rien John. Tout ce que je peux te dire, c'est que cette statuette semble ancienne. Très ancienne. Datant 
peut-être même d'avant que Lovecraft n'écrive ses romans. Va savoir, on a peut-être mis la main sur une de ses 
inspirations pour ses écrits.

-- Tu ne serais pas un peu en train de te foutre de moi Bob ?

-- Si j'avais me foutre de toi, crois-moi, ce n'est pas ce sujet que j'aurais choisi. »

Je fis une pause pour observer de nouveau la statuette. Il se dégageait d'elle un je ne sais quoi de malsain. Bob avait 
raison, elle semblait ancienne. Elle était pâtinée et donnait l'impression d'avoir été polie par nombre de mains au 
cours des années. Ce n'était peut-être qu'une coïncidence, une simple coincidence. Rien de plus.

« Bon, tu peux me dire quoi d'autre dessus ?

-- Rien de bien intéressant sauf si soudainement tu voulais parfaire ta culture lovecraftienne. Je peux d'ailleurs te 
conseiller quelques bouquins qui\ldots{}

-- Laisse tomber les conseils de lecture pour le moment, on verra plus tard. Merci pour tout Bob. »

Sur ces dernières paroles je laissais Bob au milieu de ses cartons pour remonter à la surface. J'avais besoin d'un grand
bol d'air frais. En passant par mon bureau je déposai la statuette dans un tiroir et vit un mot de Gordy. Il était parti
chercher l'archéologue. Il allait, je l'espérais, nous être d'une grande aide.
