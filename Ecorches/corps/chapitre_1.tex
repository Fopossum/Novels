\chapter{Amnésie ?}

Alors que je présentais mon badge à l'œil électronique de l'ascenseur, mon téléphone vibra furieusement dans ma poche.
La cadence de la vibration indiquait qu'un message important venait d'arriver. La porte s'ouvrit au moment où
j'attrapais mon téléphone. En pénétrant dans la petite boîte exiguë, je ne pus retenir un frisson. Tout ce métal
étincelant et froid me donnait toujours l'impression que je rentrais dans une cellule sans vie, prête à m'engloutir à
jamais. 

Le message indiquait que ma présence était requise au plus vite. Une nouvelle affaire pour moi. Les étages défilaient 
et la porte s'ouvrit enfin sur la section criminelle. Le long couloir desservait les nombreux bureaux et les salles de 
réunion. Tout au bout se trouvait le vaste bureau du commissaire et je dirigeais mes pas vers cet endroit. Après avoir 
légèrement frappé à la porte de verre je rentrai dans la pièce sans attendre confirmation. La commissaire était assise à 
son bureau et elle regardait d'un air absent le fin écran devant elle.

Son regard se posa sur moi, légèrement au dessus de ses fine lunettes cerclées de métal. Bien qu'elle aurait pu se faire
corriger chirurgicalement son défaut de vue, elle continuait à porter des lunettes. Ce qui, je devais avouer, lui 
donnait un certain charme légèrement suranné. Son bureau était sobre et dépouillé de tout le superflu. Elle ressemblait 
à une vierge de glace dans son royaume. Froide et incisive. Éclats métalliques dans le regard.

« Ah, Tersant, vous voilà. Asseyez-vous.

— J'ai eu votre message. Qu'elle est donc cette affaire bizarre qui doit retenir mon attention ? Lui lançai-je en
m'asseyant.

— C'est assez hors du commun. Un homme a été retrouvé nu, errant le long de l'autoroute. Il présentait des blessures et
paraissait en état de choc d'après les retours des pompiers. Ils l'ont transporté vers Purpan.

— Bon, un fou à poil le long de l'autoroute. En quoi est-ce que la crime à quelque chose à voir dans cette histoire ? »

Le commissaire ôta ses lunettes et entreprit de les nettoyer. Même sans ses lunettes elle restait attirante et je ne pus
m'empêcher de jeter un coup d'œil à son décolleté mis en valeur par un tailleur sur mesure. Elle sembla s'en rendre
compte mais ne parut pas s'en formaliser outre mesure. En remettant ses lunettes, elle me dit d'une voix douce « Il
affirme qu'il a été enlevé, qu'il était ligoté à une table d'autopsie et qu'il a pu s'échapper par miracle. »

C'était bien ma veine, un illuminé. Un doux dingue.

« Et qu'en disent les psy ?

— Il paraît sain d'esprit et n'est pas délirant. À vous de jouer maintenant Tersant. Le dossier est accessible sur le
Réseau. Ça vous donnera l'occasion de tester votre implant en conditions réelles. »

Je me levais, la saluais brièvement et redescendais au plus vite au parking. Une seule phrase résonnait dans mon
esprit : « Mais quelle affaire de merde ! » Je savais que mon refus l'implant l'avait longtemps irrité mais là, me 
coller sur ça… C'est le genre d'affaire qu'on filait aux bleus. Pas trop d'enjeux. Pas trop de gaffes potentielles.

Le commissaire faisait partie des zélotes prêchant la surveillance à tout crin. Elle vouait une espèce d'idolâtrie sans
bornes à tout ce qui était électronique et ne jurait que par les \emph{grandes avancées} dans la lutte contre le crime
permises par toutes ces babioles. Ça me mettait hors de moi et cela nous avait valu elle et moi quelques frictions. Et
depuis, j'avais l'impression qu'elle me faisait payer ce qu'elle avait l'air de considérer comme de l'insubordination. 
Je ne comprenais pas trop comment elle fonctionnait avec moi. Froide comme l'acier d'abord, plus tard chaleureuse, 
presque amicale.

L'ouverture de la porte de l'ascenseur sur le parking mit fin à mes tergiversations et je m'engouffrais dans la
voiture. En programmant la destination j'accédais au dossier via le Réseau. L'affichage tête haute s'illumina et les
principales informations se dessinèrent en lettres brillantes sur le pare-brise. Je devais concéder à la technologie
que les voitures entièrement automatisée avaient quand même l'avantage de permettre de lire au volant sans danger.

Mon homme s'appelait Francisco Diaz, était d'origine espagnole, avait 31 ans. C'était un musicien qui cachetonnait à
droite et à gauche. Deux condamnations mineures sur son casier : possession de drogue en vue de sa consommation et
ébriété sur la voie publique. Plutôt beau gosse d'après sa photo. Le type même du fêtard qui abuse un peu et qui ne sait
plus le nom de la nana qu'il a levé le matin au réveil. Il devait avoir pris un truc bizarre et a préféré inventer cette
histoire plutôt que d'avouer qu'il était complètement défoncé et qu'il était incapable de se rappeler de sa fin de 
soirée. Sa disparition avait été signalée deux jours plus tôt par les membres de son groupe qui s'étonnaient de ne pas 
le voir à une répétition.

Lien vers son identité numérique… Et voilà… Comme je m'en doutais, encore un qui balançait tout de sa vie privée à tout
le monde. Je passais rapidement sur les photos, dont certaines étaient répréhensibles au regard des dernières lois sur
la protection des mineurs et remarquait que l'intégralité de son emploi du temps figurait en bonne place dans son petit
monde électronique. J'appris donc que la veille du signalement de sa disparition, il devait aller boire un coup avec
les membres de son groupe.

D'un geste rapide, j'accédais à ses relevés de carte bancaire et je découvris vite que des consommations avaient été
payé la veille du signalement de sa disparition dans un bar du centre.

Ouverture de la page du bar. Tiens, un bar spécialisé dans les bières belges. Intéressant. Il avait ouvert quelques
mois plus tôt. Je ne le connaissais pas encore. Il allait falloir que j'aille y faire un tour, pour demander au barman
si il se souvenait de notre homme.

Même si l'affaire me paraissait simplissime, j'aimais faire les choses bien et ne rien laisser au hasard. Je crois que
cette qualité — certains me considéraient trop tatillon — qui me permettait de rester flic malgré les rapports pas
forcément favorables rédigés par mes différents supérieurs au cours de ma carrière.

Ma voiture stoppa au poste de garde de l'hôpital et je déclinais mon identité et la raison de ma venue au planton qui
m'indiqua le bâtiment où attendait mon fêtard.

À peine arrivé au bureau d'accueil du service, un médecin s'approcha de moi et me tendis la main.
Il avait l'air parfaitement détendu et jovial. Un grand sourire se dessinait sur son visage. Je jetai un coup d'œil
rapide au badge ornant sa blouse : Docteur \textsc{Courtois}. À première vue, il portait bien son nom.

« Vous devez être le capitaine Tersant ?

— Oui en effet. Que pouvez me dire sur notre homme.

— Il va bien. Du moins aussi bien que l'on peut aller après quelques heures de manche nu dans la campagne. Il a
quelques égratignures et les pieds dans un état lamentable. Le plus étonnant et dérangeant est qu'il a été dépecé 
dans le dos. Un rectangle de peau d'environ 10 centimètres par vingt lui a été retiré. La découpe est nette et sans 
bavures. Vraisemblablement réalisée au scapel laser. La plaie a été ensuite cautérisée et traitée. Il est en état de 
choc et a un peu de mal à faire des phrases complètes.

— Vous voulez dire qu'il est… Je laissais ma voix s'éteindre en tamponnant ma tempe de mon index

— Non pas fou du tout. Uniquement choqué. Mais il s'en remettra vite avec un suivi correct. Et de vous rencontrer pour
que vous écoutiez son histoire lui permettra d'aller mieux. »

Je levais vers lui un regard interrogateur. Je voyais mal comment ma présence pouvait le faire aller mieux.

« Si vous voulez bien me suivre, je vous explique en chemin. »

Il se retourna et avança dans le couloir. Je restais à sa hauteur tout en notant mentalement la disposition des lieux.
Dans le cas où il faudrait un présence policière pour le protéger — ou le surveiller — je voulais avoir une petite idée
de l'emplacement des accès et de la topographie des lieux. Je sortais mon carnet de notes.

Le toubib reprit alors la parole et sa voix si fit un peu plus professorale :

« Tout d'abord, sachez que nos premiers examens psychiatrique ne montrent pas a priori de symptômes d'une maladie
mentale. Rien dans son dossier médical passé ne montre de signes précurseurs d'une pathologie mentale. Il était 
parfaitement saint d'esprit avant cette… Mésaventure pourrait-on dire.

— Donc vous me dites que c'est un type tout à fait normal ?

— Disons, un peu trop fêtard pour que son foie tienne longtemps la cadence, mais un type normal. Comme vous et 
moi, rajouta-t-il avec un clin d'œil malicieux.

— Donc, à part son amour pour les boissons alcoolisées, rien de notable.

— Non. Parfaite condition physique. Il s'entretient régulièrement, ne fume pas. Mais revenons en à ce qui motive votre
présence ici. Ce qu'il décrit l'a profondément choqué. Il souffre d'amnésie rétrograde, vraisemblablement passagère…

— Amnésie rétrograde ? C'est à dire ? Le coupai-je rapidement tout en notant \emph{Amnésie ??}.

— Il ne se souvient pas des heures précédent son réveil. Cela est dû aux drogues que nous avons décelé dans son
organisme. La douleur dans son dos n'a pas dû arranger les choses. Il a découvert qu'il avait été dépecé ici. 

— On l'aurait drogué avant de l'enlever ? Nouvelle note dans mon carnet.

— On ne peut pas l'affirmer avec certitude. Nous avons trouvé des traces de GHB.

— Le GHB ? Je croyais cette drogue dépassée depuis longtemps. »

Il s'arrêta devant une porte. Le vert clair de la peinture agressait mes yeux et je sentais revenir la douleur de
l'implant. Lancinante. Toujours là. Il allait falloir que j'avale un autre cachet pour la calmer. Et avant d'interroger
la victime.

« Le fait qu'il y ait du GHB dans son sang n'indique pas qu'il l'ait absorbé volontairement. Même si on en trouve très
difficilement de nos jours, on connait quelques précurseurs qui métabolisés par l'organisme sont dégradé en GHB et ont
donc l'effet voulu. Il est tout à fait possible qu'il l'ait absorbé à son insu »

Alors qu'il m'expliquait cela, je sortis de ma poche la plaquette de comprimés et en j'en avalai un rapidement.

« Vous êtes souffrant ?

— Rien de grave, un mal de tête qui passera vite. Mais continuez je vous en prie.

— Comme vous voulez. Où en étais-je ? Ah, oui. l'amnésie. Un des effets du GHB peut être l'amnésie. Nous ne pouvons
dire si celle-ci est causée par la drogue ou par le choc. Seul le temps pourra apporter des réponses à cette question.
Nous allons de toute manière le garder en observation quelques jours et nous vous tiendrons bien évidemment au courant
dès qu'il y aura du nouveau.

— Merci. Puis-je vous recontacter en cas de besoin lors de l'enquête ?

— Bien sûr. Et je vais vous faire parvenir une copie de son dossier dans les plus brefs délais. »

Il me serra à nouveau la main et s'éloigna, me laissant devant la porte qui me séparait de ma victime.

Je notais rapidement sur mon carnet les quelques questions que j'allais lui poser. Maintenant que j'avais appris qu'il
était sain d'esprit, qu'il avait peut-être été drogué et surtout qu'il y avait littéralement laissé la peau, cela 
changeait la donne. Je n'avais plus affaire à un bringueur qui avait perdu un pari idiot mais bel et bien à un 
rescapé d'une expérience traumatisante.

Je levais la main et frappais doucement à la porte.

Une voix assourdie me donna la permission d'entrer.

Une chambre d'hôpital, classique. Un seul lit. Mon client avait les moyens. Il paraissait jeune et en bonne santé.
Seules quelques égratignures étaient visibles sur ses bras et ses mains qui tenaient à l'heure actuelle une tablette.
Sitôt réveillé et le voilà en train de se reconnecter au Réseau. Il posa sa tablette et tourna son regard vers moi.

Je tendis la main vers lui et me présentait « Capitaine Tersant, Police Judiciaire. Je viens relever votre déposition.
» Ce faisant, j'activais mon implant pour que l'audition soit intégralement enregistrée. Une simple pensée suffisait à
le mettre en marche et à l'arrêter. Un léger picotement dans la main gauche me signala que l'enregistrement avait
débuté.

Il serra ma main. Une poignée de main un peu mollassonne et vaguement moite — exactement du genre de celles que je
n'aimais pas particulièrement — avec un léger signe de tête. Il prit la parole :

« Comment ça se passe ? Qu'est-ce que je dois faire ?

— Ça va aller tout seul. Je vais d'abord vous poser quelques question de routine puis vous me raconterez votre
histoire. Essayez d'être précis. Le moindre détail peut avoir son importance, même si cela vous semble futile ou sans
intérêt, dites le quand même.

— D'ac… D'accord… »

Sa voix était presque un murmure.

Je lui tendis mon téléphone après avoir lancé le logiciel de reconnaissance d'empreintes « Je dois d'abord vérifier
votre identité. Pouvez-vous poser le pouce sur ici s'il vous plaît ? Là, dans le rectangle. »

Le téléphone releva l'empreinte avec un \emph{bip bip} irritant — allez savoir pourquoi, si l'engin ne faisait pas de
bruit, les gens croyaient qu'il ne fonctionnait pas — puis vibra quelques secondes plus tard. L'écran affichait le
pédigrée de l'individu alité devant moi. Je le gratifiez d'un « Merci bien » puis repris :

« 12 avril, 11 h 32, début de l'audition. Plaignant identifié comme étant monsieur Francisco Diaz, 31 ans, musicien.
Victime probable d'enlèvement, séquestration et torture. L'audition se déroule dans sa chambre à l'hôpital Purpan à 
Toulouse. Je me dois de vous informer que cette déposition est intégralement enregistrée grâce à l'implant que je 
possède.»

Je fis un légère pause et m'installais sur la chaise posée à côté du lit. Je le regardais droit dans les yeux. Il
paraissait absent. Comme un voile dans ses yeux. Sûrement les calmants.

« Monsieur Diaz, vous engagez-vous à ne pas travestir la vérité et à n'omettre aucun détail utile à l'enquête qui
pourrait résulter de vos déclarations ?

— Oui, bien sûr. Sa voix tremblait légèrement.

— Nous pouvons commencer. Veuillez s'il vous plait me décrire les circonstances qui on ont amené à votre présence ici
même.

— Et bien… Comment dire, je ne me souviens plus très bien. J'ai des trous dans mes souvenirs. Je ne sais pas trop par
où commencer.

— Essayez de commencer par le début. Quel est le dernier souvenir clair que vous avez de la soirée où vous avez disparu.

— C'est… C'est assez vague. Je me souviens être allé au Calice pour retrouver des amis. Nous avons bu deux ou trois
bières. Peut-être plus. Je ne sais plus trop bien. Vers une heure du matin, je suis parti pour rentrer chez moi. Je me
souviens avoir pris le métro puis plus rien. Le black out jusqu'à ce que je me réveille ligoté.

— Vous souvenez-vous d'avoir croisé quelqu'un en particulier qui vous aurait paru suspect ?

— Non non. Pas au Calice. Le bar à ouvert il y a peu. La clientèle n'est pas encore très large. Je connais bien le
patron et nous y avons joué une ou deux fois avec mon groupe pour faire un peu de pub. Les gens qui viennent pour
l'instant sont des habitués. Quelques gens de passage, mais je n'ai pas lié connaissance avec eux. Je crois me souvenir 
d'avoir discuté avec une jeune femme, mais ça reste flou.

— D'accord. Et sur le chemin du retour ou dans le métro ?

— Rien de notable. Les noctambules habituels. Rien ne m'a choqué sur le chemin du retour. La dernière chose dont je me 
souviens de cette nuit là c'est d'entrer dans le métro et de m'assoir dans la rame. »

Il avait l'air sincère et il donnait l'impression de vraiment se creuser la tête sur ce qu'il avait fait cette nuit là.

Je pris le temps de noter quelques mots. Il me fallait un accès aux enregistrements des caméras sur son chemin. Et
aussi celles du métro.

« Vous souvenez-vous quelles rues vous avez emprunté pour rejoindre le métro ?

— Non… Je pourrais vous l'indiquer sur un plan, mais j'ai toujours du mal à retenir le noms des rues. Et ça, ça ne date
pas d'hier, ajouta-t-il dans un petit rire qui disparu instantanément dans une grimace de douleur.

— Excusez-moi, mon dos me fait mal et me démange atrocement. Les médecins m'ont expliqué que cela était dû au 
traitement de régénération de la peau qu'ils me faisaient suivre. Le même que pour les grands brûlés.

— Je comprends. Donc, vous entrez dans la station, vous passez les portiques et quand la rame arrive, vous vous 
installez dedans puis plus rien ?

— C'est ça. J'ai bien peur de ne pas vous être d'une grande aide…

— Ce n'est pas grave. Nous utiliserons les enregistrements pour vous localiser précisément. À quelle station
descendez-vous habituellement ?

— Aux argoulets. J'aime bien marcher un peu avant de rentrer chez moi. Ça dégrise un peu. »

Je notais \emph{Argoulets puis marche}. Ce n'était pas de chance. Comme on s'éloignait du centre les caméras se
faisaient plus rares. Cela allait être plus dur de le suivre après sa sortie du métro. À cet instant, mon téléphone
m'indiqua que je venais de recevoir un mail. C'était le docteur Courtois qui m'indiquait le lien pour accéder au dossier
médical de Diaz. Il avait été rapide.

« Parlez moi maintenant de votre réveil si vous le voulez bien. »

%\fancybreak{* * *}

Je sentis instantanément le malaise. Alors qu'il était déjà pâle, Diaz blêmit encore un peu plus.
La simple évocation de son réveil semblait le secouer encore plus.

« Allez-y doucement. Prenez le temps qu'il vous faut. Vous voulez un verre d'eau ?

— Non merci. Ça va aller. C'est juste que… J'ai peur que vous ne me croyez pas. En y réfléchissant bien, après ce que
je vais vous dire, vous allez me caser avec les doux dingues qui racontent s'être fait enlevés par des extraterrestres
pour des expériences bizarres. »

Je tiquais. Je n'avais pas franchement envisagé les choses sous cet angle. Instantanément me vint à l'esprit l'image de
petits gris tenant à la main une sonde se penchant sur Diaz. Je ne pus retenir un sourire et je tentais vainement de
chasser cette pensée en le rassurant :

« Non, pas du tout. Je suis déjà au courant des grandes lignes. Allez-y, je vous crois. Et le fait que votre dos soit 
amoché me fait dire que ce qui vous est arrivé n'est pas de l'affabulation.

— Je vois que ça vous fait sourire.

— Disons que d'imaginer un petit gris à vos côté entre deux carcasses de vache autopsiées est assez… Comique n'est-ce
pas ?

— Oui, vous avez raison. Il reprit quelques couleurs.

— Donc, vous vous réveillez et… »

Je laissais la phrase en suspend pour lui repasser la main. Il ne fallait pas que je donne l'impression d'orienter son
témoignage. Ce qu'il avait à m'apprendre devait être dit avec ses mots. Surtout pas les miens. Je fis mine de noter
quelque chose, mais plus pour me donner une contenance qu'autre chose. J'avais du mal à oublier le petit gris opérant 
avec un sourire sadique le dos de Diaz pour y prélever un grand rectangle de peau. Je n'espérais qu'une chose, c'est 
que ces images mentales ne se retrouveraient pas sur les enregistrements de l'implant. Parce que sinon, les collègues 
allaient se payer une bonne tranche de rigolade en visionnant la déposition.

« D'abord j'ai eu du mal à voir autour de moi. Une très forte lumière m'éblouissait. Vous savez, comme celle chez le
dentiste où dans les salles d'opération ?

— Oui, j'imagine assez bien, continuez lui dis-je en hochant la tête.

— J'avais froid et j'étais attaché. J'étais couché sur une surface métallique et j'avais des sangles aux chevilles et
aux poignets. J'ai d'abord cru que je faisais un cauchemar et je me suis dit que j'allais vite me réveiller. Mais comme
le temps passait, il a bien fallu que je me rende à l'évidence. J'étais à poil sanglé à une table. Après mes yeux se
sont habitués à la lumière, j'ai pu voir que j'étais dans une grande pièce. Blanche, immaculée. Vraiment comme une
salle d'opération. »

Il reprit son souffle et se servit un verre d'eau. J'en profitais pour noter rapidement \emph{Salle d'op' ? Médecin ?}
Cela paraissait un peu irréel. Le petit gris me regarda avec un sourire narquois.

« Une salle d'opération vous dites ?

— Oui. Des carreaux blancs partout. Et cette énorme lampe au dessus de moi. Il ne manquait que les instruments au
tableau.

— Et vous ne sentiez pas de douleur dans votre dos ?

— Non pas du tout. Les médecins pensent que j'étais encore anesthésié.

— D'accord. Qu'avez-vous fait ensuite ?

— J'ai commencé par crier. Fort et longtemps. Suffisamment pour avoir mal à la gorge. Et je crois bien m'être pissé
dessus de trouille.

— Personne n'est venu ?

— Non. Pas un bruit. Rien. J'ai attendu. Je ne sais pas combien de temps. Ça m'a paru infini. J'ai essayé de bouger mes
bras. Le droit était solidement attaché mais la sangle gauche était un peu plus lâche. En forçant j'ai réussi à
extraire ma main. À partir de là, j'ai pu me détacher l'autre main et les jambes. Je me suis redressé… — Sa voix baissa
d'un ton — Et c'est là que j'ai vu que j'étais allongé sur une table d'autopsie ou d'opération. Comme on en voit à la
télé dans les séries policières. Sauf que là, c'était réel et bien tangible. »

Je continuais à noter frénétiquement. J'imaginais assez bien l'horreur qu'il avait dû ressentir en se réveillant dans
cette ambiance. Tout à fait digne d'un scénario de film ou de série avec un tueur psychopathe. À la différence, comme
le disait Diaz, que là, c'était foutrement réel. Je frissonnais légèrement. On a beau être flic et endurci, ce genre de
récit garde un côté assez terrifiant. D'autant plus lorsqu'on se dit qu'il n'était pas encore conscient de toute 
l'horreur qu'il avait subi. 

« Continuez, je vous crois. Une image mentale très précise de ce que me décrivait Diaz s'esquissait.

— Je me suis levé et j'ai regardé partout dans la pièce. Et c'est là que j'ai vu les instruments. Ils n'étaient pas
dans mon champ de vision quand j'étais allongé. Ils reposaient sur une desserte dans un coin de la pièce. Des ciseaux,
des scapels, des pinces et une espèce de scie. J'ai attrapé un des scalpels et j'ai cherché si il y avait autre chose
qui pourrait me servir, mais rien de plus.

— Qu'avez-vous fait du scalpel ? Le rapport n'indique pas qu'il était en votre possession lorsque l'on vous a retrouvé.

— Je l'ai perdu dans la forêt durant ma fuite.

— D'accord. Poursuivez.

— Je me suis dirigé vers la porte. Elle était déverrouillée. Je l'ai ouverte lentement. Elle donnait sur un couloir
dont la lumière était éteinte. J'ai cherché un interrupteur mais je n'en ai pas trouvé. La lumière venant de la
salle d'opération suffisait à me permettre de voir un peu. J'ai suivi le couloir sur quelque mètres en passant deux
portes. Je me suis arrêté devant la première, et j'y ai collé mon oreille pour essayer d'entendre si des bruits venaient
de derrière. Lorsque j'ai essayé de l'ouvrir, elle était verrouillée. J'ai donc continué à marcher dans le couloir
jusqu'à un escalier. J'ai monté quelques marches, je dirais une quinzaine au plus et ma tête a heurté ce que je pensais
être le plafond. La lumière de la salle ne suffisait plus à bien m'éclairer et j'ai failli tomber en me cognant. Je me
souviens m'être retrouvé à genoux, me tenant la tête, les larmes aux bords des yeux.

— Est-ce que vous avez remarqué quelque chose de particulier dans le couloir ? Comme une décoration particulière ?

— Je vous avoue que non. Je n'ai pas fait attention à la décoration. Je voulais juste me tirer vite fait.

— Je comprends. \emph{D'autres portes. Cave étendue.}

— En fait, ce sur quoi je m'étais cogné la tête n'était pas le plafond mais une autre porte. Dans le plafond. J'arrivais
à distinguer de fins rais de lumière entre les deux battants. J'ai trouvé la poignée à tâtons et j'ai réussi à l'ouvrir.
Elle non plus n'était pas fermée à clef. Je montais les dernières marches pour me retrouver au milieu de nulle part. La
porte était en fait au niveau du sol et il n'y avait rien autour que la campagne. J'étais dans un champ. Loin de tout.
La seule lumière visible venait de la Lune et des étoiles.

— Avez-vous remarqué un autre bâtiment aux alentours ?

— Non, rien. J'ai distingué une forêt un peu plus loin et je suis parti dans cette direction. Aussi vite que je le
pouvais. Avant d'arriver à la lisière, j'ai enjambé une clôture. Et j'ai récolté pas mal d'égratignures sur les
barbelés. Une fois dans les bois, j'ai continué ma route en essayant de me repérer aux étoiles. C'est à ce moment que
j'ai regretté de ne pas m'être plus intéressé à l'astronomie plus jeune au lycée. Vous y connaissez quelque choses aux
étoiles vous ?

— Pas du tout, je crois que j'aurais été dans la même situation que vous — le petit gris ne souriait plus du tout — et
que je n'aurais pas fait mieux. C'est tout juste si j'arrive à trouver la Grande Ourse. Je lui adressait un regard
compréhensif.

— J'ai marché un certain temps qui m'a paru être une éternité et j'ai fini par m'adosser à un arbre. Mes pieds me
faisaient souffrir le martyr. J'avais dû me les accrocher sur les barbelés sans m'en rendre compte et la marche dans la
forêt n'arrangeait rien. J'ai pleuré. Je crois que je n'ai jamais autant chialé de ma vie. Et, épuisé, j'ai fini par
m'endormir. Pas très longtemps puisque quand je me suis réveillé il faisait nuit, mais suffisamment pour perdre mon
scalpel. Je grelottais de froid et c'est là que la douleur dans mon dos s'est réveillée. J'ai pensé que c'était à 
cause de l'arbre ou des griffures des barbelés lorsque je les ai enjambés et il m'a fallu longtemps pour réussir à me 
mettre debout et à continuer à marcher.
J'avais énormément de mal à faire un pas après l'autre. Chaque fois que je posais le pied par terre, de terribles
douleurs remontaient le long de mes jambes et mon dos me tiraillait. Mais je n'avais que la fuite comme option en 
espérant tomber sur quelqu'un qui pourrait m'aider. »

Il se servit un nouveau verre d'eau. Aussi dingue que pouvait être son histoire, il était crédible. Ses yeux étaient
embués de larmes. Il attrapa un mouchoir en papier pour s'éponger les yeux.

« Au bout d'un long moment, j'ai fini par apercevoir au loin les lumière de l'autoroute et je me suis dirigé vers elles.
Je n'ai jamais été aussi content d'être un piéton aux abords d'une autoroute que cette nuit. Très vite, un patrouilleur
est arrivé et m'a embarqué.

— Il a sûrement été prévenu par les autres conducteurs qui venaient de vous voir en passant.

— Sûrement oui. Et j'imagine assez bien la tête qu'ils ont dû faire en me voyant dans le plus simple appareil en train
de tendre le pouce. »

Il se mit à rire. Et moi aussi d'ailleurs. Je dois avouer que j'aurais été surpris par un type à poil faisant du stop.

« Avez-vous quelque chose à ajouter à votre déclaration ?

— Oui, une chose. Je me souviens assez bien d'une odeur chimique. Je suis incapable de vous dire quels produits
c'étaient, mais ça agressait le nez, ça me paraissait acide. Et ça venait clairement d'une des autres pièces qui
donnaient sur le couloir. Je crois que je me souviendrais toujours de cette odeur.

- Vous pourriez la reconnaître si vous la sentiez de nouveau ?

- Je pense que oui. C'était vraiment très particulier et je n'avais jamais senti ça avant.

- C'est noté. Autre chose ?

- Oui. Coffrez l'enfant de salaud qui m'a fait ça.

— C'est bien ce que je compte faire. Avez-vous un avocat ?

— Oui, celui qui s'occupe des contrats de mon groupe. Pourquoi ?

— Prenez dès aujourd'hui contact avec lui. Si nous attrapons ce type, vous en aurez besoin. »

Je me levais alors et sortis mon portefeuille pour attraper une carte de visite que je lui tendis :

« Si vous vous souvenez de quoique ce soit, même un détail, n'hésitez surtout pas à m'appeler.

— Merci capitaine.

— Fin de la déposition de monsieur Diaz. Capitaine Tersant. »

Une nouveau picotement dans la main m'indiqua que l'implant avait arrêté l'enregistrement. Je me dirigeais vers la
porte lorsque Diaz me lança :

« Capitaine ? Vous croyais que vous allez le chopper ?

— Je vais tout faire pour. »

J'ouvris la porte et sortis dans le couloir. J'avais du pain sur la planche. Beaucoup.