\chapter{Prologue}

Le neuro-implant me démangeait. Une démangeaison profonde. Avec un gros problème. Il est impossible de gratter quelque
chose situé à l'intérieur de soi.

\emph{« Chéri ! Nous avons eu le feu vert vert pour passer aux essais cliniques sur l'homme. Les précédents tests se 
sont avérés suffisamment encourageants et concluants pour que les autorités sanitaires acceptent que nous passions à 
l'étape finale »}

\emph{Je levais les yeux du rapport que j'étudiais. Elle rayonnait de bonheur. Elle avait une bouteille de champagne à 
la main et elle alla chercher deux flûtes dans la cuisine.}

\emph{« Nous devons fêter ça dignement ! »}

Les toubibs m'avaient prévenu. Pendant quelque jours après la pose de l'implant, j'allais ressentir de temps à autre
une « petite gêne ». Petite, mon œil ! C'était intolérable. L'impression qu'une armée de fourmis utilisait mes nerfs
comme des toboggans. 

\emph{Le commissaire me regardait droit dans les yeux. J'avais l'impression qu'elle fouillait mon âme à la recherche de 
réponses essayant d'extirper la vérité. Elle me disséquait littéralement, tel un légiste se penchant sur un cadavre pour 
en extraire les dernières images de la vie.}

\emph{« Tersant, je ne vous laisse plus le choix. Vous nous avez fait perdre trop de temps en tergiversations.}

\emph{— Madame, vous connaissez mes raisons. Vous savez parfaitement ce que je dois à cette ce type d'intervention.}

\emph{— Je sais. Et vous savez comme moi que votre femme était volontaire pour être implantée.}

\emph{— Vous comprenez donc ce que ça m'a coûté…}

\emph{— Oui, très bien. Et rappelez-vous que grâce à elle, ces problèmes n'arrivent plus. Mais c'est un ordre. »}

Je pris une des pilules magiques qui m'avaient été prescrites à la sortie de l'hôpital. Elles étaient censées faire 
disparaître la gêne en quelques minutes. C'est du moins le baratin que m'avait servi le pharmacien en me remettant en 
mains la boîte. « Attention, c'est puissant comme médicament. On recense quelques cas d'addiction ». Il avait rajouté 
ces derniers mots au moment où je sortais ma carte pour payer. « Merci Bien, bonne journée monsieur » avec un rictus 
désagréable.

\emph{Son agonie avait été infinie. D'abord les tremblements. Puis les pertes de mémoires. Je la regardais se faner et 
se racornir. Une fleur coupée qu'on aurait subitement cessé d'arroser. Le diagnostic qui tombe. Le rejet de l'implant 
et la nécrose gagnant les tissus environnants.}

Trois jours que je tournais en rond dans mon appartement. Incapable de faire quoi que ce soit. Je pensais sans cesse à 
ce maudit implant qui risquait de me détruire le cerveau. Et toujours cette démangeaison que les pilules calmaient. Sur 
ce point là, on ne m'avait pas menti. Il était temps de retourner au charbon.

\emph{L'ultime décision. Intolérable. Les condoléances des proches. Le sentiment de vide après. Ce vide impossible à 
remplir, à oublier. Blessure jamais complètement cicatrisée. Plaie profonde toujours prête à se rouvrir. Reprendre le 
boulot. Se jeter à corps perdu dedans pour tenter de gommer son absence. Le déménagement. Faire ses cartons. Essayer de 
commencer une nouvelle vie.}

Je rentrais dans ma voiture et programmais la direction du commissariat central place du Capitole. Je ne croisais 
aucune autre voiture dans les rues animées et bondées de chalands qui s'écartaient de mon chemin. Depuis que tout les 
véhicules à moteur avaient été bannis de la ville pour favoriser les transports en commun moins polluants, seuls les 
services d'état disposaient de véhicules individuels. Un des rares privilèges auquel ma fonction donnait droit.

Ma voiture s'engouffra dans le parking souterrain maintenant réservé aux véhicules officiels. Elle alla se garer sur sa 
place attitrée et la voix suave du GPS débita son « Vous êtes arrivé à destination ». Je détestais cette voix. Je ne 
comprendrai jamais pourquoi il fallait obligatoirement que ma voiture me parle sur un ton sensuel, à la limite de 
l'obscène. La portière s'ouvrit et je me dirigeais alors vers l'ascenseur d'un pas rapide.