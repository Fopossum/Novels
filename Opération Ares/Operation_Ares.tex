\documentclass[ebook]{memoir}

% Inclusion des headers
\input headers.tex

\title{Opération ARES}
\author{L'auteur}

\begin{document}
\maketitle

\begin{verse}
Ph’nglui mglw’nafh Cthulhu R’lyeh wgah’nagl fhtagn
\end{verse}

% Pas d’assaut du GIGN mais négociation avec le narrateur.
% Le narrateur rentrera dans le bâtiment, accompagné par un autre inspecteur.
% Toute la scène sera filmée par des micro caméras situées dans l'équipement.
% La victime ne survivra pas longtemps à ses blessures après sont accident.
% L’information fuite très vite
% le tueur attend bien sagement les forces de l’ordre
% Les victimes sont embaumées et conservées dans des poses grotesques, tenant 
% des livres. Technique de Fragonard employée. Donc => écorchés.
% Le meurtrier sera capturé
% Le meurtrier est une meurtrière !
% Les psy disent qu’elle connait la différence entre le bien et le mal
% mais que cette différence n’est pas la même que pour tout le monde
% Découverte chez le tueur de bouquins, reliés en peau humaine qui contiennent
% les délires du tueur. Un livre par victime, relié avec la peau de la victime.
% Procès à huis clos à cause de l’horreur.
% Accusée condamnée ?
% Toutes les pièces n’ont pas été présentées au procès.
% Le narrateur attend ses collègues après avoir posté son récit et fin de 
% l'histoire.
% Le narrateur est en instance de divorce. Bien qu'il ne vive plus avec sa
% future ex-femme il garde quand même quelques relations avec elle.
% Son ex-femme va disparaître, sans laisser de traces. Enlevée par la tueuse.
% Cette dernière provoquera une rencontre avec le narrateur et le laissera
% tomber amoureux d'elle. Rencontre "fortuite" dans la rue, boire un café, etc.
% Cela lui permettra de suivre l'enquête et de manipuler le narrateur.
% Pour mettre la pression, elle fera comme pour la dernière victime et enverra
% aux forces de l'ordre les papiers d'identité de l'ex-femme du narrateur.
% Le courrier est tapé à la machine, pas d'empreintes, posté depuis le centre
% ville de Toulouse.
% 

\input corps/chapitre_1.tex
% Introduction. On pose l'ambiance.

\input corps/chapitre_2.tex
% Comment on en est arrivé à découvrir qu’un tueur en série rôde.

\input corps/chapitre_3.tex
% Le procès et l’impression de spoliation de la part du narrateur.

\input corps/chapitre_4.tex
% À l’intérieur de la maison.

\input corps/chapitre_5.tex
% Ce qui a été caché.

\input corps/chapitre_6.tex
% Maintenant, je suis soulagé d’avoir tout raconté.

\end{document}