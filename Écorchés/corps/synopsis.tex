L'intrigue se déroule dans un futur indéterminé où l'hygiénisme et la pensée
unique font loi.

Les gens sont de plus en plus tracés, surveillés, fichés.

La criminalité violente a beaucoup baissé et les meurtres sont devenus de plus
en plus rares au fil des années. On recense surtout de la criminalité en col
blanc et rares sont les crimes sordides encore perpétrés.

Le héros, flic, porte un neuro-implant, activé automatiquement à l'entrée dans
une salle d'interrogatoire. Il est normalement désactivé le reste du temps mais
peut l'être à la demande par son porteur.

Il s'active aussi automatiquement lorsque son porteur rencontre quelqu'un lié à
une enquête en cours car il détecte la présence de la puce RFID de la carte
d'identité des suspects.

Il peut aussi être activé à distance par la hiérarchie ou les services internes
sans que le porteur ne puisse l'empêcher ou même en être au courant. (Possibles
dérives ?)

Notre héros va d'abord se retrouver avec une affaire déconcertante dès le
début. Un homme est retrouvé nu, errant aux abords de la ville. Il n'a aucun
papier mais ses empreintes permettent de l'identifier. Pas de casier, quelques
infractions mineures comme ivresse sur la voie publique automatiquement
sanctionnées grâce à la reconnaissance faciale des centaines de caméras
automatisées disposée un peu partout en ville.

Il a été drogué et ne se souvient de rien (GHB ?). Il ne se souvient que de la
pièce où il s'est réveillé, sanglé à ce qui pourrait ressembler à une table
d'opération ou d'autopsie. Une des sangles de ses bras était mal fixée et cela
lui a permis de s'échapper en vitesse. Il se rappelle d'un couloir sombre,
d'une espèce de vieille ferme et qu'il a longtemps erré dans la campagne sans
croiser âme qui vive avant d'arriver à la limite de la ville où il fut trouvé
par une vieille dame qui promenait son chien.

La dernière chose qu'il se rappelle avant son réveil c'est d'être allé boire un
verre avec des amis à lui dans un bar et qu'il est rentré seul après. Son
emploi du temps est confirmé par ses relevés de comptes bancaires et par sa
carte d'abonnement au métro.

Le barman confirmera que c'est un client régulier de son établissement et qu'il
a pour habitude de vite lier connaissance avec les gens. Il repart toujours
seul ou avec ses amis.

Il a en fait été remarqué dans le bar par la tueuse précédemment et avait
obtenu de lui son numéro. Elle a décidé de frapper cette nuit là, bien
longtemps après sa première rencontre avec lui.

La victime habite dans une zone résidentielle en périphérie de la ville, peu
couverte par les caméras de sécurité et les enregistrements disponibles pour la
nuit de l'enlèvement ne montrent rien de particulier. Toutes les voitures sur
les bandes appartiennent à des gens du quartier et les seules personnes à pied
visibles ont toutes été identifiées et blanchies. À l'exception de deux dont
les visages n'ont pas été captés par les caméras.

La victime avait pour habitude de publier ses moindres faits et gestes sur le
réseau visible par tout le monde. N'importe qui pouvait donc savoir où elle se
rendait pour boire avec ses amis.

La tueuse, très intelligente, dispose de tout un arsenal de cartes fausses mais
parfaitement valides qui lui permettent d'utiliser les transports en commun
dans la vile sans être tracée. Elle a accès aux plans complets des réseaux de
caméras de par sa position haut placé dans la hiérarchie policière. Elle est de
fait la supérieure du héros.

Elle peut donc suivre tous les faits et gestes du héros en déclenchant à
distance le neuro-implant de ce dernier. Elle sera donc au courant en direct de
l'avancée de l'enquête. De plus, ses connaissances du système lui permettent
d'effacer ses traces malgré le Réseau.

Le Réseau est le successeur d'Internet tel que nous le connaissons. Hautement
contrôlé par les états, tout ce qui s'y passe est systématiquement analysé
et stocké pour servir de preuve le cas échéant. Transactions bancaires,
habitudes de surf des utilisateurs, tout est tracé et soigneusement interprété
pour dresser le profil des gens.

Il existe un réseau alternatif, monde des hackers et autres défenseurs de la
Liberté. Entièrement chiffré et décentralisé, il est aussi bien utilisé
par des hacktivistes que par des « terroristes ». C'est par là, et par des
backdoors soigneusement dissimulées dans les applications gouvernementales que
la tueuse passe pour effacer ses traces.

Elle profite de son poste au sein des forces de police pour se renseigner sur
ses futures victimes dans le cadre d'enquêtes de routines destinées à traquer
la moindre déviance à la Loi du Grand Conseil.

Alors que l'enquête va commencer à piétiner, elle fera livrer par porteur le
portefeuille de la victime. Commençant ainsi son jeu du chat et de la souris.

Là encore, impasse. Pas de traces exploitables dans ce qui s'y trouve. Toutes
les puces de sécurité ont été soigneusement désactivées.

L'agence du livreur a été contactée via le réseau, le paiement effectué en ligne
et le colis a été déposé dans une agence de campagne. Les caméras révèlent que
la personne qui a déposé le colis ressemble à un homme de taille moyenne, aux
cheveux longs. Il porte une casquette et son visage n'est jamais visible. Les
détecteur de puces de l'agence l'ont identifié comme une personne vivant à Brest
et l'enquête réalisée par les collègues sur place montrent qu'il ne pouvait pas
être au même moment à Brest et dans la région toulousaine.

Elle fera parvenir petit à petit d'autres portefeuilles, révélant ainsi que
certaines disparitions pourraient être liées à cette affaire. À chaque fois,
les portefeuilles arriveront par livreur, paiement en ligne et description
vague de celui qui poste. À chaque fois, l'identité de la personne qui poste ne
correspondra pas et sera celle d'une autre vivant dans une autre ville à des
centaines de kilomètres et ne pouvant physiquement se trouver dans les agences
au même moment.

Le point commun entre les identités et que toutes ces personnes on eut au moins
une fois affaire avec la Justice et sont fichées dans le Fichier central le la
police.

Le troisième portefeuille reçu sera celui du père de la tueuse. Ceci lui
permettra de se rapprocher de l'enquête de manière légale et lui permettra de
prendre contact avec le héros. Là commencera le jeu de la séduction pour en
faire une de ses prochaines victimes.

Au fur et à mesure que l'enquête stagne, la relation entre la tueuse et le
héros se renforce. Elle commence donc à lui révéler d'où elle vient, que sa
famille a fait fortune au siècle dernier et qu'elle dispose d'une maison de
campagne dans la région où elle aimerait qu'ils aillent pour passer un weekend
et oublier un peu le boulot.

Le weekend se passe normalement, la seule chose un peu choquante étant un
écorché en buste dans la grande bibliothèque. Elle révèlera que la maison de
campagne était en fait la maison de sa famille mais que son travail l'a obligé
à venir habiter en ville mais qu'elle y retourne assez souvent pour se
ressourcer. L'écorché est dans la famille depuis des générations. On raconte
que ce serait un Fragonnard et elle l'a toujours connu. C'est en fait le buste
de son propre père dont elle s'est elle-même occupée.

Les weekends à la campagne s'enchaînent et le héros goûte de plus en plus à
cette liberté loin de la ville, profitant de ces moments pour se plonger dans
les livres de la bibliothèque.

Au bout de deux mois environ, il se rendra compte d'un changement sur le buste
exposé. Il semble féminin désormais. C'est à ce moment là que sa vision se
troublera et qu'il s'évanouira.

À son réveil, il sera sanglé à la table, et sa maîtresse lui racontera quel
plaisir elle a pris à jouer avec lui. Elle lui révèlera que le premier buste
qu'il a vu était celui de son père. Qu'elle est un peu triste de faire ce
qu'elle s'apprête à faire mais qu'elle le doit. Elle s'est attaché à lui et
qu'elle ne veut pas qu'il disparaisse comme tous les autres. Et que grâce à ce
qu'elle va faire, il sera avec elle pour toujours. En même temps qu'elle lui
dit tout ça, il se rend compte qu'il n'a plus ses jambes. « Tu comprends mon
chéri, le précédent s'était échappé. Je ne voulais pas que ça arrive avec toi. »

« Tu vois, toute cette technologie que tu as a ta disposition ne t'a jamais
permis de me soupçonner, alors que tu couchais avec ta proie. Mais était-ce
bien moi la proie ? »

Fin du livre sur un black out du héros dont la dernière vision sera celle des
bustes des précédentes victimes qui « l'observent ».