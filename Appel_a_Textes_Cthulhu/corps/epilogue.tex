\chapter[Épilogue]{Épilogue}
Le réveil sonna. Le bip-bip strident me vrilla les tympans et me força à me réveiller. Encore une nuit peuplée de ces 
mêmes cauchemars qui me hantaient depuis maintenant deux semaines. Depuis cette nuit d'Halloween où mon univers 
avait basculé dans l'irréel et la folie. Mais ça irait mieux maintenant. Penser à désactiver le réveil. Je suis en 
\emph{vacances} !

Fini la chasse aux sectes et aux sorciers. Je laissais tout cela à mes successeurs et à Gordy. Ils sauraient 
parfaitement comment gérer les choses. J'en avais enfin fini avec tout ça et je comptais bien enfin profiter de mon 
temps. La retraite. Enfin. Des années passées au service de la communauté qui avaient eu raison de mon couple, de ma 
famille et enfin de ma santé. Je me redressai pour me lever. La douleur fut brève et fulgurante. Encore une alerte 
de mon vieux coeur. Je devais absolument essayer de me ménager un peu. 

J'allais enfin pouvoir profiter de mes petits enfants et m'occuper du jardin. L'idée que je ne devais pas aujourd'hui 
aller au bureau remuer les histoires sombres de tous ces inconnus me fit sourire. Mais avant tout cela, il fallait 
que je pense à appeler mon cardiologue. Et que je range les cartons contenant toutes les affaires que j'avais rammené du 
bureau hier. Toute une vie d'enquêtes qui tenait dans quelques boîtes en carton.

Je me levai avec difficultés, enfilai les pantoufles en forme de pattes de chat que ma petite fille de quatre ans avait 
choisi pour mon dernier anniversaire et me dirigeai d'un pas résolu et pelucheux vers la cuisine pour aller me préparer
un café. Alors que je passais la porte de ma chambre, j'évitais de justesse le chat qui se dirigeait ventre à terre vers 
ma chambre. Depuis des années, il était mon seul compagnon et profitait toujours de mon lever pour que je lui laisse 
la place chaude sous la couette.

Je l'entendis miauler. Ce n'était pas dans ses habitudes et l'appelai pour qu'il me rejoigne. Nouveau miaulement, 
différent des habituels. Bah, ça lui passerait avant peu. Je me dirigeais vers la cafetière lorsque qu'il vint sauter 
sur le plan de travail et s'assoir devant moi tout en continuant à miauler. Il me regardait droit dans les yeux.

« Mais qu'est-ce que tu as toi ? »

Je ne m'attendais pas à une réponse. Mais un long miaulement rauque, venu du fond de sa gorge accueillit ma question. Il 
se passait décidément quelque chose. Il pencha la tête sur le côté et miaula de nouveau. L'espace d'un instant, je cru 
percevoir une question. J'étais persuadé que si il savait parler, il serait en train de me demander si j'allais bien.

« Mais oui, tout va bien. Ne t'en fais pas. »

Nouveau miaulement et il détala vers la chambre. Voulait-il me montrer quelque chose ? À tout les coups, il m'avait 
encore ramené une souris et voulait me montrer à quel point il m'aimait.

En passant la porte, mon souffle se coupa. Je manquais d'air et il me fallut quelques secondes pour reprendre mon 
souffle. Le chat était sur le lit et surmontait une forme massive.

« Merde… Qu'est-ce que… »

La forme sur le lit était un homme. Allongé. Il me tournait le dos. Le chat se tourna vers moi et miaula de nouveau. Il 
me regardait avec insistance m'invitant à venir vers lui. Je m'approchai de l'homme. Il paraissait endormi profondément. 
Les questions se bousculaient dans ma tête. Rien de tout cela n'était normal.

Comment cela était-il possible ?

Comment un homme pouvait-il être en train de dormir dans \emph{mon} lit.

Par \emph{où} était-il rentré sans que je ne l'entende ?

Comment avait-il fait pour passer inaperçu ? 

Le chat sauta du lit emportant dans son mouvement le drap qui recouvrait 
l'homme et s'assit par terre emportant.

« Nom de Dieu ! »

L'homme portait mon pyjama.

% Décrire le choc et l'envol vers l'espace. La vision apocalyptique du chemin 
% de brouillard venant des étoiles vers la Terre et des choses qui l'arpentent.
% Compréhension des rêves. De cette impression d'avoir été choisi.
% Tristesse pour le monde qui va être détruit.
% Pensée pour le chat, ses enfants et ses petits enfants qui vont atrocement
% souffrir à partir de maintenant.

\fancybreak{$* * *$}

La silhouette encapuchonnée s'approcha du large trône en pierre où son maître attendait.

« Alors ?

— L'offrande a été faite.

— Parfait. Tout s'est déroulé comme prévu. Les étoiles sont alignées.

— Et maintenant Maître ?

— Son règne va débuter et Il saura nous récompenser. Vous avez bien servi. Votre dévotion à notre cause a été 
exemplaire et vous ferez partie des légendes.

— Merci Maître. »

Un sourire se dessina sous la capuche, dans l'aura verte des torches qui brûlaient le long des murs.

\fancybreak{$* * *$}

L'annonce dans la rubrique nécrologique du \emph{Post} tenait en quelques lignes.

\emph{L'inspecteur John Hefat est décédé d'une crise cardiaque à l'âge de 62 ans après 40 ans de bons et loyaux 
services dans les forces de l'ordre. La cérémonie se déroulera dans la plus stricte intimité le dimanche 17 novembre 
2013 en la Basilique Saint Georges. Les pensées de sa famille et de ses collègues l'accompagnent.}

\begin{verse}
\emph{Ph’nglui mglw’nafh Cthulhu R’lyeh wgah’nagl fhtagn \\
Iä Iä Cthulhu fhtagn}
\end{verse}