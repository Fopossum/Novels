\documentclass[12pt,a4paper,oneside]{memoir}

% Les headers
\usepackage{fontspec}
\usepackage{xunicode}
\usepackage{xltxtra}
\usepackage{polyglossia}
\setmainlanguage{french}
\usepackage[unicode=true,
            %pdfauthor={Fred P.},
            pdftitle={Le caveau.},
            pdfsubject={Nouvelle dans le style de H.P. Lovecraft},
            pdfkeywords={Horreur, Cthulhu, Policier},
            pdfcreator={XeLaTeX},
            hidelinks]{hyperref}
% Des marges de dingues
\usepackage[top=3.5cm, bottom=3.5cm, left=3.5cm, right=3.5cm]{geometry}
\pagestyle{plain}
\setromanfont[Mapping=tex-text]{Times New Roman}
\setmonofont[Mapping=tex-text]{Courier New}
%Redefinition de la taille des interlignes. Berk.
\renewcommand{\baselinestretch}{1.5}

\title{Le caveau}
% Pas de noms dans le fichier d'après les règles du concours donc on commente.
%\author{Fred Passerin.}
% Je ne veux pas que \maketitle me colle la date, donc c'est vide.
\date{}

\begin{document}
\maketitle

% Ici, c'est le copyright / Copyleft.
\vspace*{\stretch{1}}

%Commenté pour que le document final ne contienne pas de noms.
%Copyright \copyright{} 2013 Fred Passerin.

\vspace{\baselineskip}
Copyleft: cette oeuvre est libre, vous pouvez la copier, la diffuser et la modifier selon les termes de la Licence Art 
Libre (\url{http://artlibre.org}) version 1.3 ou ultérieure.
\thispagestyle{empty}

\mainmatter
C'était la nuit. Il y eut un bruit sec. Puis un cri. Suivi d'un autre, et encore d'un autre… Comme un chant de haine. 
Ils cessèrent d'un coup. Le silence s'installa à nouveau dans la pénombre…

L'inspecteur Hefat s'approcha du lampadaire à gaz situé au coin de la rue et observa sa montre. 2 h 17. Même si il 
n'était plus de service depuis longtemps, un flic restait un flic. Il ne pouvait ignorer cet appel. Les cris 
provenaient de la direction de l'ancien cimetière et il accéléra le pas vers la vieille grille rouillée qui se 
découpait 
dans les hauts murs d'enceinte. L'ombre de la grille qui se déployait devant lui, léchant les pierres tombales à 
l'abandon avait quelque chose de dérangeant. Les nuages fins qui courraient devant la Lune déjà haute dans le ciel 
voilaient partiellement sa lumière et donnaient un côté fantomatique à la scène.

« Ressaisis-toi un peu John » marmonna-t-il en posant la main sur la grille. Le cadenas et la chaîne qui d'habitude 
barraient l'entrée du lieu aux passant gisaient sur le sol. Dans la lueur blafarde, l'inspecteur remarqua que la 
coupure sur les maillons de la chaîne était nette. On avait utilisé une pince.

Il sorti son Colt de son holster et avança prudemment entre les tombes noyées dans les herbes folles. Çà et là, des ifs 
centenaires déployaient leur longues branches qui dessinaient des formes absurdes dans la lumière ambiante. Une lumière 
verte diffusait depuis le fond du cimetière et nimbait l'atmosphère d'une aura glauque. Au fur et à mesure qu'il se 
rapprochait, il ralentit son rythme et utilisa au mieux les tombes pour se dissimuler.

La lueur poisseuse venait d'un caveau tout proche dont les portes grandes ouvertes grinçaient dans la brise légère qui 
venait de se lever. Une odeur infecte agressa alors les narines de l'inspecteur. Une odeur telle qu'il n'en avait 
jamais 
connu. Son cœur se mit à battre plus fort dans sa poitrine et sa prise se resserra sur la crosse nacrée de son Colt 
dont 
il libéra le cran du sureté. Le contact du métal froid le rassura quelque peu.

Il faillit défaillir quand il entra dans le caveau tellement l'odeur était insoutenable. Il lui sembla reconnaitre du 
souffre. La puanteur était telle qu'elle semblait littéralement coller à sa veste et à son gilet, imprégnant la 
moindre fibre en profondeur, se mêlant à la sueur et dégoulinant entre ses omoplates. L'air lourd et épais charriait 
l'odeur de la mort et du désespoir.

«Il fallait que ça me tombe dessus, à moi, à deux semaines de la retraite. Qu'est-ce que je fais ici…» Une pulsation 
malsaine emplit ses oreilles. Une psalmodie impie. Un chant dans une langue inconnue et trainante. Des mots 
incompréhensibles qui lui glacèrent le sang, s'insinuant jusque dans ses os. Un langue qui ne devrait pas être. Qui 
portait dans ses vers la promesse d'abominations sans nom, de destruction aveugle et de damnation éternelle. Des mots 
venus d'un autre âge.

L'étrange mélopée se répétait sans cesse, variant d'intensité mais toujours terrifiante. La teinte verdâtre de la 
lumière conférait aux ors et aux cuivres ornant l'intérieur du caveau un aspect malsain. Un escalier aux marches 
taillées dans le marbre s'ouvrait en face de lui et plongeait vers les profondeurs de la Terre. Il descendait 
prudemment 
la volée de marches et déboucha dans un couloir dallé. Des torches placées sur les murs brûlaient, noyant 
l'environnement de cette violente teinte verte. L'odeur de mort encore plus oppressante qu'à la surface présageait du 
pire.

L'inspecteur se retourna et jeta un œil vers le haut de l'escalier. S'il devait fuir, ce serait la seule voie possible. 
Il n'aimait pas se sentir dépourvu de solutions de repli et cela le mit encore plus mal à l'aise. Il décida 
néanmoins de continuer vers le fond du couloir. Il fallait qu'il découvre ce qui se tramait ici, sous le cimetière. Les 
chants étaient maintenant plus distincts. Il semblaient monter de la pièce située au bout du couloir dont l'accès était 
barré par une lourde porte en bois massif richement ouvragée de fines gravures réalisées dans un métal inconnu qui 
renvoyait des éclats violacés et dorés malgré la lumière verte baignant le lieu.

Prudemment, Hefat posa sa main sur la poignée et tenta d'entrouvrir la porte. Elle refusa de bouger d'un pouce et resta 
fermée. Il colla alors son oreille au bois tiède et put distinguer les mots des cantiques. Des mots qu'il transcrirait 
plus tard dans son rapport.

\begin{verse}
\emph{Ph’nglui mglw’nafh Cthulhu R’lyeh wgah’nagl fhtagn \\
Iä, Iä, Cthulhu fhtagn}
\end{verse}

Une décharge électrique parcourut son échine. Il fut prit de soudaines sueurs froides, peinant à respirer. Son sang 
lui sembla s'épaissir dans ses veines, son estomac fut pris de soubresauts violents et il s'en fallut de peu pour qu'il 
ne perde le contrôle de sa vessie. Une terreur primale déferlait le long de ses nerfs à la vitesse de la lumière et 
l'immobilisait sur place. Il ne pouvait pas se détacher de cette porte qui cachait, il en état sûr, des pratiques 
impies 
auxquelles seuls des fous pouvaient s'adonner. Avec de grandes difficultés il décolla son oreille du bois et observa 
les 
enluminures décorant la porte.

Des phrases dans une langue inconnue, peut-être celles des cantiques, accompagnait des dessins de créatures abominables 
indescriptibles. Elle semblaient sortir des délires éthyliques d'un ivrogne en manque d'alcool bon marché. Soudain, les 
chants s'interrompirent et une odeur méphitique s'insinua à travers la porte. Des senteurs inconnues de l'inspecteur. 
Quelque chose qui ne venait sûrement pas de ce monde. La puanteur était telle que l'inspecteur se demanda comment il 
pouvait encore respirer.

Un grognement sourd, presque palpable suivi d'un cri perçant déchira l'air. Un autre cri plus rauque répondit au 
premier. Le grognement continuait et diffusait ses basses épouvantables tout autour de Hefat. Les chants 
recommencèrent, plus rapides, accélérant. Les battements de son cœur à l'unisson s'emballèrent subitement. Une douleur 
aiguë traversa sa poitrine et son bras gauche. Il chercha frénétiquement sa petite boîte de pilules et en avala deux. 
Si 
il voulait survivre, il lui fallait du renfort.

Faire demi-tour et remonter l'escalier s'avéra être une épreuve quasi insurmontable. Il lutta contre l'engourdissement 
de ses membres et la douleur et alla s'effondrer à une dizaine de mètres du caveau maudit, adossé y une tombe en piteux 
état. La douleur diminua mais resta là pour lui rappeler qu'il avait été le témoin de quelque chose qu'il n'aurait pas 
dû voir. Mais il n'avait pas le temps de s'apitoyer sur son sort. Le cri qu'il avait entendu provenait, il en était 
certain, de la même personne qui avait poussé ceux le poussant à se détourner du chemin de son logement.

Lentement, difficilement il fit le chemin inverse et retourna dans la rue où à quelques mètres de lui se trouvait une 
cabine téléphonique réservée aux forces de police. Il allait pouvoir donner l'alerte.

\fancybreak{$* * *$}

Vingt minutes plus tard, une dizaine d'hommes lourdement armés déboulèrent dans la rue. Hefat prit la parole :

« Messieurs, je tiens à ce que tout se passe dans le calme. Il est possible qu'une victime innocente soit retenue par 
des fous et nous devons tout faire pour qu'elle ressorte vivante. Je pense que nous avons affaire à des sectateurs d'un 
culte satanique et je les veux vivants pour pouvoir les interroger. Ils sont actuellement retranchés dans un caveau du 
vieux cimetière, protégés par une porte massive. Le couloir donnant accès à la salle où ils se trouvent n'est pas très 
large et nous pourrons tenir à deux de front au maximum. »

Il fit une pause et observa ses hommes. Tous étaient aguerris et malgré l'heure tardive en pleine forme et prêts à en 
découdre. L'équipe qu'il avait devant lui avait l'habitude de ce genre d'interventions musclées depuis que les lois sur 
la prohibition avaient été promulguées. Il avait sorti son carnet de notes où il dessina un plan succinct des lieux 
qu'il fit passer aux hommes.

« Je veux veux quatre hommes dehors pour sécuriser le périmètre, un de plus restera à l'entrée du caveau. Le reste 
avec moi dans le couloir. Soyez sur vos gardes, on ne sait pas à quoi on doit s'attendre, alors préparons nous au pire. 
Nous garderons le silence le plus complet jusqu'à ce que nous soyons devant la porte. Des questions ? »

Aucun homme ne broncha. On pouvait lire dans leurs yeux la détermination à mener à bien cette opération.

Dans le cimetière tout était calme. Pas un bruit ne venait déranger la quiétude des morts. L'étrange lumière verte 
avait disparu. Rien ne laissait imaginer que quelques minutes plus tôt des évènements anormaux s'y déroulaient. Les 
policiers investirent le caveau désormais désert. Aucune torche ne brulait dans le couloir. L'odeur toujours présente 
obligeait les hommes à se protéger tant bien que mal le nez avec un tissu.

Hefat commençait à se demander si il n'avait pas rêvé lorsque qu'après s'être annoncés et n'avoir obtenu de réponse le 
bélier défonça la serrure de la lourde porte de bois.

Le spectacle qui attendait les policiers était atroce. À la lueur des lampes électrique se dévoilait un autel maculé 
d'une substance sombre et poisseuse. Des signes cabalistiques étaient gravés dans la pierre et sur les mur de la salle 
carrée d'environ dix mètres de côtés. La puanteur épouvantable prenait à la gorge. L'un des hommes s'appuya contre un 
mur et rendit son repas dans des gargouillis liquides.

Personne. La pièce était vide de toute vie. Les hommes médusés détaillaient la pièce avec la peur dans le regard. Ils 
ne s'attendaient pas à se trouver sur les lieux de ce qui paraissait être une cérémonie sacrificielle en l'honneur 
d'une divinité avide de sang et de violence. L'inspecteur Hefat refusait de croire qu'il était arrivé trop tard pour 
sauver la pauvre victime de ces pratiques impies. Il s'effondra dans un coin et des larmes vinrent mouiller son visage 
pendant que ses hommes exploraient la pièce et découvraient une alcôve où trônait la statue d'une abjecte créature dont 
la tête n'était qu'un amas de tentacules.

La journée suivante fut consacrée à l'inventaire complet de ce qui se trouvait dans le sinistre caveau. On y retrouva 
tout un attirail visiblement destiné à exécuter des sacrifices impies et à pratiquer des rituels d'invocations atroces. 
Dans une cache secrète du mur, sous l'alcôve de l'ignoble statue on retrouva un livre monstrueux. Il fut immédiatement 
envoyé à l'Université Miskatonic à Arkham où il fût formellement attesté qu'il s'agissait d'une copie en parfait état 
du 
\emph{Necronomicon} de l'Arabe fou Abdul Alhazred.

D'après des architectes mandatés pour l'enquête, le caveau avait été construit bien avant la découverte du Nouveau 
Monde par Christophe Colomb. Certains avançaient même que l'endroit datait d'avant même l'âge de l'Homme Moderne ce qui 
était impossible mais démenti par les fossiles qui furent retrouvés dans une pièce attenante.

L'inspecteur eut du mal à dormir dans les jours qui suivirent. Ses nuit étaient peuplées par les cris entendus en cette 
sinistre nuit du 31 octobre. Des cauchemars de plus en plus vivants où se mouvaient des créatures venues du 
fond des âges dans des lieux à la géométrie absurde et impossible hantaient son sommeil.

\fancybreak{$* * *$}

L'annonce dans la rubrique nécrologique du \emph{Post} ne faisait que quelques lignes.

\emph{L'inspecteur John Hefat a choisi de mettre fin à ses jours à l'âge de 58 ans après 40 ans de bons et loyaux 
services dans les forces de l'ordre. La cérémonie se déroulera dans la plus stricte intimité le dimanche 15 novembre 
1925 en la Basilique Saint Georges. Les pensées de sa famille et de ses collègues l'accompagnent.}
\end{document}