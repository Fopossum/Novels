\chapter{Sans nom}

« Une vache dans un pré. Un ciel d'été. Une bibliothèque. Un tas de spaghettis ? Non ? Mais en réalité c'est quoi ce
truc là ? »

L'espèce de casque autour de ma tête bourdonnait. Les écrans placés devant mes yeux représentaient des animaux, des 
situations, des formes abstraites. Je devais les décrire avec précision, même si quelques fois cela relevait de la
gageure.

« Essayez de rester concentré sur les planches M. Tersant. Si nous voulons que le diagnostic soit le plus précis
possible, il nous faut récupérer le maximum d'informations.

-- Vous êtes marrant. Vous me passez devant les yeux des choses quelques fois trop abstraites que je n'arrive même pas à
comprendre.

-- C'est tout à fait normal. Nous avons aussi besoin de ce genre de choses pour cartographier votre réseau neuronal et
comprendre si l'implant pose véritablement un problème ou si vos symptômes viennent d'ailleurs. »

J'étais déjà passé par la case scanner et IRM. Nous avions aussi testé les différentes fonctionnalités de l'implant,
enregistrement, transmission et rediffusion et tout s'était passé normalement. J'étais d'ailleurs étonné de la facilité
avec laquelle je m'y étais fait finalement. Normalement cet examen devait être le dernier que je devais subir avant 
d'avoir, je l'espérais, une réponse. Cela faisait trop longtemps que Bob -- j'avais fini par le baptiser comme ça -- me
tenait compagnie et me narguait de son petit sourire narquois. La plupart du temps, il n'était pas là, bien 
heureusement. Mais il avait cette fâcheuse tendance à apparaître aux moments les plus inopportuns, lors des briefings 
ou lorsque je devais me concentrer sur le dossier. Il lui était même arrivé une ou deux fois de venir me tenir 
compagnie sous la douche. Un moment assez… Gênant.

J'étais en train de devenir dingue.

« Monsieur Tersant, nous en avons terminé. Nous allons très vite vous rappeler pour vous communiquer les résultats. »

L'infirmière retira le casque de ma tête en m'adressant un grand sourire. Mes yeux eurent du mal à s'habituer à la
lumière crue qui tombait du plafond et je sentis des larmes brouiller ma vision. Tout en me tendant un mouchoir elle me
dit « Ne vous en faites pas, c'est tout à fait normal. Vos yeux vont vite s'habituer. »

Tout en essuyant mes larmes j'observais à nouveau l'endroit où je me trouvais. Des murs pavés de carreaux d'un blanc
éclatant, des machines ronronnantes et bourdonnantes d'activité électrique, des écrans où se succédaient des indications
qui n'avaient pour moi aucun sens -- \emph{Si tu étais là mon amour, tu comprendrais tout ce qui s'affiche ici, tu
pourrais me dire ce qui ne va pas chez moi, tu pourrais me rassurer} -- toutes ces personnes qui virevoltaient d'une
machine à l'autre, caressant de leurs doigts gantés des claviers et des écrans. J'assistais à un ballet technologique à 
la chorégraphie complexe qui se déroulait sous mes yeux encore embués.

Je descendis du fauteuil sur lequel j'avais été installé et me dirigeai vers la porte. Soudain, une image me traversa
l'esprit. Je me retournai et observait un peu plus la pièce. J'avais en face de moi la copie quasi conforme de la pièce
où devaient être maintenues les victimes dont les photos commençaient à remplir le dossier. Peut-être pas dans les
dimensions mais dans l'esprit. Il me fallait chercher si des hôpitaux ou des cliniques désaffectés se trouvaient dans la
région. Voire même des cliques vétérinaires ou des morgues.

La description de l'endroit qui nous avait été faire par Diaz, et le peu de décors présent sur les photos qui nous
étaient parvenues ne permettaient pas de deviner la taille de la pièce, mais, j'en aurais mis ma main à couper, il nous
fallait rajouter ces informations dans notre recherche géographique.

L'infirmière s'était tournée vers moi et m'interrogea du regard. « Non non, ce n'est rien, rien d'important. Merci et
bonne journée » répondis-je à son regard tout en ouvrant la porte de la pièce.

Dès que je fus sorti du pavillon, j'attrapais mon téléphone et vérifiais la date du prochain rendez-vous. D'ici une 
semaine je devais rencontrer un psychiatre pour un entretien. Pour l'heure, des anxiolytiques m'avaient été prescrits et 
j'allais rajouter de nouvelles petites pilules à ma collection. Ne pas oublier de passer par la pharmacie avant de 
retourner au bureau. Le psy que j'avais croisé en coup de vent avait été très clair. Si Bob apparaissait le plus souvent 
dans les moments de tension, il fallait que je sois le plus détendu possible pour éviter au maximum qu'il ne pointe son 
nez.

J'appelais Jensain. Il décrocha à la troisième sonnerie.

« Jensain, Tersant à l'appareil. T'en es où de la délimitation de la zone géographique de recherche ?

-- Guère plus avancé qu'avant. Nous avons réduit légèrement la zone, mais rien de probant pour l'instant.

-- Dis-moi, tu pourrais essayer en rajoutant les hôpitaux, cliniques, cliniques vétérinaires et autres établissements 
de pompes funèbres ? Je suis persuadé que c'est ce genre d'ancien bâtiment que nous recherchons. Ils ont tous en commun 
des salles d'opération ou de \emph{préparation}. Et je pense que notre tueur n'a pas construit son local de ses mains 
pas plus qu'il n'a dû s'adresser à des artisans locaux pour le faire construire. Il est resté discret jusqu'à 
maintenant, aucune raison qu'il ne l'ait pas été avant.

-- Je m'y mets de suite. Mais tu sais, je ne suis pas sûr que ça change depuis la dernière fois. Jusqu'à maintenant, 
on a fait chou blanc. Au fait, le commissaire a demandé à ce que tu l'appelles dès que tu seras arrivé ici. Elle nous
a dit qu'il y avait du nouveau et est partie en trombe.

-- Noté. Je serai là dans une demie-heure environ. »

Soudain, une bouffée d'angoisse me prit à la gorge. Les hauts bâtiments d'acier brillant de l'hôpital m'écrasaient. Ils 
avaient beau être espacés sur de larges esplanades de verdure aux coins ombragés et accueillants, j'avais le sentiment 
qu'ils se rapprochaient de moi et cherchaient à m'engloutir. Je savais que c'était une illusion. Je le savais 
parfaitement, mais mon cerveau malade ne le voyait pas du tout de cet œil. Je me hâtais en direction d'un banc sous un 
saule pleureur, à côté d'un petit cours d'eau. Assis, je pris ma tête entre mes mains et essayait de refouler la 
boule qui me prenait à la gorge.

Alors que je levais les yeux, Bob, toujours muet, m'observait, accoudé à une vasque de fleurs un peu plus loin dans le 
parc. Je fermais les yeux, inspirais profondément, soufflais, inspirais, soufflais… Lorsque je rouvrit les yeux, il 
n'était plus là mais avait été remplacé par une jeune femme dont le regard semblait se poser sur moi. Sans que je ne 
l'ai invité elle s'approcha et s'assit à mes côtés.

« Ça te dérange pas ?

-- N… Non, pas du tout.

-- T'as pas une clope des fois ? »

Je sortis mon paquet et lui en tendais une.

« Ouah. Une gitane ?! Dis donc, t'es un homme un vrai toi !

-- Mouais, répondis-je maussade, on peut dire ça comme ça.

-- Et pourquoi t'es là ? J't'ai jamais vu ici. Et j'connais tout le monde ici. C'est un peu moi la patronne tu vois. »

En me disant cela, elle redressa les épaules et bomba le torse. Son index fit des va et vient entre son œil droit et 
l'horizon. Elle semblait fatiguée et excitée à la fois. Ses longs cheveux blonds étaient ramassés en un chignon qui se 
voulait strict mais qui partait dans tous les sens. Elle n'était ni belle ni laide, sans signe distinctifs si ce n'est 
une espèce d'étincelle dérangeante dans le regard qu'elle avait d'un bleu azur.

« Je viens juste passer quelques examens pour le boulot. Rien de bien méchant.

-- Ah, c'est ça. T'es un nouveau. T'es comme tous les autres. Un fou. Y'a que deux catégories de gens ici. Ceux qui 
sont fous et ceux qui vont le devenir. C'est comme ça. »

Je la regardais, interloqué. Qu'est-ce qu'elle me racontait ?

« Ben quoi ? Tu crois quoi ? Ici, les gens finissent toujours par devenir cinglés. Même si tout allait bien avant. »

Elle tira longuement sur sa clope et souffla un gros nuage de fumée lentement par le nez. Elle soulignait ses paroles 
par de grands gestes désordonnés.

« Tu vois ? Y'a tout un tas de petites bêtes qu'on peut pas voir. Et les médecins là, ils ont beau avoir toutes les 
machines sophistiquées qu'ils veulent, ils peuvent pas lutter. Parce que les bestioles, tu vois, elles sont plus fortes 
que les machines. Tu pourras faire ce que tu veux, la bestiole, elle vaincra toujours. Moi,ça va que je suis immunisée. 
C'est à cause de c'qui m'est arrivé gamine. Et c'est pour ça que c'est moi qui commande ici. Parce que moi, tu vois, je 
suis complètement saine et que je serai jamais infectée par ces saloperies. Mais ils veulent pas me laisser sortir. Ils 
disent que je suis cinglée. Mais c'est pas vrai. C'est eux les fous. Et en fait, tu sais quoi, ils ont peur ! Ils ont 
peu de ce qu'il comprennent pas. Et moi, ils me comprennent pas tu vois. »

Ses paupières s'étaient presque entièrement fermées, ne laissant apparaitre qu'un trait fin sur l'oeil. Elle s'était 
rapprochée de moi, comme pur me confier un secret de la plus grande importance.

« J'vais t'dire un truc mec. Toi aussi t'es immunisé. J'le sens. J'déconne pas. Les bestioles elles peuvent rien 
contre toi. »

Au moment ou elle terminait sa phrase, un infirmier se dirigeait vers nous. Elle ajouta rapidement :

« J't'ai rien dit mec. J't'ai juste tapé une clope. Faut pas qu'ils sachent les autres. Faut pas qu'ils sachent que je 
ne suis pas la seule. Surtout pas. Sinon, on est morts. »

L'infirmier, une espèce de montagne de muscles de près de deux mètres à l'air doux comme un nounours m'interpella :

« J'espère qu'elle ne vous a pas dérangé. Sophie à tendance à quelque fois discuter un peu trop -- regard chargé de 
reproches en direction de la jeune femme -- et à importuner les gens de passage.

-- Non non, pas de problème. Je lui ai juste offert une cigarette.

-- Ah c'est bien elle ça. Toujours à aller taxer les gens. Sophie, tu retournes au pavillon, tu sais que tu n'as pas le 
droit de te balader seule dans le parc.

-- Mouais, répondit-elle d'un air maussade.

-- Au revoir monsieur. Merci pour la cigarette.

-- De rien, avec plaisir, répondis-je. »

Son regard si animé s'était littéralement éteint. Plus une étincelle de vie.

Je me levais à mon tour. Il fallait que je retourne vite au commissariat. J'étais déjà pas mal en retard et 
j'allais me faire souffler dans les bronches. Mais avant ça, je devais traverser une bonne partie du parc.  J'avais 
appris que souvent, les établissements psychiatriques avaient été construits hors des murs de la ville. Les gens 
« normaux » ne devaient pas risquer de rencontrer les fous. Être confronté à la folie nous ramenait trop souvent à nous 
poser des questions sur notre propre santé mentale. Il y avait toujours eu de la méfiance envers les gens donc le 
cerveau ne fonctionnait pas tout à fait comme il aurait dû.

Cette conversation surréaliste m'avait laissé un goût bizarre dans la bouche. Non pas qu'elle ait raison sur les 
« bestioles », ça non, mais plutôt sur le fait qu'il n'y ait que deux types de personnes. Les dingues et ceux qui 
allaient le devenir. Après tout, j'étais en train de lentement glisser de l'autre côté de la barrière. Qu'est-ce qui 
faisait que l'on passait définitivement de l'autre côté. Que s'était-il passé pour que notre tueur ait quitté le monde 
des sains d'esprits pour s'enfoncer dans la noirceur de ses crimes ? Et surtout pour quelle raison avait-il subitement 
décidé de nous faire parvenir les photos et les papiers des victimes ? Il s'était passé un mois depuis que Diaz avait 
disparu et s'était échappé de l'enfer où il avait été maintenu. Et pendant ce mois, rien de nouveau. Les enquêtes de 
proximité n'avait rien révélé de particulier qui pourrait nous mettre sur la trace de notre chirurgien amateur. \emph{Le 
Chirurgien}, avec une majuscule. C'est comme ça que les journaleux avaient baptisé notre homme. On avait beau essayer de 
limiter les fuites, mais l'affaire avait fini par filtrer et les journaux se délectaient de notre impuissance à traquer 
un fantôme. J'aimerais bien les voir à notre place tiens. Pas une piste à se mettre sous la dent, des dizaines de gens 
interrogés dans le cadre des enquêtes de proximité, voisins de la victime, personnes qu'ont avait fini par retrouver à 
partir des bandes de vidéo surveillance et rien. Nada. Pas le moindre début de piste. Diaz qui ne se souvenait de rien 
de plus que lors de sa première déposition.

Les recherches sur les réseau de drogues illicites n'avaient permis que de mettre sous les barreaux des petites frappes 
qui n'étaient pas liées à l'affaire. Au moins, c'était toujours ça de pris, mais ça ne nous avait pas permis d'avancer. 
La hiérarchie devenait de plus plus insistante et nous mettait la pression pour obtenir des résultats.

Nous n'avions pas avancé d'un pouce jusqu'à ce que la semaine dernière atterrisse sur mon bureau le portefeuille de 
Diaz accompagné d'une photo de lui. Personne n'était capable d'expliquer comment il avait bien pu arriver là. Dans une 
enveloppe qui m'était nommément adressée. Pas de cachet de la poste dessus. La scientifique avec décortiqué le paquet 
et n'avait pu nous donner que des informations inutiles. Enveloppe banale qui pouvait avoir été achetée dans n'importe 
quelle papéterie ou supermarché. L'encre utilisée provenait d'un bic tout ce qu'il y a de plus commun.  Pas une fibre, 
pas une empreinte. Le portefeuille s'était révélé tout aussi muet.

Il contenait les papier de Diaz mais il avait été nettoyé de fond en comble. La encore, pas de traces de fibres ou de 
pollens. Par contre, Diaz avait été formel lorsque nous lui avions présenté pour lui faire confirmer que c'était le 
sien, il était en meilleur état que la dernière fois qu'il l'avait eu en main. Cela avait été confirmé par le labo. Le 
cuir avait été nourri et recousu. Il était en ce moment même entre les mains de spécialistes des cuirs pour tenter de 
déterminer si la manière dont avait été faite la restauration pouvait nous indiquer quelque chose. Les résultats 
devaient arriver d'ici peu. Je croyais me souvenir que c'était pour aujourd'hui.

Puis le deuxième portefeuille était arrivé de la même manière que le premier. Toujours sans qu'on puisse expliquer 
comment notre homme avait pu le déposer. Nous nous étions bien sûr demandé si c'était quelqu'un du commissariat mais 
nous étions tous irréprochables. On ne devenait pas flics sans que notre passé ait été épluché en long et en travers et 
nous étions tous blancs comme neige. Du moins, à notre entrée en fonctions. Et je voyais mal comment avec nos emplois 
du temps l'un de nous pourrait avoir l'occasion de kidnapper quelqu'un et de le torturer pendant des jours sans que 
l'on ne remarque son absence. Et comme personne n'était en vacance lorsque Diaz avait disparu, cela réglait 
définitivement le problème.

La sonnerie de mon téléphone m'arracha brutalement à mes pensées. Merde, le commissaire.

« Tersant ? Qu'est-ce que vous foutez bon Dieu ?

-- Je suis encore à l'hôpital. Les examens ont duré plus longtemps que prévu. Je suis en train de marcher vers ma 
voiture.

-- Alors au lieu de marcher, courrez et rejoignez-moi illico au labo. On a du nouveau, et c'est pas 
franchement joli. »

Même pas le temps de dire quoi que ce soit, elle avait déjà raccroché. La ton de sa voix m'avait foutu la chair de 
poule. J'accélérais le pas. Un peu plus loin, Bob se marrait et se dirigeait en trottinant vers ma voiture.